%-----<>-----<>-----<>-----<>-----<>-----<>-----<>-----<>-----<>-----<>
% DONT EDIT MANUALLY THIS FILE !!!!!!
% This files automatically generated by odoc.pl from 
% source file ""
%-----<>-----<>-----<>-----<>-----<>-----<>-----<>-----<>-----<>-----<>
\index{ALE\_flag@\verb+ALE_flag+}
\item\verb+int ALE_flag+ {\rm(default=\verb|0|)}:

Flag to turn on ALE (Arbitrary Lagrangian-Eulerian) computation. 
 (found in file: \verb+advdife.cpp+)
%----<>----<>----<>----<>----<>----<>----<>----<>----<>----<>
\index{axisymmetric@\verb+axisymmetric+}
\item\verb+string axisymmetric+ {\rm(default=\verb|"none"|)}:

Add axisymmetric version for this particular elemset.
 (found in file: \verb+advdife.cpp+)
%----<>----<>----<>----<>----<>----<>----<>----<>----<>----<>
\index{compute\_dDdU\_term@\verb+compute_dDdU_term+}
\item\verb+int compute_dDdU_term+ {\rm(default=\verb|1|)}:

Compute the term non-symmetric term
 correspoding to nonlinearities in the
 diffusive Jacobian.
 (found in file: \verb+advdife.cpp+)
%----<>----<>----<>----<>----<>----<>----<>----<>----<>----<>
\index{compute\_fd\_adv\_jacobian@\verb+compute_fd_adv_jacobian+}
\item\verb+int compute_fd_adv_jacobian+ {\rm(default=\verb|0|)}:

Compute finite difference jacobian of fluxes for checking the
 analytical one. For each element the following norms are printed:
 analytical jacobian \verb+|A_a|+ , numerical jacobian \verb+|A_n|+ and the
 difference \verb+|A_a-A_n|+ . Incrementing \verb+compute_fd_adv_jacobian==1+
 increases the verbosity. If \verb+=1+ the maximum values over all the
 elemset are printed. If \verb+=2+ the errors for all elements are
 reported. Finally, if \verb+=3+ also the jacobians themselves are
 printed. For 2 and 3, if \verb+compute_fd_adv_jacobian_elem_list+ is
 set, then only those elements are printed. If
 \verb+compute_fd_adv_jacobian_rel_err_threshold+ is set then only those
 elements for which the error is greater than the given threshold
 are reported.  Also, be warned that when run in parallel, printing
 for a lot of elements in different processors may be messy.
 (found in file: \verb+advdife.cpp+)
%----<>----<>----<>----<>----<>----<>----<>----<>----<>----<>
\index{compute\_fd\_adv\_jacobian\_eps@\verb+compute_fd_adv_jacobian_eps+}
\item\verb+double compute_fd_adv_jacobian_eps+ {\rm(default=\verb|1e-4|)}:

The perturbation scale for computing the numerical jacobian
 (see \verb+compute_fd_adv_jacobian+ ).
 (found in file: \verb+advdife.cpp+)
%----<>----<>----<>----<>----<>----<>----<>----<>----<>----<>
\index{compute\_fd\_adv\_jacobian\_random@\verb+compute_fd_adv_jacobian_random+}
\item\verb+double compute_fd_adv_jacobian_random+ {\rm(default=\verb|1.0|)}:

Report jacobians on random elements (should be in range 0-1).
 (found in file: \verb+advdife.cpp+)
%----<>----<>----<>----<>----<>----<>----<>----<>----<>----<>
\index{compute\_fd\_adv\_jacobian\_rel\_err\_threshold@\verb+compute_fd_adv_jacobian_rel_err_threshold+}
\item\verb+double compute_fd_adv_jacobian_rel_err_threshold+ {\rm(default=\verb|0.|)}:

Report elements whose relative error in computing
 flux jacobians exceed these value.
 (found in file: \verb+advdife.cpp+)
%----<>----<>----<>----<>----<>----<>----<>----<>----<>----<>
\index{compute\_reactive\_terms@\verb+compute_reactive_terms+}
\item\verb+int compute_reactive_terms+ {\rm(default=\verb|1|)}:

key for computing reactive terms or not
 (found in file: \verb+advdife.cpp+)
%----<>----<>----<>----<>----<>----<>----<>----<>----<>----<>
\index{geometry@\verb+geometry+}
\item\verb+string geometry+ {\rm(default=\verb|cartesian2d|)}:

Type of element geometry to define Gauss Point data
 (found in file: \verb+advdife.cpp+)
%----<>----<>----<>----<>----<>----<>----<>----<>----<>----<>
\index{indx\_ALE\_xold@\verb+indx_ALE_xold+}
\item\verb+int indx_ALE_xold+ {\rm(default=\verb|1|)}:

Pointer to old coordinates in
 \verb+nodedata+ array excluding the first "ndim" values
 (found in file: \verb+advdife.cpp+)
%----<>----<>----<>----<>----<>----<>----<>----<>----<>----<>
\index{lumped\_mass@\verb+lumped_mass+}
\item\verb+int lumped_mass+ {\rm(default=\verb|0|)}:

Use lumped mass (used mainly to avoid oscillations for small time steps).
 (found in file: \verb+advdife.cpp+)
%----<>----<>----<>----<>----<>----<>----<>----<>----<>----<>
\index{precoflag@\verb+precoflag+}
\item\verb+int precoflag+ {\rm(default=\verb|0|)}:

Flags whether we are solving a precondioned
system with the dual time strategy
 (found in file: \verb+advdife.cpp+)
%----<>----<>----<>----<>----<>----<>----<>----<>----<>----<>
\index{shocap@\verb+shocap+}
\item\verb+double shocap+ {\rm(default=\verb|0.0|)}:

Add a shock capturing term
 (found in file: \verb+advdife.cpp+)
%----<>----<>----<>----<>----<>----<>----<>----<>----<>----<>
\index{shocap\_aniso@\verb+shocap_aniso+}
\item\verb+double shocap_aniso+ {\rm(default=\verb|0.0|)}:

Add an anisotropic shock capturing term
 (found in file: \verb+advdife.cpp+)
%----<>----<>----<>----<>----<>----<>----<>----<>----<>----<>
\index{use\_Ajac\_old@\verb+use_Ajac_old+}
\item\verb+int use_Ajac_old+ {\rm(default=\verb|0|)}:

Use the advective Jacobian in the previous time step
 for the SUPG stabilization term. This accelerates
 convergence of the Newton iteration.
 (found in file: \verb+advdife.cpp+)
%----<>----<>----<>----<>----<>----<>----<>----<>----<>----<>
\index{use\_fastmat2\_cache@\verb+use_fastmat2_cache+}
\item\verb+int use_fastmat2_cache+ {\rm(default=\verb|1|)}:

Uses operations caches for computations with the FastMat2
 library for internal computations. Note that this affects also the
 use of caches in routines like fluxes, etc...
 (found in file: \verb+advdife.cpp+)
%----<>----<>----<>----<>----<>----<>----<>----<>----<>----<>
\index{use\_GCL\_compliant@\verb+use_GCL_compliant+}
\item\verb+int use_GCL_compliant+ {\rm(default=\verb|0|)}:

Use the GCL compliant versin of the algorithm 
 (found in file: \verb+advdife.cpp+)
%----<>----<>----<>----<>----<>----<>----<>----<>----<>----<>
\index{use\_log\_vars@\verb+use_log_vars+}
\item\verb+int use_log_vars+ {\rm(default=\verb|0|)}:

Use log-vars for $k$ and $\epsilon$
 (found in file: \verb+advdife.cpp+)
%----<>----<>----<>----<>----<>----<>----<>----<>----<>----<>
\index{use\_low\_gpdata@\verb+use_low_gpdata+}
\item\verb+int use_low_gpdata+ {\rm(default=\verb|0|)}:

Use 1 Gauss point for some matrix terms.
In this application for diffusive terms grad_N_D_grad_N
and for stabilized advection terms P_supg_A_grad_N.
 (found in file: \verb+advdife.cpp+)
%----<>----<>----<>----<>----<>----<>----<>----<>----<>----<>
\index{weak\_form@\verb+weak_form+}
\item\verb+int weak_form+ {\rm(default=\verb|1|)}:

Use the weak form for the Galerkin part of the advective term.
 (found in file: \verb+advdife.cpp+)
%----<>----<>----<>----<>----<>----<>----<>----<>----<>----<>
%-----<>-----<>-----<>-----<>-----<>-----<>-----<>-----<>-----<>-----<>
% DONT EDIT MANUALLY THIS FILE !!!!!!
% This files automatically generated by odoc.pl from 
% source file ""
%-----<>-----<>-----<>-----<>-----<>-----<>-----<>-----<>-----<>-----<>
