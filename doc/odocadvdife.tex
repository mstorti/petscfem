%-----<>-----<>-----<>-----<>-----<>-----<>-----<>-----<>-----<>-----<>
% DON'T EDIT MANUALLY THIS FILE !!!!!!
% This files automatically generated by odoc.pl from 
% source file ""
%-----<>-----<>-----<>-----<>-----<>-----<>-----<>-----<>-----<>-----<>
\index{axisymmetric@\verb+axisymmetric+}
\item\verb+string axisymmetric+ {\rm(default=\verb|"none"|)}:

Add axisymmetric version for this particular elemset.
 (found in file: \verb+advdife.cpp+)
%----<>----<>----<>----<>----<>----<>----<>----<>----<>----<>
\index{beta\_supg@\verb+beta_supg+}
\item\verb+double beta_supg+ {\rm(default=\verb|1.|)}:

Weights the temporal term with $N+\beta P$, i.e.
$\beta=0$ is equivalent to weight the temporal term a la
Galerkin and $\beta=1$ is equivalent to do the consistent SUPG weighting.
 (found in file: \verb+advdife.cpp+)
%----<>----<>----<>----<>----<>----<>----<>----<>----<>----<>
\index{compute\_fd\_adv\_jacobian@\verb+compute_fd_adv_jacobian+}
\item\verb+int compute_fd_adv_jacobian+ {\rm(default=\verb|0|)}:

Compute finite difference jacobian of fluxes for checking the
 analytical one. For each element the following norms are printed:
 analytical jacobian \verb+|A_a|+ , numerical jacobian \verb+|A_n|+ and the
 difference \verb+|A_a-A_n|+ . Incrementing \verb+compute_fd_adv_jacobian==1+
 increases the verbosity. If \verb+=1+ the maximum values over all the
 elemset are printed. If \verb+=2+ the errors for all elements are
 reported. Finally, if \verb+=3+ also the jacobians themselves are
 printed. For 2 and 3, if \verb+compute_fd_adv_jacobian_elem_list+ is
 set, then only those elements are printed. If
 \verb+compute_fd_adv_jacobian_rel_err_threshold+ is set then only those
 elements for which the error is greater than the given threshold
 are reported.  Also, be warned that when run in parallel, printing
 for a lot of elements in different processors may be messy.
 (found in file: \verb+advdife.cpp+)
%----<>----<>----<>----<>----<>----<>----<>----<>----<>----<>
\index{compute\_fd\_adv\_jacobian\_eps@\verb+compute_fd_adv_jacobian_eps+}
\item\verb+double compute_fd_adv_jacobian_eps+ {\rm(default=\verb|1e-4|)}:

The perturbation scale for computing the numerical jacobian
 (see \verb+compute_fd_adv_jacobian+ ).
 (found in file: \verb+advdife.cpp+)
%----<>----<>----<>----<>----<>----<>----<>----<>----<>----<>
\index{compute\_fd\_adv\_jacobian\_random@\verb+compute_fd_adv_jacobian_random+}
\item\verb+double compute_fd_adv_jacobian_random+ {\rm(default=\verb|1.0|)}:

Report jacobians on random elements (should be in range 0-1).
 (found in file: \verb+advdife.cpp+)
%----<>----<>----<>----<>----<>----<>----<>----<>----<>----<>
\index{compute\_fd\_adv\_jacobian\_rel\_err\_threshold@\verb+compute_fd_adv_jacobian_rel_err_threshold+}
\item\verb+double compute_fd_adv_jacobian_rel_err_threshold+ {\rm(default=\verb|0.|)}:

Report elements whose relative error in computing
 flux jacobians exceed these value. 
 (found in file: \verb+advdife.cpp+)
%----<>----<>----<>----<>----<>----<>----<>----<>----<>----<>
\index{geometry@\verb+geometry+}
\item\verb+string geometry+ {\rm(default=\verb|cartesian2d|)}:

Type of element geometry to define Gauss Point data
 (found in file: \verb+advdife.cpp+)
%----<>----<>----<>----<>----<>----<>----<>----<>----<>----<>
\index{lumped\_mass@\verb+lumped_mass+}
\item\verb+int lumped_mass+ {\rm(default=\verb|0|)}:

Use lumped mass (used mainly to avoid oscillations for small time steps).
 (found in file: \verb+advdife.cpp+)
%----<>----<>----<>----<>----<>----<>----<>----<>----<>----<>
\index{shocap@\verb+shocap+}
\item\verb+double shocap+ {\rm(default=\verb|0.0|)}:

Add a shock capturing term
 (found in file: \verb+advdife.cpp+)
%----<>----<>----<>----<>----<>----<>----<>----<>----<>----<>
\index{use\_fastmat2\_cache@\verb+use_fastmat2_cache+}
\item\verb+int use_fastmat2_cache+ {\rm(default=\verb|1|)}:

Uses operations caches for computations with the FastMat2
 library for internal computations. Note that this affects also the
 use of caches in routines like fluxes, etc... 
 (found in file: \verb+advdife.cpp+)
%----<>----<>----<>----<>----<>----<>----<>----<>----<>----<>
\index{use\_log\_vars@\verb+use_log_vars+}
\item\verb+int use_log_vars+ {\rm(default=\verb|0|)}:

Use log-vars for $k$ and $\epsilon$
 (found in file: \verb+advdife.cpp+)
%----<>----<>----<>----<>----<>----<>----<>----<>----<>----<>
\index{weak\_form@\verb+weak_form+}
\item\verb+int weak_form+ {\rm(default=\verb|1|)}:

Use the weak form for the Galerkin part of the advective term.
 (found in file: \verb+advdife.cpp+)
%----<>----<>----<>----<>----<>----<>----<>----<>----<>----<>
%-----<>-----<>-----<>-----<>-----<>-----<>-----<>-----<>-----<>-----<>
% DON'T EDIT MANUALLY THIS FILE !!!!!!
% This files automatically generated by odoc.pl from 
% source file ""
%-----<>-----<>-----<>-----<>-----<>-----<>-----<>-----<>-----<>-----<>
