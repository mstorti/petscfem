%__INSERT_LICENSE__
\Section{The FastMat2 matrix class}

\SSection{Introduction} 

Finite element codes usually have two levels of programming.
In the outer level a large vector describes the \emph{``state''} of
the physical system. Usually this vector has as many entries as the
number of nodes times the number of fields minus the number of
fixations (i.e. Dirichlet boundary conditions). This vector can be
computed at once by assembling the right hand side and the stiffness
matrix in a linear problem, iterated in a non-linear problem or
updated at each time step through solution of a linear or non-linear
system. The point is that, at this outer level, you perform global
assemble operations that build this vector and matrices. At the inner
level, you perform a loop over all the elements in the mesh, compute
the vector and matrix contributions of each element and assemble them
in the global vector/matrix. From one application to another, the
strategy at the outer level (linear/non-linear, steady/temporal
dependent, etc...) and the physics of the problem that defines the
FEM matrices and vectors may vary. 

The FastMat2 matrix class has been designed in order to perform matrix
computations at the element level. It is assumed that we have an outer
loop (usually the loop over elements) that is executed many times, and
at each execution of the loop a series of operations are performed
with a rather reduced set of local vectors and matrices. There are
many matrix libraries but often the performance is degraded for small
dimensions. Sometimes performance is good for operations on the whole
matrices, but it is degraded for operations on a subset of the elment
of the matrices, like columns, rows, or individual elements. 
This is due to the fact that acessing a given element in the matrix
implies a certain number of arithmetic operations. Otherwise, we can
copy the row or column in an intermediate object, but then there is an
overhead due to the copy operations. 

The particularity of FastMat2 is that at the first execution of the
loop the address of the elements used in the operation are cached in
an internal object, so that in the second ans subsequent executions of
the loop the addresses are retrieved from the cache. 

\SSection{Example}\label{sec:fastmat2_example}  

Consider the following simple example. We are given a 2D finite
element composed of triangles, i.e. an array \verb+xnod+ of
$2\times\Nnod$ doubles with the node coordinates and an array
\verb+icone+ with $3\times\Nelem$ elements with node
connectivities. For each element $0<j<\Nelem$ its nodes are stored at
\verb|icone[3*j+k]| for $0\le k\le 2$. We are required to compute the
maximum and minimum value of the area of the triangles. This is a
computation which is similar to those found in FEM analysis. 
For each element we have to load the node coordinates in local vectors
$!x_1$, $!x_2$ and $x_3$, compute the vectors along the sides of the
elements $!a=!x_2-!x_1$ and $!b=!x_3-!x_1$. The area of the element
is, then, the determinant of the $2\times 2$ matrix \lfrc{$!J$} formed by
putting \lfrc{$!a$} and \lfrc{$!b$} as rows.

The FastMat2 code for the computations is like this

%\listinginput{1}{fm2ex.cpp}
\begin{verbatim}
  Chrono chrono;
  FastMat2 x(2,3,2),a(1,2),b(1,2),J(2,2,2);
  FastMatCacheList cache_list;
  FastMat2::activate_cache(&cache_list);
  // Compute area of elements
  chrono.start();
  for (int ie=0; ie<nelem; ie++) {
    FastMat2::reset_cache();
    for (int k=1; k<=3; k++) {
      int node = ICONE(ie,k-1);
      x.ir(1,k).set(&XNOD(node-1,0)).rs();
    }
    x.rs();
    a.set(x.ir(1,2));
    a.rest(x.ir(1,1));

    b.set(x.ir(1,3));
    b.rest(x.ir(1,1));

    J.ir(1,1).set(a);
    J.ir(1,2).set(b);
    
    double area = J.rs().det()/2.;
    total_area += area;
    if (ie==0) {
      minarea = area;
      maxarea = area;
    }

    if (area>maxarea) maxarea=area;
    if (area<minarea) minarea=area;
  }
  printf("total_area %g, min area %g,max area %g, ratio: %g\n",
	 total_area,minarea,maxarea,maxarea/minarea);
  printf("Total area OK? : %s\n",
	 (fabs(total_area-1)<1e-8 ? "YES" : "NOT"));
  double cpu = chrono.elapsed();
  FastMat2::print_count_statistics();
  printf("CPU: %g, number of elements: %d\n"
	 "rate: %g [sec/Me], %g Mflops\n",
	 cpu,nelem,cpu*1e6/nelem,
	 nelem*FastMat2::operation_count()/cpu/1e6);
  FastMat2::void_cache();
  FastMat2::deactivate_cache();

\end{verbatim}

Calls to the static members (those starting with \verb+FastMat2::+)
are related with the caching manipulation and will be discussed
later. Matrix are dimensioned in line 2, the first argument is the
number of dimensions, and then follow the dimensions, for instance
\verb+FastMat2 x(2,3,2)+ defines a matrix which 2 indices ranging from
1 to 3, and 1 to 2 respectively. The rows of this matrix will store
the coordinates of the local nodes to the element. FastMat2 matrices
may have any number of indices. Actually the library is compiled for a
maximum number of indices (10 by default. This limit may be modified
by redefining variable {\tt\$max\_arg} in script \verb+readlist.eperl+
and recompile.) Also they can have zero dimensions, which stands for
scalars. 

\SSSection{Current Matrix view} 
In lines 9 to 12 the coordinates of the nodes are
loaded in matrix \verb+x+. The underlying philosophy in FastMat2 is
that you can set \emph{``views''} of the matrix without actually made
any copies of the underlying values. For instance the operation
\verb+x.ir(1,k)+ (for \emph{``index restriction''}) sets a view of
\verb+x+ so that index 1 is restricted to take the value $k$ reducing
in one the number of dimensions of the matrix. As \verb+x+ has two
indices, the operation \verb+x.ir(1,k)+ gives a matrix of dimension
one consisting in the $k$-th row of $x$. A call without arguments like
in \verb+x.ir()+ cancels the restriction. Also, the function \verb+rs()+
(for \emph{``reset''}) cancels the actual view. 

\SSSection{Set operations} 
The operation \verb+a.set(x.ir(1,2))+ copies the contents of the
argument \verb+x.ir(1,2)+ in \verb+a+. Also we can use \verb+x.set(y)+
with \verb+y+ a Newmat matrix (\verb+Matrix y+) or an array of doubles
(\verb+double *y+). 

\SSSection{Dimension matching} The \verb+x.set(y)+ operation requires
that \verb+x+ and \verb+y+ have the same ``viewed'' dimensions.  As the
\verb+.ir(1,2)+ operation restricts index to the value of 2,
\verb+x.ir(1,2)+ is seen as a row vector of size 2 and then can be
copied to $a$. If the ``viewed'' dimensions don't fit then an error is
issued.

\SSSection{Automatic dimensioning} 
In the example, \verb+a+ has been dimensioned at line
2, but most operations perform the dimensioning if the matrix has not
been already dimensioned. For instance, if at line 2 we would declared
\verb+FastMat2 a+ only, without specifying dimensions, then at line
14, the matrix is created and dimensioned taking the dimensions from
the argument. The same applies to \verb+set(Matrix &)+ but not to
\verb+set(double *)+ since in this last case the argument
(\verb+double *+) don't posses information about his dimensions. Other
operations that define dimensions are products and contraction
operations. 

\SSSection{Concatenation of operations} Many operations return a
reference to the matrix (return value \verb+FastMat2 &+) so that
operations may be concatenated as in \verb+A.ir(1,k).ir(2,j)+.

\SSection{Caching the adresses used in the operations} 

If caching is not used the performance of the library is poor,
typically one to two orders of magnitude slower than the cached
version, and the cached version is very fast, in the sense that almost
all the CPU time us spent in performing multiplications and additions,
and negligible CPU time is spent in auxiliary operations. 

The idea with caches is that they are objects
(\verb+class FastMatCache+) that store the adresses needed for the
current operation. In the first pass through the body of the loop a
cache object is created for each of the operations, and stored in a
list of class \verb+FastMatCacheList+. This list is basically an STL
vector of pointers to cache objects. When the body of the loop is
executed the second time and the following, the adresses are not
needed to be recomputed but they are read from the cache instead. The
use of the cache is rather automatic and requires little intervention
by the user but in some cases the position in the cache-list can get
out of sync with respect to the execution of the operations and severe
errors may occur.

The typical use can be seen in the example shown in section
\S\ref{sec:fastmat2_example}. First we have to create a
\verb+FastMatCacheList+ object as shown in line 3 and activate it with
as in line 4. The outer loop here is the loop over elements. For the
second and following executions of the body of the loop, you must
``rewind'' the cache-list, this is done in line 8 with the
\verb+FastMat2::reset_cache()+ operation (see figure~\ref{fg:cache}).

The \verb+deactivate_cache()+ call at line 43 causes that subsequent
operations after this line will not be cached. This call is required
if subsequent calls to FastMat2 cached operations will be
executed. Otherwise there will be an error since those posterior calls
will not have a corresponding cache in the cache-list. 

The \verb+void_cache()+ call deletes the cache claiming the space used
by it. It's not required but it is a good idea in order to save memory
space. It may be required if the loop will be called with other
dimensions. For instance a FEM loop may be called first for triangles
and then for quadrangles, and in this two calls the dimensions of the
involved matrices are different. 

The general layout of a code section which uses caching is like this
%
\begin{verbatim}
...           // Initialization. Not cached operations. 

FastMatCacheList cache_list;           // Cache-list object
FastMat2::activate_cache(&cache_list); // activates the cache

for (j=0; j<N; j++) {    // Outer loop. N very large number

  FastMat2::reset_cache();            // Rewinds the cache list
          
  op_1;
  op_2;
  op_3;                               // Cached operations 
  ...                                       
  op_N;
  
   
}                        // End of outer loop 

FastMat2::void_cache();               // free memory
FastMat2::deactivate_cache();         // deactivates cache
\end{verbatim}

\begin{figure*}[htb]
\centerline{\includegraphics{./OBJ/cache.\EPSPDF}}
\caption{Cache of operations for linear segment of code}
\label{fg:cache}
\end{figure*}
 
\SSSection{Branching} 

Caching \{ is straightforward if the sequence of operations are executed
linearly, with the same local variables, without branching or looping.
Special operations have to be executed if there are branching
conditions (\verb+if+ sentences) that alter the order of execution of
the operations in the loop. For instance an ``if'' sequence like this
%
\begin{alltt}
\allttbraces%
....
op_prev_1;               // operations previous to the branch
op_prev_2;
...
op_prev_k;
if (\Arg{condition 0}) \{
  op_0_1;
  op_0_2;                // conditional block for branch 0
  ...
  op_0_N;
\} else if (\Arg{condition 1}) \{
  op_1_1;
  op_1_2;                // conditional block for branch 1
  ...
  op_1_N;
\} else \{
  op_1_1;
  op_1_2;               // conditional block for the `else' branch
  ...
  op_1_N;
\}
....
op_pos_1;               // operations posterior to the branch
op_pos_2;
...
op_pos_k;
\end{alltt}
%
If branch '0' is followed in the first execution of the block, then
the cache will look like that shown in figure~\ref{fg:branchnotok}.
%
\begin{figure*}[ht]
\centerline{\includegraphics[height=0.7\hsize]{./OBJ/branch.\EPSPDF}}
\caption{Cache list produced when branch 0 is chosen.}
\label{fg:branch}
\end{figure*}
%
\begin{figure*}[ht]
\centerline{\includegraphics[height=0.7\hsize]{./OBJ/branchnotok.\EPSPDF}}
\caption{Cache list produced when branch 0 is chosen.}
\label{fg:branchnotok}
\end{figure*}
%
When in a subsequent execution of the loops another branch is chosen,
say branch 1, then when reading trying to execute operation
\verb+op_1_1+ the library will find in the cache list a cache
corresponding to operation \verb+op_0_1+. 

This is solved by creating \emph{``branch points''} in the cache list
and choosing the appropriate branch as shown in the following
code (see figure~\ref{fg:branch}):
%
\begin{alltt}
\allttbraces%
....
op_prev_1;               // operations previous to the branch
op_prev_2;
...
op_prev_k;
FastMat2::branch();
if (\Arg{condition 0}) \{
  FastMat2::choose(0);
  op_0_1;
  op_0_2;                // conditional block for branch 0
  ...
  op_0_N;
\} else if (\Arg{condition 1}) \{
  FastMat2::choose(1);
  op_1_1;
  op_1_2;                // conditional block for branch 1
  ...
  op_1_N;
\} else \{
  FastMat2::choose(\Arg{N});
  op_e_1;
  op_e_2;               // conditional block for the `else' branch
  ...
  op_e_N;
\}
FastMat2::leave();
....
op_prev_1;               // operations posterior to the branch
op_prev_2;
...
op_prev_k;
\end{alltt}
%

The \verb+branch()+ call tells the library that several branchs will
start from there and a branch point is created. Then each conditional
block code must start with
\begin{alltt}choose(\Arg{j})\end{alltt} where \Arg{j} is a number that
must be unique among all other branches. Finally when leaving all the
branches we must call \verb+leave()+ in order to tell the library that
the mainstream of the cache list must be retaken. 
Branches can be nested at any level. 

The call to branching is not needed if the ``execution path'' is the
same for all executions of the loop.  This usually happens when the
condition refer to some global option that is uniform over all
elements.  For instance if branch '0' corresponds to ``include
turbulence model A'' and branch '0' to model B, then the same branch
is executed for all the elements and there is no need to call the
static functions.

\SSSection{Loops executed a non constant number of times}\label{sec:loop}  

Another special case is when there are loops inside the body of the
outer loop.  Note that no special branching is needed in general if
the loop is executed a fixed number of times, since the sequence of
operations is not altered from one execution to another.  
For instance consider the following piece of code
%
\begin{verbatim}
//   Case A. Inner code executed a fixed number of times
...
for (int k=0; k<N;k++) {              // N very large - Outer loop 
  block_before;
  for (int ll=0; ll<3; ll++) {
    inner_block;                      // Operations that act on the
                                      // same matrices.
  }
  block_after;
}
\end{verbatim}
%
Then the cache list generated in the first execution of the loop will
be
%
\begin{verbatim}
block_before
inner_block
inner_block
inner_block
block_after
\end{verbatim}
%
and this is OK since the number of times \verb+inner_block+ is
executed is always the same.  If the operations that are performed
inside the loop are the the same for all executions of the loop but
are executed an irregular number of times, then we can use a sequence
as follows
%
\begin{verbatim}
//   Case B. Inner code executed a variable number of times
FastMatCachePosition cp1;
...
for (int k=0; k<N;k++) {              // N very large - Outer loop 
  block_before
  FastMat2::get_cache_position(cp1);
  int n=irand(1,5);
  for (int ll=0; ll<n; ll++) {
    FastMat2::jump_to(cp1);
    inner_block;                      // Operations that act on the
                                      // same matrices.
  }
  FastMat2::resync_was_cached();
  block_after;
\end{verbatim}
%
Here the number of times the inner block is executed may vary randomly
from 1 to 5.  (\verb+irand(m,n)+ returns an integer number randomly
distributed between \verb+m+ and \verb+n+.) The
\verb+FastMatCachePosition+ class objects store the position of the
actual computation in the cache list. So that the call to
\verb+jump_to()+ at the start of the loop restarts the position in the
cache to the desired one. After leaving the loop we call to
\verb+resync_was_cached()+ in order to resync the cache list. 

This is OK if the inner loop is executed at least once the in the
first execution of the outer loop. If it happens that in the first
execution of the loop the inner loop is not entered, then the cache
list will contain \verb|block_before,block_after| and when the inner
block will be entered in subsequent executions of the loop and error
will arise since there will be missimng caches. 

To fix this we have to combine this with branching as here
%
\begin{verbatim}
FastMatCachePosition cp1;
for (int k=0; k<N;k++) {    // N very large - Outer loop 
  ....                      // Previous block
  FastMat2::branch();       // Allows conditional execution 
  FastMat2::choose(0);
  
  FastMat2::get_cache_position(cp1);  
  n=irand(0,5);
  if (k==0) n=0; // This is the critical case. 
                 // n=0 the first execution
    	         // of the loop. 
  for (int ll=0; ll<n; ll++) {
    FastMat2::jump_to(cp2);
    ...                     // inner_block
  }
  FastMat2::resync_was_cached();
  FastMat2::leave();
  ...                       // posterior block
}
\end{verbatim}
%
Off course, if the number of times the inner loop is executed is very
large, and the most time consuming part is the execution of this loop,
then it may be convenient to choose this loop as the ``outer'' one. 

%<*>---<*>---<*>---<*>---<*>---<*>---<*>---<*>---<*>---<*>---<*>
\SSSection{Masks can't traverse branches} 

Another restriction is that if branching is used, the mask that is
active at a certain \verb+FastMat2+ cached operation must be the same
independently of the path that the code have followed, for instance 
consider the following code
%
\begin{verbatim}
FastMat2 a,b,c;
// resize and set `a,b,c'
for (int j=0; j<N; j++) {
  FastMat2::branch();
  if (condition) {
    FastMat2::choose(0);
    a.is(...).ir(...);      // (B)
    // operate on masked `a'
  }
  FastMat2::leave();
  c.prod(a,b);              // (A) Wrong!! `a' may have or
                            // not the mask set
}
\end{verbatim}
%
When the code reaches the \verb+prod()+ method at line \verb+(A)+, it
can have executed or not the block inside the \verb+if+, so that the
mask set in line \verb+(B)+ may or may not be active at line
\verb+(A)+. This is clearly an error, and to avoid it the safest way is to
always reset the masks at the outlet of a branched block like in line
\verb+(C)+ as follows. 
%
\begin{verbatim}
FastMat2 a,b,c;
// resize and set `a,b,c'
for (int j=0; j<N; j++) {
  FastMat2::branch();
  if (condition) {
    FastMat2::choose(0);
    a.is(...).ir(...);      // (B)
    // operate on masked `a'
    a.rs();                 // (C)
  }
  FastMat2::leave();
  c.prod(a,b);              // (A) OK!! `a' has not mask. 
}
\end{verbatim}

%<*>---<*>---<*>---<*>---<*>---<*>---<*>---<*>---<*>---<*>---<*>
\SSSection{Efficiency} 

%Efficiency, memory required by the caches...
%Amortization time

As we mentioned before, When caching is enabled there is a gain in
speed of ten to one hundreth, and the library is very performant. Of
course, the first execution of loop is not cached and represents and
overhead that has to be amortized by executing the loop in cached mode
many times. The average speed increases when the number of executions
of the loop is increased. The cut point, i.e. the number of executions
of the loop for which the excution speed falls to one half the speed
obtained for very large number o execution is currently between 10 and
30, so that for loops larger than 200 the overhead time spent in
building the caches is negligible. 

Another issue is the memory required by the caches. First there is
some space required by the caches themselves and then, there is a copy
of the addresses of the elements involved. For instance in a
\verb+a.set(b)+ operation with \verb+a+ and \verb+b+ of size $n\times
m$, say, we have to store $2mn$ addresses. Usually this overhead in
memory requirement is negligible, since the amount of variables and
operations needed in the element routines are very small as compared
with the size of the problem itself. However, some care must be taken
when caching large inner loops. For instance in code A,
section~\S\ref{sec:loop}, if the inner loop is executed a constant,
but very large, number $M$ of times, then the amount of the cache
required is proportional to $M$. Then, even if, as discussed before,
no operations like those used in code B are required, it may be
adviceable to spend some time in insert these calls in order to reduce
memory cache and overhead time. Again, in the limit of $M$ very large,
it will be more convenient to choose this loop as the ``outer'' one.

\SSection{Synopsis of operations} 
%A synopsis of the current implemented operations can be found in the
%online reference manual.

{\raggedright 
\SSSection{One-to-one operations} These are operations that take one
FastMat2 argument as in \verb+FastMat2& add(const FastMat2 & A)+.\hfil 
The operations are from one element of \verb+A+ to the corresponding
element in \verb+*this+. 

The one-to-one operations implemented so far are
%
\begin{itemize}
\item \verb+FastMat2& set(const FastMat2 & A)+
              Copy matrix 

\item \verb+FastMat2& add(const FastMat2 & A)+
              Add matrix 

\item \verb+FastMat2& rest(const FastMat2 & A)+
              Substract a matrix 

\item \verb+FastMat2& mult(const FastMat2 & A)+ 
              Multiply (element by element) (like Matlab \verb+.*+). 

\item \verb+FastMat2& div(const FastMat2 & A)+ 
             Divide matrix (element by element, like Matlab \verb+./+). 

\item \verb+FastMat2& axpy(const FastMat2 & A, const double alpha)+ 
             Axpy operation (element by element): \verb+(*this) += alpha * A+ 
\end{itemize}

\SSSection{In-place operations}

These operations perform an action on all the elements of a matrix. 

\begin{itemize}
\item \verb+FastMat2& set(const double val=0.)+
              Sets all the element of a matrix to a constant value 
\item \verb+FastMat2& scale(const double val)+ 
              Scale by a constant value 
\item\verb+FastMat2& add(const double val)+
              Adds constant val 
\item\verb+FastMat2& fun(scalar_fun_t *function)+
              Apply a function to all elements 
\item\verb+FastMat2& fun(scalar_fun_with_args_t *function, void *user_args)+
              Apply a function with optional arguments to all elements 
\end{itemize}

\SSSection{Generic ``sum'' operations (sum over indices)}

These operations perform some operation an all the indices of a given
dimension resulting in a matrix which has less number of indices. It's
a generalization of the \verb+sum/max/min+ operations in Matlab that
returns the specified operation per columns, resulting in a row vector
result (one element per column). Here you specify a number of integer
arguments, in such a way that 
%
\begin{itemize}
\item if the $j$-th integer argument is
positive it represents the position of the index in the resulting
matrix, otherwise 
\item if the $j$-th argument is -1 then we perform the specified
operation (sum/max/min etc...) over all this index.
\end{itemize}
%
For instance if we declare \verb+FastMat2 A(4,2,2,3,3)+ then
\verb+B.sum(A,-1,2,1,-1)+ means
%
\begin{equation} 
   B_{ij} = \sum_{k=1..2,l=1..3} A_{kjil},\ \ \text{for}\ \
   i=1..3,\ j=1..2
\end{equation}
%
These operation can be extended to any binary associative
operation. So far we have implemented the following
%
\begin{itemize}
\item \verb+FastMat2& sum(const FastMat2 & A, const int m=0, ...)+
              Sum over all selected indices 
\item \verb+FastMat2& sum_square(const FastMat2 & A, const int m=0, ...) +
              Sum of squares over all selected indices 
\item \verb+FastMat2& sum_abs(const FastMat2 & A, const int m=0, ...) +
              Sum of absolute values all selected indices 
\item \verb+FastMat2& min(const FastMat2 & A, const int m=0, ...) +
              Minimum over all selected indices 
\item \verb+FastMat2& max(const FastMat2 & A, const int m=0, ...) +
              Maximum over all selected indices 
\item \verb+FastMat2& min_abs(const FastMat2 & A, const int m=0, ...) +
              Min of absolute value over all selected indices 
\item \verb+FastMat2& max_abs(const FastMat2 & A, const int m=0, ...) +
              Max of absolute value over all selected indices 
\end{itemize}

\SSSection{Sum operations over all indices}

When the sum is over all indices the resulting matrix has zero
dimensions, so that it is a scalar. You can get this scalar by
creating an auxiliar matrix (with zero dimensions) casting with
operator \verb+double()+ as in
%
\begin{verbatim}
   FastMat2 A(2,3,3),Z;

   ...  // assign elements to A

   double a = double(Z.sum(A,-1,-1));
\end{verbatim}
%
or using the \verb+get()+ function 
%
\begin{verbatim}
   double a = Z.sum(A,-1,-1).get();
\end{verbatim}
%
without arguments, which returns a double. In addition there is for
each of the previous mentioned ``generic sum'' function a companion
function that sums over all indices. The name of this function is
obtained by appending \verb+_all+ to the generic function
%
\begin{verbatim}
   double a = A.sum_square_all();
\end{verbatim}
%
The list of these functions is
%
\begin{itemize}
\item \verb+double sum_all() const +
              Sum over all indices 
\item \verb+double sum_square_all() const +
              Sum of squares over all indices 
\item \verb+double sum_abs_all() const +
              Sum of absolute values over all indices 
\item \verb+double min_all() const +
              Minimum over all indices 
\item \verb+double max_all() const +
              Maximum over all indices 
\item \verb+double min_abs_all() const +
              Minimum absolute value over all indices 
\item \verb+double max_abs_all() const +
              Maximum absolute value over all indices 
\end{itemize}

\SSSection{Export/Import operations}

These routines allow to convert matrices from or to arrays of doubles
and Newmat matrices
%
\begin{itemize}
\item \verb+FastMat2& set(const Matrix & A) +
              Copies to argument from Newmat matrix 
\item \verb+FastMat2& set(const double *a) +
              Copy from array of doubles 
\item \verb+const FastMat2& export(double *a) const +
              exports to a double vector 
\item \verb+FastMat2& export(double *a) +
              exports to a double vector 
\item \verb+const FastMat2& export(Matrix & A) const +
              Exports to a Newmat matrix 
\item \verb+const FastMat2& export(Matrix & A) const +
              Exports to a Newmat matrix 
\end{itemize}

\SSSection{Static cache operations}

These routines control the use of the cache list.

\begin{itemize}
\item \verb+static void activate_cache(FastMatCacheList *cache_list_=NULL) +
              Activates use of the cache 
\item \verb+static(void) +
              Deactivates use of the cache 
\item \verb+static void reset_cache(void) +
              Resets the cache 
\item \verb+static void void_cache(void) +
              Voids the cache 
\item \verb+static void branch(void) +
              Creates a branch point 
\item \verb+static void choose(const int j) +
              Follows a branch 
\item \verb+static void leave(void) +
              Leaves the current branch 
\item \verb+static double operation_count(void) +
              Computes the total number of operations in the cache list 
\item \verb+static void print_count_statistics() +
              Print statistics about the number of operations of
                each type in the current cache list 
\end{itemize}
} % end of \raggedright

% \SSection{Caching} 

% This is rather straight-forward when the matrix operations are
% performed in all the execution of the loop in the same maner. When
% there is some

% Local Variables: *
% mode: latex *
% tex-main-file: "petscfem.tex" *
% End: *
