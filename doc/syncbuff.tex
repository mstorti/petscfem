%__INSERT_LICENSE__
\Section{Synchronized buffer}

One difficult task in parallel programming is printing from the slave
nodes. In MPI, in general,it is not guaranteed that printing from the
nodes is possible and in the MPICH implementation output from the
nodes get all mixed and scrambled.

PETSc provides a functions in order to facilitate this task. 
The user can call \verb+PetscSynchronizedPrintf(...)+ as many times as
he wants en each nodes. The output is concatenated in each node to a
buffer, and then a collective call to
\verb+PetscSynchronizedFlush(...)+ flushes all the buffers \emph{in
  order} to the standard output. There is a similar function for
files \verb+PetscSynchronizedFPrintf(...)+ but it turns out that the
flushing of standard output and files is mixed. In addition, even in
the case of writing only to standard output, the output is not sorted
properly. 

The objective of the \verb+SyncBuffer<T>+ template class and the
\verb+KeyedOutputBuffer+ class is to have a synchronized output device
that sorts the lines written by the nodes.  The idea is that one
defines a class (say \verb+KeyedObject+) that must support the
following member functions
%
\begin{verbatim}
class KeyedObject {
public:
  // Default constructor 
  KeyedObject();
  // Copy Ctor
  KeyedObject(const KeyedObject &ko);
  // Dtor.
  ~KeyedObject();
  // Used for sorting the lines
  friend int operator<(const KeyedObject& left, 
           const KeyedObject& right);
  //  Call back for the distributed container. Return
  // size of the buffer needed to store this element.
  int size_of_pack() const;
  // Effectively packs the object into the buffer, 
  // upgrading the pointer. 
  void pack(char *&buff) const;
  // Extracts the object from the buffer, upgading the buffer. 
  void unpack(const char *& buff);
  // Print the object
  void print();
};
\end{verbatim}
%
Then the user can load objects into the buffer, and finally call the
\verb+flush()+ method to dump the actual content of the buffer to the
output. The \verb+flush()+ is equivalent to 
%
\begin{itemize}
\item sending all objects to the server, deleting them from the
  original node, 
\item sorting them according to the \verb+operator<(...)+ defined, and
\item calling \verb+print(...)+ on all of them \emph{on the server.}
\end{itemize}
%
The underlying container is a list so that you can manipulate it with
the standard list accessors, but the ideal is to push elements with
\verb+push_back(KeyedObject obj)+, or pushing a clean object with
\verb+push_back()+ and then access the elements with
\verb+back()+. Note that you must implement the copy constructor for
the \verb+KeyedObject+ class. Objects with the same key are not
overwritten, i.e. if several elements with the same key are loaded on
the same or differente processors, then all of them are printed to the
output. As the sorting algorithm used is stable, objects with the same
key loaded in the same processor,  remain in the same order as were
entered. Typical usage is as follows
%
\begin{verbatim}
#include <src/syncbuff.h>

class KeyedObject {
  // define methods as declared above
};

SYNC_BUFFER_FUNCTIONS(KeyedObject);

int main() {
   SyncBuffer<KeyedObject> sb;

   // Insert objects
   sb.push_back(obj1);
   // ...
   sb.push_back(obj1);
   
   // flush the buffer. 
   sb.flush();
}
\end{verbatim}
%
The macro \verb+SYNC_BUFFER_FUNCTIONS(...)+ takes as argument the name
of the basic object class and generates a series of wrapper
functions. (This should be done in the templates themself but due to
current limitations in template specialization it has to be done
through macros.)

\SSection{A more specialized class}

A more simple class derived from \verb+SyncBuffer<...>+ has been
written. This class is based on \verb+SyncBuffer<...>+ using as
\verb+KeyedObject+ the class \verb+KeyedLine+ which has simply an
integer and a C-string. Typical usage is
%
\begin{verbatim}
#include <src/syncbuff2.h>

  KeyedOutputBuffer kbuff;
  AutoString s;

  for ( ... ) {
    s.clear();
    // Load string `s' with `cat_sprintf' functions
    // ....

    // Load in buffer
    kbuff.push(k,s);
  }

  kbuff.flush();
\end{verbatim}
%
If you want to dump the buffer on another stream, like a file for
instance, then you have to set the
\verb+static FILE *KeyedLine::output+ field of the \verb+KeyedLine+
class. Also, the \verb+static int KeyedLine::print_keys+ flag controls
whether the key number is printed at the beginning of the line or
not. For instance the following code sends output to the
\verb+output.dat+ file without key numbers. 
%
\begin{verbatim}
  FILE *out = fopen("output.dat","w");
  KeyedLine::output = out;
  KeyedLine::print_keys = 0;
  kbuff.flush();
  fclose(out);
\end{verbatim}
%
Also, the member \verb+int KeyedOutputBuffer::sort_by_key+ (default 1)
controls whether to sort by keys prior to printing, then you can set 
%
\begin{verbatim}
kbuff.sort_by_key = 0;
\end{verbatim}
%
if you want to disable sorting. 

Some notes regarding usage of this class are:

\begin{itemize}
\item You can have several \verb+SyncBuffer<..>+'s or
\verb+KeyedOutputBuffer+'s at the same time, and you can
\verb+flush(...)+ them independently. 

\item \textbf{Memory usage:} All items sent to the buffer with
  \verb+push()+ are kept in memory in a temporary buffer. When
  \verb+flush()+ is called all objects are sent to the master, printed
  and al buffers are cleared. So that you must guarantee space enough
  in memory for all this operations. 

\item \textbf{Implementation details:} Data is sent from the nodes to
  the master with point to point MPI operations, which is far more
  efficient than writing all nodes to a file via NFS. Sorting of the
  objects by key in the master is done using the \verb+sort()+
  algorithm of the \verb+list<...>+ STL container, which is $O(N\log
  N)$ operations. 
\end{itemize}

% Local Variables: *
% mode: latex *
% tex-main-file: "petscfem.tex" *
% End: *

