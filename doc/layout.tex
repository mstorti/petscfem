%__INSERT_LICENSE__
\Section{General layout of the user input data file}

\index{user input data file}
%
Input to application packages (like \verb+ns+, \verb+advective+ or
\verb+laplace+) is feed via input data files, usually with extension
\verb+.dat+. This file contains global options, node coordinates,
element connectivities, physical properties, fixations and boundary
conditions, etc... Even if the precise format of the file may be
changing it is worth to describe the general layout. 

The file is divided in sections: the 
\verb+table+ sections, the \verb+nodedata+ section, several
\verb+elemset+ sections, the \verb+fixa+ section and \verb+constraint+
section. 

Each section starts with the keyword identifying the section in a
separate line, followed by several parameters in the same
line. Follows several lines that makes the section, ending with a
terminator of the form \verb+__END_<+\emph{some-identifier}\verb+>__+,
for instance \verb+__END_HASH__+. (Note that these terminators
start and end with double underscores (\verb+__+) while single underscores are
used to separate words inside the keyword).  For instance, an elemset
section is of the form

\begin{alltt}
elemset volume_euler 4

geometry cartesian2d
ndim 2
npg 4
chunk_size 1000
lumped_mass 1
shock_capturing 1
gamma 1.4
\Arg{... other element options go here }
__END_HASH__
    1    2   81   80
    2    3   82   81
    3    4   83   82
    4    5   84   83
    5    6   85   84
    6    7   86   85
    7    8   87   86
    8    9   88   87
\Arg{... more element connectivities follow here }
__END_ELEMSET__
\end{alltt}

In this example, the keyword \verb+elemset+ is followed by the
parameters \verb+volume_euler+ that is the elemset type, and
\verb+4+ that is the number of nodes connected to each element. The
line starting with \verb+props+ 
describes some per-element quantities (more on this later, see
\S\ref{sec:getprop}). Follows the assignation of some values to
parameters for the actual elemset, for instance the value \verb+4+ is
assigned to the \verb+npg+ parameter, that is, the number of Gauss
points. The assignation of parameters ends with the \verb+__END_HASH__+
terminator. Follows the element connectivities, one per line, ending
with the terminator \verb+__END_ELEMSET__+. 

% \SSection{Tables}

\SSection{Preprocessing the user input data file}

\index{Preprocessing}
%
It's very handy to have some preprocessing capabilitites when reading
input files, for instance including files, if-else constructs, macro
definitions, inline arithmetics, etc... Some degree of preprocessing
is performed inside \pfem{} -- this includes file inclusion, skipping
comments, and continuation lines and is described in
section~\S\ref{sec:fstack}. Off course, this internal preprocessing
may be combined with any previous preprocessing package, such as
\verb+m4+ or \verb+ePerl+. In section~\S\ref{sec:fstack} we describe
internal preprocessing while in section\S\ref{sec:eperl} we describe 
preprocessing with ePerl. 

The reason to have this two mechanisms of preprocessing is the
following. Preprocessing with ePerl (or m4, or any other packages) is
very powerful and well supported, however the mechanism is to create
an intermediate file, and this file may be very large for large
problems, so that some internal preprocessing including, at least, the
file inclusion is needed. On the other hand, including all the
preprocessing capabilities of preprocessing as in ePerl is beyond the
scope of \pfem{}, so that we preferred to keep with this two levels of
preprocessing. The idea is to have a small user data file where the
strong capabilities of an external preprocessing package may be used
while performing file inclusion of very large files containing node
coordinates, element connectivities and so on, as the file is reading,
avoiding the creation of large intermediate files.  In addition, this
allows the user to choose the external preprocessing package.

\SSection{Internal preprocessing. The FileStack class}\label{sec:fstack}  

\index{preprocessing!internal}
%
This class allows reading from a set of linked files (the \pfem{}
data file including node coordinates, mesh connectivities, etc...)
with some preprocessing capabilities. Supports file inclusion,
comments and continuation lines. The preprocessing capabilities
supported in this class are the minimal ones needed in order to treat
very large files efficiently. More advanced preprocessing capabilities
will be added in a higher level layer written in Perl or similar (may
be ePerl?).

\SSSection{Syntax} 

The syntax of comments and continuation lines is borrowed from Unix
shells,

\begin{description}

\item[Comments:] From the first appearance of ``\verb+#+'' to the end
of the lines is considered a comment. 

\item[Continuation lines:] If a line ends in ``\verb+\+'',
(i.e. if backslash is the last character in the line, before newline
``\verb+^J+'') then the next line is appended to the previous one to
form a logical single line. 

\item [File inclusion: ] The directive 
%
\begin{verbatim}
__INCLUDE__ path/to/file
\end{verbatim}
%
inserts the contents of ``\verb+file+'' in directory
``\verb+path/to/+'' to this file in the actual point. Directories may
be absolute or relative (we use ``\verb+fopen+'' from
``\verb+stdio+''). File inclusion may be recursive to the limit of the
quantity of files that can be kept open simultaneously by the system.

\item [Echo control: ] The directives \verb+__ECHO_ON__+,
\verb+__ECHO_OFF__+ controls whether the lines read from input should
be copied to the output. Usually one may be interested in copying some
part of the input to the output in order to remember the parameters of
the run. As implemented so far, this feature is recursive so that if
included files (with the internal preprocessing, i.e. with the
\verb+__INCLUDE__+ directive) will be also copied to the output, unless you
enclose the \verb+__INCLUDE__+ line itself with a \verb+__ECHO_OFF__+,
\verb+__ECHO_ON__+ pair. For instance

\begin{verbatim}
__ECHO_ON__
global_options
...             # more options here
nsaverot 50
viscosity 13.3333333333333
weak_form 0
...             # and here
__END_HASH__

nodes  2  2  3
                # do not echo coordinates
__ECHO_OFF__
__INCLUDE__ stabi.nod.tmp
__ECHO_ON__
__END_NODES__
...
\end{verbatim}

\end{description}

\SSSection{Class internals} 

As its names suggests, the class is based in a stack of files. When a 
``\verb+get_line()+'' is issued a new line is read, comments are
skipped and continuation lines are resolved. if the line is an
``\verb+__INCLUDE__+'' directive, then the current line is ``pushed''
in the stack and the new file is open and is the current file. 

\SSection{Preprocessing with ePerl}\label{sec:eperl}  

\index{preprocessing!external}
\index{ePerl}
\index{Perl}
\index{embedded Perl}
%
ePerl (for \emph{``embedded Perl''}) is a package that allows
inclusion of Perl commands within text files. Perl is the
\emph{``Practical Extraction and Report Language''}, a scripting
language optimized for scanning arbitrary text files, extracting
information from those text files, and printing reports based on that
information. For more information about ePerl see
\htmladdnormallink{http://www.engelschall.com/sw/eperl/}
{http://www.engelschall.com/sw/eperl/}, while for more information on
Perl see \htmladdnormallink{http://www.perl.com}{http://www.perl.com}.

Describing the possibilities of preprocessing with ePerl are far
beyond the scope of this manual. We will describe some basic
techniques of use in the context of writing \pfem{} user data files. 

\SSSection{Basics of ePerl}

ePerl allow you to embed Perl commands within text, enclosing them
within \verb+<:+\emph{ and  }\verb+:>+ delimiters, for instance

\begin{verbatim}
<: $pi = 2*atan2(1,0); $theta = sin($pi/4); :>
...
some text here
...

theta <: print $theta :>
beta <: print 4*$theta :>
\end{verbatim}

results, after being processed by ePerl in

\begin{verbatim}
 
...
some text here
...

theta 0.707106781186547
beta 2.82842712474619
\end{verbatim}

The basic rules are

\begin{itemize}
\item Variables start with \verb+$+ following by a C-like identifier
(alphanumeric plus underscore, case sensitive), for instance \verb+$alpha+ or
\verb+$my_variable+. 
\item Statements end in semicolon. 
\item The text inserted in place of the ePerl block is the output of
the commands inside the block. This output is done in Perl with
the \verb+print()+ function, but in ePerl there is a shortcut of the
form \verb+<:=+\emph{expression}\verb+:>+. 
\item Mathematical functions \verb+sin+, \verb+cos+, \verb+tan+, \verb+exp+,
        \verb+atan2(y,x)+ are available, as in C, powers $x^y$ are expressed
        as \verb+x**y+ (i.e. Fortran like). 
\end{itemize}

\SSSection{Variables}

You can define variables to use them after in different places, and
also in mathematical expressions

\begin{verbatim}
<: $Reynolds = 1000; $L = 2; $rho = 1.345; 
   $velocity=3.54; $Grashof=6.e4; :>
...
mu <:=$rho*$velocity*$L/$Reynolds:>
Nusselt <:=(($Reynolds*$Grashof)**0.25):>
\end{verbatim}
%
which results in
%
\begin{verbatim}

...
mu 0.0095226
Nusselt 88.0111736793393
\end{verbatim}

\SSSection{Text expansion}
 
It is common to have several lines of text that have to be repeated
several times. For instance some options that hae to be applied to
several elemsets. The trick is to assign the text to a variable via
the ``here-in document'' \verb+<<EOT+ feature and then inserting in
the appropriate places, for instance

\begin{verbatim}
<: $common_options = <<EOT;
option1   value1
option2   value 2
EOT
_:>

...

elemset type1 4
props
<:=$common_options:>
option3 value3

...
__END_ELEMSET__

elemset type2 3
props
<:=$common_options:>
option4 value4

...
__END_ELEMSET__
\end{verbatim}
%
that expands to
%
\begin{verbatim}


...

elemset type1 4
props
option1   value1
option2   value 2

option3 value3

...
__END_ELEMSET__

elemset type2 3
props
option1   value1
option2   value 2

option4 value4

...
__END_ELEMSET__
\end{verbatim}

Note the use of the underscore just before the \verb+:>+ terminator in
the first ePerl block. This tells to ePerl not to include the
semicolon terminator (see the ePerl documentation for further
details.) The terminator \verb+EOT+ stands for ``End Of Text'' and may
be replaced by any similar string. It must appear in a line by itself
at the end of the text to be included. 

\SSSection{Conditional processing}

\index{Conditional processing}
%
ePerl allows conditional processing, as with the C preprocessor, with
\verb+#if-#else-#endif+ constructs as in
%
\begin{verbatim}
<:$method = "iterative"; :>
...

#if $method eq "iterative"
maxits 100
tolerance 1e-2
#else
maxits 1
tolerance 1e-10
#endif
...
\end{verbatim}
%
expands to
%
\begin{verbatim}

...

maxits 100
tolerance 1e-2
...
\end{verbatim}

Also, lines starting with \verb+#c+ are dicarded as
comments. Conditional preprocessing and comments are enabled with the
``\verb+-P+'' flag, so that make sure you have this flag enabled when
preprocessing the \verb+.epl+ file (probably in the \verb+Makefile+
file). Note that \pfem{} comments (those starting with numeral
``\verb+#+'') may collide with the ePerl preprocessing directives, so
that when commenting out lines in \pfem{} input files it is safer to
leave a space between the ``\verb+#+'' character and the commented
text

\begin{verbatim}
# commented text          (OK)
#commented tex            (but dangerous!!)
\end{verbatim}

\SSSection{File inclusion}

\index{File inclusion}
%
In addition to the inclusion allowed in the internal preprocessor via
the \verb+__INCLUDE__+ command, ePerl has his own inclusion directive,
for instance
%
\begin{verbatim}
some text
...
#include /home/mstorti/PETSC/petscfem/doc/options.txt
...
another text
\end{verbatim}
%
and provided file \verb+options.txt contains+ 
%
\begin{verbatim}
# File options.txt
opion1 value1
option2 value2
\end{verbatim}
%
then the previous block expands to
%
\begin{verbatim}
some text
...
# File options.txt
opion1 value1
option2 value2
...
another text
\end{verbatim}
%
Including with the internal preprocessing directive \verb+__INCLUDE__+
has the advantage of not creating an intermediate file. On the other
hand, including with the ePerl directive, allows recursive ePerl
preprocessing and more versatility in defining the inclusion path
(with the \verb+@INC+ list, see Perl documentation). 

\SSSection{Use of ePerl in makefiles}

\index{ePerl!in makefiles}
%
Usually user data files have extension \verb+.dat+. When preprocessing
with ePerl the convention is to use \verb+.epl+ suffix for the file
written by the user with ePerl commands, i.e. the files to be
preprocessed, and suffix \verb+.depl+ for the preprocessed file. 
A line in the Makefile of the form
%
\begin{verbatim}
%.depl: %.epl
         eperl -P $< > $@
\end{verbatim}
%
assures the translation when needed. 

\index{ePerlini}
\SSSection{ePerlini library} 

Some useful constants and functions are found in the file
\verb+eperlini.pl+. This may be included in the user data file with
the following line

\begin{verbatim}
<:require 'eperlini.pl':>// # Initializes ePerl 
\end{verbatim}

It includes a definition for {\tt \$PI} (=$\pi$), trigonometric and
hyperbolic functions, and others. 

A common mistake when using preprocessing packages like ePerl is to
edit manually the preprocessed file \verb+.depl+, instead to edit the 
\verb+.epl+ file. In order to avoid this we write protect the
\verb+.depl+ file, for instance the section in the Makefile is
replaced by 
%
\begin{verbatim}
%.depl: %.epl
        if [ -e $@ ] ; then chmod +w $@ ; rm $@ ; fi
        eperl -P $< > $@
        chmod -w $@
\end{verbatim}
%
In addition, inclusion of the \verb+eperlini.pl+ library inserts the
following comment in the included file
%
\begin{verbatim}
# DON'T EDIT MANUALLY THIS FILE !!!!!!
# This files automatically generated by ePerl from 
# the corresponding `.epl' file. 
\end{verbatim}

\SSSection{Errors in ePerl processing} 

If the preprocessing stage with ePerl gives some error (on
\verb+STDERR+) preprocessing is stopped no ePerl output is given. For
instance, if you include a directive like 
%
\begin{verbatim}
<:=atanh(2.):>
\end{verbatim}
%
the output looks like
%
\begin{verbatim}
ePerl:Error: Perl runtime error (interpreter rc=255)

---- Contents of STDERR channel: ---------
atanh: argument x must be |x| < 1.
------------------------------------------
\end{verbatim}
%
In such a case you have to fix the ePerl commands prior to any further
debugging of the \pfem{} run. 

\SSection{General options} 

The following options apply to all the modules. 

\SSSection{Read mesh options}

This options are read in the \verb+read_mesh()+ routine

\begin{itemize}
\input odocrmsh
\end{itemize}

\SSSection{Elemset options}\label{sec:elemset-opt}  

This options are used in the \verb+Elemset+ class

\begin{itemize}
\input odocelems
\end{itemize}

%<*>---<*>---<*>---<*>---<*>---<*>---<*>---<*>---<*>---<*>---<*> 
\SSSection{PFMat/IISDMat class options}\label{sec:iisdopt}

This options are used in the \verb+PFMat+ class

\begin{itemize}
\input odociisd
\input odociisdm
\end{itemize}

%<*>---<*>---<*>---<*>---<*>---<*>---<*>---<*>---<*>---<*>---<*> 
\SSection{Emacs tools and tips for editing data files} 

GNU Emacs (\url{http://www.gnu.org/software/emacs/}) is a powerful
text editor, that can colorize and indent text files according to the
syntax of the language you are editing. Emacs has \emph{``modes''} for
most languages (C/C++, Pascal, Fortran, Lisp, Perl, ...). We have
written a basic mode for PETSc-FEM data files that is distributed with
PETSc-FEM in the \verb+tools/petscfem.el+ Emacs Lisp file. Another
mode that can serve for colorization is the \emph{``shell-script''}
mode. In order to associate the \verb+.epl+ or \verb+.depl+ extensions
to this mode, add this to your \verb+~/.emacs+ file
%
\begin{verbatim}
(setq auto-mode-alist 
  (cons '("\\.\\(\\|d\\)epl$" . shell-script-mode) auto-mode-alist))
\end{verbatim}
\let\qq$ % Emacs is fooled by previous expression

\SSSection{Installing PETSc-FEM mode} 

In order to use the mode, you have to copy the file
\verb+tools/petscfem.el+ to somewhere accessible for Emacs
(\verb+/usr/share/emacs/site-lisp+ is a good candidate). There is also
a file \verb+tools/petscfem-init.el+ that contains some basic
configuration, you can also copy it, or insert directly its contents
into your \verb+.emacs+.
%
\begin{verbatim}
;; Load basic PETSc-FEM mode
(load-file "/path/to/petscfem/tools/petscfem.el")

;; Load `petscfem-init' or insert directly its contents
;; below and configure
(load-file "/path/to/petscfem/tools/petscfem-init.el")
\end{verbatim}
%

%<*>---<*>---<*>---<*>---<*>---<*>---<*>---<*>---<*>---<*>---<*> 
\SSSection{Copying and pasting PETSc-FEM options} 

PETSc-FEM modules have a lot of options, and is mandatory to have
fast access to all of them and to their documentation. The HTML
documentation has a list of all of them at the end of the user's
guide. For easier access there is also an info file
(\verb+doc/options.info+) that has a page for each of the
options. You can browse it with the standalone \emph{GNU Info} program
or within Emacs with the info commands. In the last case you have the
additional advantage that you can very easily find the
documentation for a given option with a couple of keystrokes and paste
options from the manual into your PETSc-FEM data file.

For jumping to the documentation for an option, put the cursor on the
option and press \verb+C-h C-i+ (that is \verb+<Ctrl-h><Ctrl-i>+, this
is the key-binding for \verb+info-lookup-symbol+). You will get in the
minibuffer something like
%
\begin{verbatim}
Describe symbol (default print_internal_loop_conv): 
\end{verbatim}
%
If you press \verb+RET+ then you jump to the corresponding page in the
info file. From there you can browse the whole manual. Pressing
\verb+s+ (\verb+Info-search+) allows to search for text in the whole
manual. When done, you can press \verb+q+ (\verb+Info-exit+) or
\verb+x+ (\verb+my-Info-bury-and-kill+ defined in
\verb+tools/petscfem.el+).

If you don't know exactly what option you are looking for, then you
can search in the manual or launch \verb+info-lookup-symbol+ with the
start of the command and then use \emph{``completion''} to finish
writing the option. 

If you want to copy some option from the info manual to the data file,
then you can use the usual keyboard or mouse \emph{copy-and-paste}
methods of Emacs. Also pressing \verb+c+
(\verb+my-Info-lookup-copy-keyword+) in the info manual copies the
name of the currently visited option to the
\emph{``kill-ring''}. Then, in the data file buffer press as usual,
\verb+C-y+ (\verb+yank+) to paste the last killed option. If you paste
several options from the manual, then you can navigate between them by
pasting the last with \verb+C-y+ and then going back and forth in the
kill ring with \verb+M-y+ (\verb+yank-forw+, usually \verb+M-+ stands
for pressing the \verb+<Alt>+ or \verb+<Escape>+ key). 

For more information, see the Emacs manual, specially the
documentation for the \verb+info+ and  \verb+info-lookup+ modes. 

% Local Variables: *
% mode: latex *
% tex-main-file: "petscfem.tex" *
% End: *
