\Section{Text hash tables}

\index{text hash tables}
%
In many cases, options are passed from the user data file to the
program in the form of small databases which are called \emph{``text
hash tables''}. This consist of several lines, each one starting with
a keyword and followed by one or several values.  When reading this
data in the \verb+read_mesh()+ routine, the program doesn't know
anything about neither the keywords nor the appropriate values for
these keywords. Just when the element is processed by the element
module via the \verb+assemble()+ function call, for instance
assembling residuals or matrices, the element routine written by the
application writer reads values from this database and apply it to the
elemset.  The text hash table is stored internally as a correspondence
between the keys and the values, both in the form of strings. The key
is the first (space delimited) token and the value is the rest of the
string, from the end of the space separator to the end of the
line. Usually the values are strings like a ``C'' identifier
(\verb+chunk_size+ for instance). As the values are stored in the form
of strings, almost any kind of values may be stored their, for
instance 
%
\begin{verbatim}
non_linear_method Newton    # string value
max_iter 10                 # integer value
viscosity 1.2e-4            # double value
\end{verbatim}
%
or lists and combinations of them. It's up
to the application writer to decode this values at the element routine
level. Several routines in the \verb+getprop+ package help in getting
this (see \S\ref{sec:getprop}).  

Some of this options are used for internal use of the \pfem{} code,
for instant \verb+chunk_size+ sets the quantity of elements that are
processed by the elemset routine at each time. 

In section \S\ref{sec:text_hash_elemset}, some specific properties of
the elemset properties text hash table are described. 

\SSection{The elemset hash table of properties}\label{sec:text_hash_elemset}  

\index{elemset properties}
%
A very common problem in FEM codes is how to pass element physical or
numerical properties (conductivities, viscosities, etc... ), to the
element routine. In \pfem{} you can pass arbitrary \emph{``per elemset''}
quantities via an \emph{``elemset properties hash table''}. This comes after the
element header as in the following example

\begin{verbatim}
elemset nsi_tet 4 fat 4   # elemset header 
props                     # per element properties (to be explained later)
#
# Elemset properties hash table
#
name My Mesh
geometry cartesian2d
ndim 2
npg 4
couple_velocity 0
weak_form 1

# physical properties
Dt .1           # time step
viscosity 1.
# next lines flags end of the properties hash table
__END_HASH__
# element connectivities
 1 3 4 2
 3 5 6 4
 5 7 8 6
...
< more element connectivities here >
...
 193 195 196 194
 195 197 198 196
 197 199 200 198
 199 201 202 200
# next lines flags end of elemset connectivities
__END_ELEMSET__
\end{verbatim}

In this example we define several properties, containing doubles
(\verb+viscosity+ and \verb+Dt+), integers (\verb+ndim+, \verb+npg+,
etc...) or strings (\verb+name+). This hash table is stored in the
form of an associative array (\emph{``hash''}) with a text key, the
first non-blank characters and a value composed of the rest of the
line. In the previous example
%
\begin{align}
  \textrm{Key: {\tt "name"}} &\to \textrm{Value: {\tt "My Mesh"}}\\
  \textrm{Key: {\tt "ndim"}} &\to \textrm{Value: {\tt "2"}}\\
  \textrm{Key: {\tt "viscosity"}} &\to \textrm{Value: {\tt "1."}}
\end{align}
%
and so on. This hash table is read in an object of type
``\verb+TextHashTable+'' without checking whether this properties apply
to the particular elemset or not. The values are stored strictly as
string, so that no check is performed on the syntax of entering double
values or integers. 

\SSection{Text hash table inclusion}

Often, we have some physical or numerical parameter that is common to
a set of elemsets, for instance gravity in shallow water, or viscosity
in Navier-Stokes.  In this case, these common properties can be put in
a separate table with a \verb+table ...+ header, and included in the
elemsets with \verb+_table_include+ directives.  The \verb+table ...+
sections (not associated to a particular elemset) have to be put in
preference before the calling elemset, for instance
%
\begin{alltt}
table steel_properties   
density 13.2
viscosity 0.001
\Arg{more steel properties here ...}
__END_HASH__

elemset nsi_tet 4
_table_include steel_properties
npg 8
weak_form 1
\Arg{more properties for this elemset here ...}
__END_HASH__
 1 3 4 2
\Arg{more element connectivities here ...}
\end{alltt}
%
Text hash tables may be recursively included to any depth. The text
hash table for an elemset may be included in other elemset 
referencing it by its \verb+name+ entry, for instance
%
\begin{alltt}
elemset nsi_tet 4
name volume_elemset_1
geometry cartesian2d
npg 4
viscosity 0.23
\Arg{more properties here ...}
__END_HASH__
1 3 4 2
\Arg{more connectivities here ...}
__END__ELEMSET__


elemset nsi_tet 4
name volume_elemset_2
_table_include volume_elemset_1 # includes properties from the
                                # previous elemset
__END_HASH__
 5 4 76 45
\Arg{more connectivities here ...}
__END__ELEMSET__

\end{alltt}

\SSection{The global table}

If one table is called \verb+global_options+ then all other hash
tables inherit their properties, without needing to explicitly include
it. For instance in this case
%
\begin{alltt}
table global_options
viscosity 0.023
tau_fac 0.5
__END_HASH__

elemset nsi_tet 4
name elemset_1
__END_HASH__
 5 4 76 45
\Arg{more connectivities here ...}
__END__ELEMSET__
\end{alltt}
%
the elemset \verb+elemset_1+ gets a value for viscosity from the
global hash table of 0.023. The \verb+global_options+ table may
include other tables as well. 

\paragraph{Note:}
In previous versions of the code, the \verb+table+ keyword may be
omitted for the \verb+global_options+ table. For instance, the
previous example can be entered as
%
\begin{alltt}
global_options
viscosity 0.023
tau_fac 0.5
__END_HASH__
\end{alltt}
%
This usage is obsolete, and will be deprecated. 

\SSection{Reading strings directly from the hash table}\label{sec:getprop}

Once inside the element routine values can be
retrieved with routines from the \verb+TextHashTable+ class, typically
\verb+get_entry+, for instance
%
\begin{verbatim}
char *geom;
thash->get_entry("geometry",geom);
\end{verbatim}
%
this returns a pointer to the internal value string ``\verb+geom+''
% fixme:= verificar 
(this is documented with the \verb+TextHashTable+ class. You can then
read values from it with routines from \verb+stdio+ (\verb+sscanf+ and
friends). You should not try to modify this strings, for instance with
the \verb+strtok()+ function.  
In that case, copy the string in a new fresh string (remember to
delete it after to avoid memory leak). In this way you can pass
arbitrary information (strings, integer, doubles) to the element
routine.

\SSection{Reading with `get\_int' and `get\_double'} 

\index{get_int@get\_int}
\index{get_double@get\_double}
%
As most of the time element properties are either of integer or double
type, two specific routines are provided ``\verb+get_int+'' and
``\verb+get_double+'', for instance
%
\begin{verbatim}
ierr = get_int(thash,"npg",&npg); 
\end{verbatim}
%
where the integer value is directly returned in ``\verb+npg+''. You
can specify also a default value and read several values at once. 

\SSection{Per element properties table}\label{sec:per_elem_prop}   

\index{per element properties table}
\index{element properties!per element}
%
Many applications need a mechanism for storing values per element, for
instance when dealing with physical properties variable varying
spatially in a continuous way. If the properties is piecewise constant,
then you can define an elemset for each constant patch, but if it
varies continuously, then you should need an elemset for each element,
which conspires with efficiency. We provide a mechanism to pass ``per
element'' values to the element routine. At the moment this is only
possible for doubles. These properties are specified in the same line
of the connectivities, for instance
%
\begin{verbatim}
elemset nsi_tet 4         # elemset header 
props   cond  cp  emiss   # name of properties to be defined "per element"
# Elemset properties hash table
geometry cartesian2d
ndim 2
npg 4
# physical properties
Dt .1           # time step
viscosity 1.
# next lines flags end of the properties hash table
__END_HASH__
# element connectivities, physical properties per element
 1 3 4 2  1.1 2.3 0.7
 3 5 6 4  1.2 2.1 0.8
 5 7 8 6  1.3 2.2 0.9
...
< more element connectivities and physical props. here >
...
 193 195 196 194  1.5 2.0 0.8
 195 197 198 196  1.6 2.1 0.7
 197 199 200 198  1.7 2.2 0.6
 199 201 202 200  1.8 2.3 0.5
# next lines flags end of elemset connectivities
__END_ELEMSET__
\end{verbatim}
%
Here we define that properties ``\verb+cond+'', ``\verb+cp+'' and
``\verb+emiss+'' are to be defined per element. We add to each element
connectivity line the three properties. There are two ways to access
these properties. The corresponding values are stored in an array
``\verb+elemprops+'' of
length $\nprops\times\Nelem$, where $\nprops$ is the number of ``per
element'' properties (3 in this example) and $\Nelem$ is the number of
elements in the mesh. Also there is a macro
``\verb+ELEMPROPS(k,iprop)+'' that allows treating it as a matrix. 
So, you can access this values with 
%
\begin{verbatim}
   cond = ELEMPROPS(k,0);  // conductivity of element k
   cp   = ELEMPROPS(k,1);  // specific heat  of element k
   emiss= ELEMPROPS(k,2);  // emissivity of element k
\end{verbatim}

\SSection{Taking values transparently from hash table or per element
   table}

With the tools described so far you can access constant properties on
one hand (the same for the whole elemset) and per element
properties. Now, given a property, you should decide whether it should
be assumed to be constant for all the elemset or whether it can be
taken ``per element''. The second is the more general case, but to
take all the possible properties as ``per element' may be too much
core memory consuming. There is then a mechanism to allow the
application writer to get physical properties without bothering of
whether the user has set them in the properties hash table or in the
per-element properties table. 

First, the application writer reserves an array of doubles large
enough to contain all the needed properties. (This doesn't scale with
mesh size so you can be generous here, or either use dynamic memory
allocation). Before entering the element loop, the macro
``\verb+DEFPROP(prop_name)+'' determines whether the property has been
passed by one or the other of the mechanisms.  This information is
stored in an integer vector ``\verb+elprpsindx[MAXPROP]+''.  Also, it
assigns a position in array ``\verb+propel+'' so that
``\verb+propel[prop_name_indx]+'' contains the given property.  Then,
once inside the element loop a call to the function
``\verb+load_props+'' loads the appropriate values on
``\verb+propel[MAXPROP]+'', independently how thay have been
defined. A typical call sequence is as follows
%
\latexonly{\catcode`\!=11}%
\begin{verbatim}
// Maximum number of properties to be loaded via load_prop
#define MAXPROP 100
int iprop=0, elprpsindx[MAXPROP]; double propel[MAXPROP];

// determine which mechanism passes `conductivity'
DEFPROP(conductivity)

// conductivity is found (after calling load_props()) in
// propel[conductivity_indx]
#define COND (propel[conductivity_indx])

// Other properties 
DEFPROP(propa)
#define PROPA (propel[propa_indx])
 
DEFPROP(propb)
#define PROPB (propel[propb_indx])
 
DEFPROP(propc)
#define PROPC (propel[propc_indx])
 
DEFPROP(propp)
#define PROPP (propel[propp_indx])

DEFPROP(propq)
#define PROPQ (propel[propq_indx])

// Total number of properties loaded
int nprops=iprop;

// Set error if maximum number of properties exceeded 
assert(nprops<=MAXPROP);

// ... code ...
  
// loop over elements
for (int k=el_start; k<=el_last; k++) {
    // check if this element is to be processed
    if (!compute_this_elem(k,this,myrank,iter_mode)) continue;

    // Load properties either from properties hash table or from
    // per element properties table
    load_props(propel,elprpsindx,nprops,&(ELEMPROPS(k,0)));

    // ... more code ...

    // use physical element property
    double veccontr += wpgdet * COND * dshapex.t() * dshapex;

    // ...
\end{verbatim}
\latexonly{\catcode`\!=\active}%

First, we allocate for 100 entries in ``\verb+elprpsindx+'' and
``\verb+propel+''  arrays, and set the counter ``\verb+prop+'' to
0. Then we call ``\verb+DEFPROP+'' for properties
``\verb+conductivity+'' and ``\verb+propa+'' thru
``\verb+propq+''. After this, we check that the maximum number of
properties to be defined is not exceeded and enter the element
loop. After checking, as usual, if the element needs processing, we
call ``\verb+load_props+'' in order to effectively load element
properties in propel and, after this, we can use them as
``\verb+propel[conductivity_indx]+'' and so on. Macro shortcuts
``\verb+COND+'', ``\verb+PROPA+'' are handy for this. 


% Local Variables: *
% mode: latex *
% tex-main-file: "petscfem.tex" *
% End: *
