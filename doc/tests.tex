%__INSERT_LICENSE__
\Section{Tests and examples} 

In this section we describe some examples that come with \pfem{}. They
are intentionally small and sometimes almost trivial, but they are
very light-weight and can be run in a short time (at most some
minutes). They are put in the \verb+test+ directory and can be run
with \verb+make tests+ command. The purpose of this tests is to check
the corect behaviour of \pfem{}, but also they can serve people to
understand the program. 

\SSection{Flow in the anular region between to cylinders}

\filehead{sector.dat}.  This example tests periodic boundary
conditions. The mesh is a sector of circular strip ($2.72 < r <
4.48$, $0 < \theta < pi/4$) and we impose. The governing equation is
$\Delta !u=0$, where $!u$ is a velocity vector of two components.  On
the internal radius we impose $!u=0$ and $!u=!t$ where $!t$ is the
tangent vector. On the radii $\theta=0$ and $\theta=\pi/4$ we impose
periodic boundary conditions so that it is close to flow in the region
between two cylinders, with the internal cylinder fixed and the
external one rotating with velocity 1. But the operator \emph{is not}
the Stokes operator. However in this case the flow for this operator
is divergence free and the gradient of pressure has no component in
the angular direction, so that the solution \emph{do} coincide with
the Stokes solution.

In the output of the test (sector.sal), we check that the solution
is aligned with the angular direction $u_x=-u_y$ at the outlet
section ($\theta = \pi/4$).

\SSection{Flow in a square with periodic boundary conditions}

\filehead{lap\_per.dat} 
This is similar to example ``sector.dat'' but in a region that is the
square $-1<x,y<+1$. $!u$ is imposed on all sides, $|u|=1$ and in the
tangential direction in the counter-clockwise sense, i.e.
$!u=(0,\pm1)$ in $x=\pm1$, $!u=(\mp1,0)$ in $y=\pm1$. By symmetry we
can solve it in $\sfr14$-th of the domain, i.e. in the square $x,y>0$
(We could solve it also in $\sfr18$-th of the region, the triangle
$y>0$, $y<x$, $x<1$). 

\SSection{The oscilating plate problem} 

\filehead{oscplate.dat}
This is the flow produced between two infinite plates when one is at
rest and the other oscillating with amplitude $A$ and frequency
$\omega$ (see figure~\ref{fg:oscplate}).  This serves to test a
problem with temporal dependent boundary conditions.  The problem is
one-dimensional and the resulting field is $u=0$, $p=\cnst$ and
$v=v(x)$. We set parameters lenght between plates $L=1$, viscosity
$\nu=1$ and we model the problem with a strip $0\le x \le 1$, $0\le y
\le 0.1$ and set periodic boundary conditions between $y=0.1$ and
$y=0$. At the plate at rest ($x=L$) we set $u,p=0$ and at $x=0$ and at
the moving plate ($x=0$) we set $u=\omega A \, \sin(\omega
t)$. The analytic solution can be found easily.  The
equation for $v$ is
%
\begin{equation}\label{eq:oeq}  
  \dep vt = \nu \dep {^2v}{x^2}
\end{equation}
%
with boundary conditions $v(0)=A\omega \, \sin(\omega t)$ and
$v(L)=0$. The solution can be found by searching for solutions of the
form
%
\begin{equation} 
   v = \expe{\zi\omega t+\lambda x}
\end{equation}
%
Replacing in (\ref{eq:oeq}) we arrive at the characteristic equation
%
\begin{equation} 
   \zi\omega  = \nu\lambda^2 
\end{equation}
%
whose solutions are
%
\begin{equation} 
    \lambda = \pm \frac{1+\zi}{\sqrt2} \, \sqrt{\omega/\nu} = \pm \lambda_+
\end{equation}
%
Setting the boundary conditions, the solution is
%
\begin{equation} 
   v(x) = \Imag{\expe{\zi\omega t} \, 
       \frac{ \expe{\lambda_+(x-L)} - \expe{-\lambda_+(x-L)}}
            { \expe{-\lambda_+L} - \expe{\lambda_+L}}}
\end{equation}
%
In the example we set $\nu=100$, $\omega=2000\pi\approx 6283$. We
choose to have 16 time steps in one period so that $Dt =
\nexp{6.2}{-5}$. The resulting profile velocity is compared with the
analytical one in figure~\ref{fg:oscplsol}.4s

\begin{figure*}[htb]
%\centerline{\includegraphics{./OBJ/oscplate.\EPSPDF}}
\caption{Oscillating plates.}
\label{fg:oscplate}
\end{figure*}
 
\begin{figure*}[htb]
%\centerline{\includegraphics{./OBJ/oscplsol.\EPSPDF}}
\caption{Velocity profile for the oscilating plate problem.}
\label{fg:oscplsol}
\end{figure*}

\SSection{Linear advection-diffusion in a rectangle}

\filehead{sine.epl}
This is an example for testing the \verb+advdif+ module. The governing
equations are
%
\begin{equation}
\begin{aligned}
     &\dep\phi t + u\dep\phi x = 0, \text{ in } 0\le x\le L_x, |y|<\infty,t>0\\
     &\phi = A \, \cos(\omega t)\, \cos(k y), \text{ at } x=0,t>0\\
     &\dep\phi n = 0, \text{ at } x=L_x, t>0\\
     &\phi = 0, \text{ at } 0\le x\le L_x, |y|<\infty,t=0
\end{aligned}
\end{equation}
%
As the problem is perdiodic in the \(y\) direction we can restrict the
analysis to one quart wave-length, i.e. if \(\lambda_y=2\pi/k\),
\(L_y=\lambda/4\) then the above problem is equivalent to
%
\begin{equation}\label{eq:advdif-1}  
\begin{aligned}
     \dep\phi t + u\dep\phi x = D \Delta\phi,& \text{ in } 0\le x\le L_x,\, 0<y<L_y,t>0\\
     \phi = A \, \cos(\omega t)\, \cos (k y),& \text{ at } x=0,t>0\\
     \dep\phi n= 0,& \text{ at } x=L_x, t>0\\
     \phi = 0,& \text{ in } 0\le x\le L_x,\, 0<y<L_y,t>0\\
     \phi = 0,&  \text{ at } y=L_y\\
     \dep\phi n = 0,&  \text{ at } y=0
\end{aligned}
\end{equation}
%
We can find the solution proposing a solution of the form
%
\[ \phi \propto \expe{\beta x +\zi \omega t)} \]
%
Replacing in (\ref{eq:advdif-1}) we arrive to a characteristic
equation in \(k_x\) of the form
%
\begin{equation} 
   \zi \, \omega +\beta u = -k_y^2 +\beta^2
\end{equation}
%
This is a quadratic equation in $\beta$ and has two roots, say
$\beta-{12}$. The general solution is
%
\[ \phi(x,y,t) = \Real{\LL(c_1 \expe{\beta_1 x} 
         +c_2 \expe{\beta_2 x}\RR)   \expe{\zi \omega t}} \]
  

% Local Variables: *
% mode: latex *
% tex-main-file: "petscfem.tex" *
% End: *
