% $Id: advdif2.tex,v 1.2 2004/01/18 23:25:23 mstorti Exp $
\SSection{The Hydrological Model (cont.).} \label{sec:hydro2}
%<*>---<*>---<*>---<*>---<*>---<*>---<*>---<*>---<*>---<*>---<*>

The implemented code solves the problem of subsurface flow in a free
aquifer, coupled with a surface net of 2D or 1D streams (\emph{``2D
Saint-Venant Model''}, 2DSVM, \emph{``1D Saint-Venant Model''}, 1DSVM,
and \emph{``Kinematic Wave model''}, KWM).  To model such system three
element sets must be used: an {\sl{aquifer}} system representing the
subsurface aquifer, a 2D or 1D (depending on the chosen model)
{\sl{stream}} element set representing the stream and a 2D or 1D
{\sl{stream loss}} element set representing the losses from the stream
to the aquifer (or vice versa).

\subsection{Subsurface Flow.}
The {\sl{aquifer}} element set is 2D linear triangle or quadrangle
elements. A per-node property $\eta$ represents
the height of the aquifer bottom to a given datum. The corresponding
unknown for each node is the piezometric height or the level of the
freatic surface at that point $\phi$.\\ The equation for the
aquifer integrated in the vertical direction is
%
\begin{equation} \label{aquifereq}
  \dep{}{t} \LL(S(\phi-\eta)\phi\RR) =
  \nabla\cdot(K(\phi-\eta)\nabla\phi) + \sum G_a, ~~~ \hbox{on $\Omega_{aq} \times
  (0,t]$},
\end{equation}
%
where $\Omega_{aq}$ is the aquifer domain, $S$ is the storativity, $K$
is the hydraulic conductivity and $G_a$ is a source term, due to rain,
losses from streams or other aquifers.

\subsection{Surface Flow.}
\subsubsection{2D Saint-Venant Model.}
The {\sl{stream}} element set represents a 2D or 1D stream of
water. It has its own nodes, separated from the aquifer nodes, whose
coordinates must coincide with some corresponding node in the
aquifer. A constant per node field represents the stream bottom height
$h_b$, with reference to the datum. That is why, normally, we have two
coordinates and the stream bottom height for each node.\\  The equations
for the 2D Saint-Venant open channel flow are the well known mass and
momentum conservation equations integrated in the vertical
direction. If we write this equations in the conservation matrix form,
we have
\begin{equation} \label{advsys}
\dep{\mbf{U}}{t}+\dep{\mbf{F}_i(\mbf{U})}{x_i}=\mbf{G}_i(\mbf{U}),
~~~~~ i=1,..,3, ~~~~ \hbox{on $\Omega_{st} \times (0,t]$},
\end{equation}
where $\Omega_{st}$ is the stream domain, $\mbf{U}=(h,hw,hv)^T$ is the
state vector and the advective flux functions in \refecu{advsys} are
\begin{equation} \label{fluxfun}
\begin{split}
\mbf{F}_1(\mbf{U}) &= (hw, hw^2+g \frac{h^2}{2}, hwv)^T,\\
\mbf{F}_2(\mbf{U}) &= (hv, hwv, hv^2+g \frac{h^2}{2})^T,
\end{split}
\end{equation}
where $h$ is the height of the water in the channel with respect to
the channel bottom, $\mbf{u}=(w,v)^T$ is the velocity vector and $g$
is the acceleration due to gravity. As in \refecu{aquifereq}, $G_s$
represents the gain (or loss) of the river, the source term is
\begin{equation} \label{sourcet}
\mbf{G}(\mbf{U}) =(G_s,~
gh(S_{0x}-S_{fx})+f_chv+C_f{\varpi}_x|\varpi|,~
gh(S_{0y}-S_{fy})-f_chw+C_f{\varpi}_y|\varpi|)^T
\end{equation}
where $S_0$ is the bottom slope and $S_f$ is the slope friction.
\begin{equation} \label{frict}
\begin{split}
S_{fx}=\frac{1}{C_hh}w|\bar{u}|, ~~~ S_{fy}=\frac{1}{C_hh}v|\bar{u}| ~~~ & \hbox{Ch\`ezy model.}\\
S_{fx}=\frac{n^2}{h^{\sfr43}}w|\bar{u}|, ~~~
S_{fy}=\frac{n^2}{h^{\sfr43}}v|\bar{u}|, ~~~ & \hbox{Manning
model.}
\end{split}
\end{equation}
where $C_h$ and $n$ (the Manning roughness) are model constants.
Generally, the effect of coriolis force, related to the coriolis
factor $f_c$, must be taken in account in the case of great lakes,
wide rivers and estuaries. The coriolis factor is given by
$f_c=2\omega \sin \psi$, where $\omega$ is the rotational rate of the
earth and $\psi$ is the latitude of the area under study.  The free
surface stresses in \refecu{sourcet} are expressed as the product
between a friction coefficient and a quadratic form of the wind
velocity, $\varpi({\varpi }_x,{\varpi }_y)$, and
\begin{equation}
C_f = c_{\varpi}\frac{\rho_{air}}{\rho},
\end{equation}
where,
\begin{equation}
\begin{split}
&c_{\varpi} = \nexp{1.25}{-3}\varpi ^{-1/5} ~~~\hbox{if $| \varpi |<
1$ m/s},\\ &c_{\varpi} = \nexp{0.5}{-3}\varpi ^{1/2} ~~~\hbox{if $1$
m/s $\le | \varpi |< 15$ m/s},\\ &c_{\varpi} = \nexp{2.6}{-3}
~~~\hbox{if $| \varpi |\geq 15$ m/s},
\end{split}
\end{equation}

\subsubsection{1D Saint-Venant Model.}
When velocity variations on the channel cross section are neglected,
the flow can be treated as one dimensional. The equations of mass and
momentum conservation on a variable cross sectional stream (in
conservation form) are,
\begin{align} \label{cons}
\begin{split}
\dep{\mbf{A}(s,t)}{t}+\dep{\mbf{Q}(\mbf{A}(s,t))}{s} &= G_s(s,t),\\
\frac{1}{\mbf{A}(s,t)}\dep{\mbf{Q}}{t}+\frac{1}{\mbf{A}(s,t)}\dep{}{s}(\beta
\frac{\mbf{Q}^2}{\mbf{A}(s,t)})+g(S_0-S_f)&+\\+g\dep{h}{s}-\frac{c_{\varpi}}{\mbf{A}(s,t)}\varpi
^2 cos\alpha &= \frac{q_t}{\mbf{A}(s,t)}(v-v_t),~~~\hbox{on
$\Omega_{st} \times (0,t]$},
\end{split}
\end{align}
where $\mbf{A}$ is the cross sectional area, $\mbf{Q}$ is the discharge,
$G_s(s,t)$ represents the gain or loss of the stream (i.e. the lateral
inflow per unit length of channel), $s$ is the arc-length along the
channel, $v=\mbf{Q/A}$ the average velocity in $s$-direction, $v_t$ the
velocity component in $s$-direction of lateral flow from tributaries,
the Boussinesq coefficient $\beta=\frac{1}{v^2\mbf{A}}\int u^2 dA$
($u$ the flow velocity at a point) and $\alpha$ the wind direction
measured from a positive line tangent to $s$ in flow direction. The
bottom shear stresses are approximated by using the Ch\`ezy or Manning
equations,
\begin{align} \label{frict1d}
\begin{split}
S_{f}=\frac{v^2}{C_h^2}\frac{P(h)}{\mbf{A}(h)}, ~~~ & \hbox{Ch\`ezy model}.\\
S_{f}=\biggl( \frac{n}{a} \biggr) ^2 v^2
\frac{P^{\sfr43}(h)}{\mbf{A}^\sfr43 (h)}, ~~~ & \hbox{Manning model}.
\end{split}
\end{align}
where $P$ is the wetted perimeter of the channel and $a$ is a
conversion factor ($a=1$ for metric units).
\subsubsection{Kinematic Wave Model.}
When friction and gravity effects dominate over inertia and pressure
forces, and, if we neglect the stress due to wind blowing and the
coriolis term, the momentum equation becomes
\begin{equation}\label{mom}
S=S_f,
\end{equation}
and \refecu{cons}
\begin{equation}\label{mass}
   \dep {\mbf{A}(h)}{t} + \dep {\mbf{Q}(\mbf{A}(h))}{s} = G_s,~~~~ \hbox{on
$\Omega_{st} \times (0,t]$},
\end{equation}
%
where $\mbf{A}$ depends, through the geometry of the channel, on the channel water
height $h$. The flow rate $\mbf{Q}$ under the KWM model is only a
function of $\mbf{A}$ through the friction law.
\begin{equation} \label{qkwm}
   \mbf{Q} = \gamma \mbf{A}^m,
\end{equation}
where $\gamma=C_h\,S^{1/2}\,P\muno$ and $m=\sfr32$ for the Ch\`ezy
friction model, and $\gamma=\bar a\,n\muno\,S^{1/2}\,P^{-2/3}$ and
$m=\sfr53$ for the Manning model; $S = \dtots{h_b}{s}$ is the slope of
the stream bottom.

\subsection{Boundary Conditions.}
\subsubsection{Boundary Conditions to simulate River-Aquifer Interactions/Coupling Term.}
The stream/aquifer interaction process occurs between a stream and its
adjacent flood-plain aquifer. The coupling term is not explicitly
included in \refecu{aquifereq} but it is treated as a boundary flux
integral. At a nodal point we can write the coupling,
\begin{equation}
  G_s = P/R_f (\phi-h_b-h),
\end{equation}
where $G_s$ represents the gain or loss of the stream, and the main
component is the loss to the aquifer and $R_f$ is the resistivity
factor per unit arc length of the perimeter. The corresponding gain to
the aquifer is
\begin{equation}
  G_a = -G_s \, \delta_{\Gs},
\end{equation}
where $\Gs$ represents the planar curve of the stream and $\delta_\Gs$
is a Dirac's delta distribution with a unit intensity per unit length,
i.e.
%
\begin{equation}
  \int f(\bar{x}) \, \delta_\Gs \dS = \int_0^L f(\bar{x}(s)) \di{s}.
\end{equation}
%
The {\sl{stream loss}} element set represents this loss, and a typical
discretization is shown in \reffig{fg:aquist}. The stream loss element is
connected to two nodes on the stream and two on the aquifer.  If the
stream level is over the freatic aquifer level ($h_b+h > \phi$) then
the stream losses water to the aquifer and vice versa. Contrary to
standard approaches, the coupling term is incorporated through a
boundary flux integral that arises naturally in the weak form of the
governing equations rather than through a source term.

\subsubsection{Initial Conditions. First, Second and Third Kind Boundary Conditions/Absorbent Boundary Condition.}
\textbf{Groundwater flow.} In the previous section, the equation that
governs subsurface flow was established. In order to obtain a well
posed PDE problem, initial and boundary conditions must be
superimposed on the flow domain and on its limits.  The initial condition
for the groundwater problem is a constant hydraulic head in the whole
region that obeys levels observed in the basin history.\\ Now, consider
a simply connected region $\Omega$ bounded by a closed curve
$\partial \Omega$ such that $\partial \Omega_{\phi}
\cup \partial \Omega_{\sigma} \cup \partial \Omega_{\phi
\sigma} = \partial \Omega$. If the stream is partially penetrating and connected, in a
Hydraulic sense, to the aquifer, we set
\begin{equation}
\begin{split}
\phi&=\phi_0, ~~~\hbox{on $\partial \Omega_{\phi} \times (0,t]$}\\
K(\phi - \eta)\dep{\phi}{n}&=\sigma_0, ~~~\hbox{on $\partial
\Omega_{\sigma} \times (0,t]$}\\
K(\phi - \eta)\dep{\phi}{n}&=C(\phi - h), ~~~\hbox{on $\partial
\Omega_{\phi \sigma} \times (0,t]$}
\end{split}
\end{equation}
where $\phi_0$ is a given water head, $\sigma_0$ is a given flux
normal to the flux boundary $\partial \Omega_{\sigma}$ and $C$ the
conductance at the river/stream interface. If a fully penetrating
stream is considered,
\begin{equation}
K(\phi - \eta)\dep{\phi}{n}=C(\phi - h), ~~~\hbox{on $\partial
\Omega_{\phi \sigma} \times (0,t]$}
\end{equation}
Finally, for a perched stream,
\begin{equation}
K(\phi - \eta)\dep{\phi}{n}=C(h_b - h), ~~~\hbox{on $\partial
\Omega_{\phi \sigma} \times (0,t]$}
\end{equation}

\textbf{Surface Flow - Fluid Boundary.} We recall that the type of a
flow in a stream or in an open channel depends on the value of the
Froud number $F_r=|\mbf{u}|/c$ (where $c=\sqrt{gh}$ is the wave
celerity ), a flow is said
\begin{itemize}
\item fluvial, for $|\mbf{u}| < c$. \item torrential, for $|\mbf{u}| >
c$
\end{itemize}
\paragraph{Saint-Venant equations.} Considering a {\sl{Cauchy}}
problem for the time-like variable $x^{dim+1}=t$ where the solution is
given in the subspace $x^{dim+1}=t=0$ as
$\mbf{U}=\mbf{U}(\mbf{x},t=0)$ and is determined at subsequent values
of $t$. If the subspace $t=0$ is bounded by a surface $\partial \Gamma
(\mbf{x})$ then additional conditions have to be imposed on that
surface at all values of $t$. This defines an {\sl{initial boundary
value problem}}. A solution for the system of the first-order PDE's
can be written as a superposition of wave-like solutions of the type
corresponding to the {\sl{n}}-eigenvalues of the matrix $\mbf{A}^k
\cdot \mbf{n}= \dep{\mbf{F}_i(\mbf{U})}{\mbf{U}}^k \cdot \mbf{n},
~~~~k=1,..,dim$, being $\mbf{n}$ the outward unit normal to the
boundary edge:
\begin{equation}
\mbf{U} = \sum_{\alpha =1}^{n}
\bar{\mbf{U}}_{\alpha}e^{\mbf{I}(\mbf{x} \cdot
\mbf{n}-\omega_{\alpha}t)} ,~~~\hbox{on $\Omega_{st} \times
\Gamma_{st} \times (0,t]$}
\end{equation}
where summation extends over all eigenvalues $\lambda_{\alpha}$.  \\
As the problem is hyperbolic, {\sl{n}} initial conditions for the
{\sl{Cauchy}} problem have to be given to determine the solution.
That is, equal number of conditions as unknowns must be imposed at
$t=0$. For initial boundary value problem the {\sl{n}} boundary
conditions have to be distributed along the limits at all values of
$t$, according to the direction of the propagation of the
corresponding waves. If the wave phase velocity, the
$\alpha$-eigenvalue of $\mbf{A}^k \cdot \mbf{n}_k$ (i.e. k-wave
projected in the interior normal direction $\mbf{n}$), is positive,
the information is propagated inside the domain. Hence, the number of
conditions to be imposed for the initial boundary value problem at a
given point of $\partial \Gamma$ is equal to the number of positive
eigenvalues of $\mbf{A}^k \cdot \mbf{n}_k$ at that point. The total
number of conditions remains equal to the total number of eigenvalues
(i.e. the order of the system). For the treatment of the boundary
conditions we will use the one dimensional projected system and
consider the sign of the eigenvalues of $\mbf{A}^k$ ($u_n + c$ and
$u_n - c$). We remark that if $\mbf{n}$ is the outward unit normal to
the boundary edge, an inflow boundary corresponds to $\mbf{u \cdot n}
< 0$ and an outflow one to $\mbf{u \cdot n} >
0$. \subparagraph{Fluvial Boundary.}
\begin{itemize}
\item inflow boundary: $\mbf{u}$ specified and the depth $h$ is
extrapolated from interior points, or vice versa.
\item outflow boundary: depth $h$ specified and velocity field
extrapolated from interior points, or vice versa.
\end{itemize}
\subparagraph{Torrential Boundary.}
\begin{itemize}
\item inflow boundary: $\mbf{u}$ and the depth $h$ are specified.
\item outflow boundary: all variables are extrapolated from interior
points.
\end{itemize}
\paragraph{Solid Wall Condition.}
We prescribe the simple slip condition over $\Gamma_{slip} ~(\subset
\Gamma_{st})$
\begin{equation}
\mbf{u \cdot n}=0
\end{equation}
\paragraph{Kinematic Wave Model.} The applicability of the
kinematic wave as an approximation to dynamic wave was discussed in
{\sl{Rodr\'\i{}guez}}(1995) and, according to
Lighthill and Whitham, subcritical flow conditions favor the kinematic
approach.\\ Since one characteristic is propagated inside the domain,
only we can specify the water head, the channel section or the
discharge at inflow boundaries (see \refecu{qkwm}).\\ \\
