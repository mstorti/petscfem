%__INSERT_LICENSE__

\Section{The general advective elemset} 

\SSection{Introduction to advective systems of equations} 

Advective system of equations are of the form
%
\begin{equation} \label{eq:advec-eq}  
 \dep{U}t+\dep{\F_i(U)}{x_i} = G
\end{equation}
%
\index{inviscid}
\index{conservative variables}
\index{Euler equations}
\index{gas dynamic equations}
\index{shallow water equations}
%
where $U$ is the \emph{``state vector''} of the fluid in
\emph{``conservative variables''}. Examples of these are the inviscid
fluid equations (the \emph{``Euler''} or \emph{``gas dynamic
equations''}, the \emph{``shallow water equations''}, and scalar
advective systems that represents the transport of a scalar property
(like temperature or concentration of a component) by a moving fluid. 

The conservative variables for the Euler equations are
%
\begin{equation} 
 U=\begin{sqmat}{c}
        \rho\\
         \rho!u\\
        \rho e
   \end{sqmat}
\end{equation}
%
for the Euler equations. In general, $U$ is a vector of $\ndof$
components. $\F_i(U)$ is the \emph{``flux vector''}. t can be thought
as a matrix of $\ndof\times\ndim$ components. The row index corresponds
to a field value, whereas the column index is a spatial dimension. 
$G$ is a source vector. For the 2D or 3D Euler equations it is null,
but if we consider 1D flow in a tube of varying section, then it has a
source term in the momentum equation, due to the reaction on the
wall. Also, for the shallow water equations, there is a reaction term
in the momentum balance equations if there is a varying bathymetry.

The relation of the flux vector with the state vector is the heart of
the advective system. In fact, the discretization of advective systems
may be put in a completely abstract setting, where the unique thing
that varies from one system to another is the definition of the flux
function itself. The discretization of advective systems in \pfem{}
has been done in this way, so that it is easy to add other advective
systems by only adding the new flux function. 

Applying the chain rule, and noting that the fluxes only depend on
position through their dependence on the state vector, we arrive to
%
\begin{equation} 
   \dep {F_i}{x_i} = A_i\, \dep U{x_i}
\end{equation}
%
where 
%
\begin{equation}\label{eq:jacflux}
A_i = \dep {F_i}U
\end{equation}
%
\index{flux, jacobians}
are the \emph{``jacobians of the advective fluxes''}. 

\SSection{Discretization of advective systems} 

Using the Finite Element Method, with weight functions $W_j(!x)$ and
interpolation functions $N_j(!x)$ results in
%
\begin{equation} 
   !M \dot{!U} + !F(!U) = !G
\end{equation}
%
where 
%
\begin{equation} 
  !U = \begin{sqmat}{c}
        U_1\\
        U_2\\
        \vdots\\
        U_\Nnod
       \end{sqmat}
\end{equation}
%
so that $!U$ has $\Ndof=\ndof\times\Nnod$ components. 

$!M$ (of dimension $\Ndof\times\Ndof$) is the mass matrix. If we look
at it as an $\Nnod\times\Nnod$ block matrix with blocks of
size $\ndof\times\ndof$, then the $i,j$ block is
%
\begin{equation}
   !M_{ij} = \int_\Omega N_i(!x) \, N_j(!x) \di{!x}
\end{equation}
%
the $i$ block of the global flux and source vector $!F$ and $!G$ are 
%
\begin{equation}
\begin{aligned}
   !F_{i} &= \int_\Omega N_i(!x) \, \dep{F_k}{x_k} \di{!x}\\
   !G_{i} &= \int_\Omega N_i(!x) \, G(!x) \di{!x}
\end{aligned}
\end{equation}
%
If the flux vector term is integrated by parts then we have the
\emph{``weak form''}
%
\begin{equation}
   !F_{i} = -\int_\Omega \dep{N_i}{x_k} \, F_k \di{!x} + \int_\Gamma
   n_k F_k(!x) \di{!x}
\end{equation}
%
\index{finite difference method}
\index{stabilizing term}
\index{SUPG}
\index{Streamline Upwind/Petrov Galerkin}
\index{perturbation function}
%
where $\Gamma$ is the boundary of $\Omega$. This formulation is the
\emph{``Galerkin''} or \emph{``centered''} one. It is equivalent to
approximate first derivatives by centered differences in the Finite
Difference Method. It is well known that the Galerkin formulation
leads to oscillations for advective systems, and this is solved by
adding a \emph{``stabilizing term''} to the discretized equations. 

\SSection{SUPG stabilization} 

In the SUPG (for \emph{``Streamline Upwind/Petrov Galerkin''})
formulation of Hughes et.al. The stabilized formulation is
%
\begin{equation}\label{eq:supg} 
   (!M \dot{!U} + !F(!U) - !G)_j
       + \sum_e \int_{\Omega_e} (\Psupg)^e_j \, (\dep Ut+\dep
       {F_i}{x_i} -G) = 0
\end{equation}
%
\index{weighted residual}
\index{consistent formulation}
%
where the whole expression corresponds to the $j$-th block of size
$\ndof$ in the global equations.  Note that, as the added term is a
\emph{``weighted residual''} form of the residual (the term in
parentheses), then the continuum solution is solution of these
discrete equations -- we say that this is a \emph{``consistent
formulation''}. $\Psupg$ is a matrix of $\ndof\times\ndof$ the SUPG
\emph{``perturbation function''}, usually defined as
%
\begin{equation} 
  (\Psupg)^e_j = \btau^e \, A_j \dep N{x_j}
\end{equation}
%
\index{intrinsic time}
where $\btau^e$ are the \emph{``characteristic''} or \emph{``intrinsic
time''} of the element, defined as 
%
\begin{equation} 
  \btau^e = \frac{h^e}{||!A||}
\end{equation}
%
where $h^e$ is the size of the element and and $||!A||$ represents
some norm of the vector of jacobians. There is a variety of
possibilities for computing both quantities. For instance $h^e$ may be
computed as the largest side of the element, or as the radius of the
circle with the same area. On the other hand, $||!A||$ may be computed
as the maximum eigenvalue of all the linear combinations of the form
$n_jA_j$, with $n_j$ a unit vector, i.e. the maximum propagation
velocity possible in the fluid, that is
%
\begin{equation}\label{eq:lammax} 
    ||!A|| = \max_{j=1,...,\ndim} \max_{k=1,...,\ndof} |\lambda^j_k|
\end{equation}
%
where $\lambda^j_k$ is the $k$-th eigenvalue of jacobian $A_j$.  For
the Euler equations, it turns out to be that this corresponds to
pressure waves propagating in the direction of the fluid and is $c+u$
where $c$ is the speed of sound and $u$ the absolute value of
velocity. For the shallow water equations, its value is $u+\sqrt{gh}$
where $g$ is gravity acceleration and $h$ the local water elevation
with respect to bottom. 

\SSection{Shock capturing}\label{sec:shockcap} 

For problems with strong shocks, (shock waves in Euler, or hydraulic
jumps in shallow water) the standard SUPG stabilizing term may not be
sufficient. Then an additional stabilizing term is added, so that the
stabilized equations are now of the form
%
\begin{equation} 
\begin{split}
   (!M \dot{!U} + !F(!U) - !G)_j
       + \sum_e \int_{\Omega_e} (\Psupg)^e_j \, (\dep Ut+\dep
       {F_i}{x_i} -G) +\\
       + \sum_e \int_{\Omega_e} \deltasc \, \dep{N_j}{x_i} \, \dep U{x_i} = 0
\end{split}
\end{equation}
%
Note that, in contrast with the SUPG term, the new, so-called shock
capturing term is no more \emph{``consistent''}. $\deltasc$ is a
scalar -- the so called \emph{``shock capturing parameter''}. Often,
when shock capturing is added, we diminish the amount of stabilization
in the SUPG term in order to compensate and not to have an
over-diffusive scheme. We will not enter in the details of this
computations, refer to \cite{HM_86B} for further details. 

\SSection{Creating a new advective system}

New advective systems may be added to \pfem{} only by defining their
flux function, jacobians and other quantities. This means that you
don't need to code details of the numerical discretization. Follow
these steps

\begin{enumerate}
\item Create the flux function in a file by
itself, in the \verb+applications/advective+ directory. (The better is
to start copying one of the existing advective systems, for instance
\verb+ffeuler.cpp+ or \verb+ffshallw.cpp+.) The arguments to flux
function routines is described in section~\ref{sec:fluxfun}. 
The name of the
function has to be of the form \verb+flux_fun_+\emph{<system>} where
\emph{<system>} identifies the new system. We assume that you write
the flux function \verb+flux_fun_new_adv_sys+ in file 
\verb+applications/advective/ffnadvs.cpp+. 

\item Add the file in the \verb+MYOBJS+ variable in the Makefile, for
instance

\begin{verbatim} 
MYOBJS = advective.o adv.o absorb.o ffeuler.o \
		ffshallw.o ffadvec.o
\end{verbatim}

\item Define the new derived classes \verb+volume_new_adv_sys+ and
\verb+absorb_new_adv_sys+ by adding a line at the end of the file
\verb+applications/advective/advective.h+ as in the following
example. 

\begin{verbatim} 
// Add here declarations for further advective elemsets. 
/// Euler equations for inviscid (Gas dynamics eqs.)
ADVECTIVE_ELEMSET(euler);    
/// Shallow water equations. 
ADVECTIVE_ELEMSET(shallow);
/// Advection of multiple scalar fields with a velocity field. 
ADVECTIVE_ELEMSET(advec);
/// My new advective system          
ADVECTIVE_ELEMSET(new_adv_sys);             // <- Add this line. 
\end{verbatim}

\item Recompile. 
\end{enumerate}

\SSection{Flux function routine arguments}\label{sec:fluxfun} 

Currently, the interface is the following
%
\begin{verbatim}
typedef int FluxFunction(const RowVector &U,
        int ndim,const Matrix &iJaco, 
	Matrix &H, Matrix &grad_H, 
	Matrix &flux,vector<Matrix *> A_jac, 
	Matrix &A_grad_U, Matrix &grad_U,  
	Matrix &G_source,Matrix &tau_supg, double &delta_sc, 
	double &lam_max, 
	TextHashTable *thash,double *propel, 
	void *user_data,int options)
\end{verbatim}
%
(This may eventually change -- in any case, if you are interested in
adding a new advective system, then see the actual description in the
\verb+advective.h+ file in the distribution.) The meaning of these
arguments are listed below. When the size is specified, it means that
the argument is a Newmat or FastMat matrix. 

In some situations the flux function routine must compute only some of
the required values. For instance, when computing the contribution of
the absorbing boundary elements there is no need to compute the
parameters regarding stabilizing terms. This is controlled with the
parameter \verb+options+ which can take the values \verb+DEFAULT+,
\verb+COMP_UPWIND+ and \verb+COMP_SOURCE+. In the list below, it is
indicated under which conditions the specific quantity must be
computed. 

%
{\raggedright
\begin{itemize}
\item\verb+const RowVector &U+ (input, size $\ndof$)
   This is the state vector -- you must return the flux, jacobians and
other quantities for \emph{this} state vector. 

\item\verb+int ndim+ (input) The dimension of the space. 

\item\verb+const Matrix &iJaco+ (input, size $\ndim\times\ndim$) 
The jacobian of the master
to element coordinates in the actual gauss points. This may be used in
order to calculate the characteristic size of the element

\item\verb+Matrix &H+ (input, size $1\times n_H$) In the shallow
water, the source term $G$ depends on the gradient of the depth
$H(x)$, and in the 1D Euler equations, on the area of the tube section
$A(x)$. This is taken into account by \pfem{} by assuming that the
user enters in the \verb+nodedata+ section ${\tt nu}=\ndim+n_H$
quantities per node, where the first $\ndim$ quantities are the node
coordinates and the rest are assumed that are node data that has to be
passed to the flux function routine (together with its gradient) in
order to compute the source term.

\item\verb+Matrix &grad_H+ (input, size $\ndim\times n_H$) the
gradient of the quantities in $H$ (see previous entry). 

\item\verb+Matrix &flux+ (output, size $\ndof\times\ndim$) Each
column is the vector $F_j$ of fluxes for each of the governing
equations. 

\item\verb+vector<Matrix *> A_jac+ (output, size: a vector of
$\ndim$ pointers to matrices of $\ndof\times\ndof$). Each matrix is
the jacobian matrix as defined by (\ref{eq:jacflux}). To access the
\verb+jd+ jacobian matrix  you may write \verb+(*A_jac[(jd)-1])+. The
macro \verb+AJAC(jd)+, defined in \verb+advective.h+  expands to this. 

\item\verb+Matrix &A_grad_U+ (output, size $\ndof\times 1$, 
compute if \verb+options & COMP_UPWIND+) This is
the accumulation term $A_k \deps{U}{x_j} = \dep{F_j}{x_j}$. 

\item\verb+Matrix &grad_U+ (input, size $\ndim\times\ndof$) The
gradient of the state vector. 

\item\verb+Matrix &tau_supg+ (output, size: either $1\times1$ or
$\ndof\times\ndof$, compute if \verb+options & COMP_UPWIND+). 
This is the $\btau$ intrinsic time scale -- it
may be either a scalar or a matrix. Beware that even in the case where
it is a scalar it must be returned as a Newmat \verb+Matrix+ object of
dimensions $1\times 1$. 

\index{shock capturing}
\item\verb+double &delta_sc+ (output, double, compute if
\verb+options & COMP_UPWIND+) 
This is the \emph{``shock capturing''}
parameter as described in \ref{sec:shockcap}. Set to 0 if no shock
capturing is added. 

\item\verb+double &lam_max+ (output, double, compute if
\verb+options & COMP_UPWIND+) 
The maximum propagation
speed. The expression is (\ref{eq:lammax}) but normally it may be
computed directly from the state vector. This is used to compute
upwind, and also the automatic and local time step. 

\item\verb+TextHashTable *thash+ (input) This is the
\verb+TextHashTable+ of the elemset. Physical and numerical parameters
can be passed from the user input data file to the flux function
routine through this (specific heat, specific heat ratio,
gravity... for instance). Beware that the flux function routine is
called at each Gauss point, and decoding of the table may be time
expensive, so that if the properties are constant over all the mesh
you can decode them once and leave the decoded data in static
variables. 

\item\verb+double *propel+ (input, double array, compute if
\verb+options & COMP_UPWIND+) 
This is the table
of per-element properties. Physical and numerical parameters that are
not constant for all the elemsets, can be passed from the user input
data file to the flux function routine through this. 

\item\verb+void *user_data+ (input). Arbitrary information may be
passed from the main routine to the flux function through this
pointer. 

\item\verb+int options+ (input, integer). 

\item\verb+Matrix &G_source+ (output, size $\ndof\times 1$, 
           compute if \verb+options & COMP_SOURCE+) The
           source vector. 
\end{itemize}
}

%<*>---<*>---<*>---<*>---<*>---<*>---<*>---<*>---<*>---<*>---<*> 
\SSSection{Options}

\begin{description}
%<*>---<*>---<*>---<*>---<*>---<*>---<*>---<*>---<*>---<*>---<*> 
\item[General options:] ~

\begin{itemize}
\input odocadv
\end{itemize}

%<*>---<*>---<*>---<*>---<*>---<*>---<*>---<*>---<*>---<*>---<*> 
\item[Generic elemset ``{\tt advecfm2}'':] ~

\begin{itemize}
\input odocadve
\end{itemize}

%<*>---<*>---<*>---<*>---<*>---<*>---<*>---<*>---<*>---<*>---<*> 
\item[Flux function``{\tt ffeulerfm2}'':] Euler eqs.

\begin{itemize}
\input odocadvfe
\end{itemize}

%<*>---<*>---<*>---<*>---<*>---<*>---<*>---<*>---<*>---<*>---<*> 
\item[Flux function``{\tt ffswfm2}'':] Shallow water eqs.

\begin{itemize}
\input odocadvfs
\end{itemize}


\end{description}

% Local Variables: *
% mode: latex *
% tex-main-file: "petscfem.tex" *
% End: *
