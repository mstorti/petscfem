%__INSERT_LICENSE__
%<*>---<*>---<*>---<*>---<*>---<*>---<*>---<*>---<*>---<*>---<*> 
%section rotary_save
\label{sec:rotary_save}
Sometimes it is interesting to save the state vector with a certain
frequency in a ``append'' manner, i.e. appending the state vector at
the end of the file. However, this posses the danger of storing too
much amount of data if the user performs a very long run. The ``rotary
save'' mechanism allows writing only a certain amount of the recent
states. The mechanism basically saves the state vector each
\verb+nsaverot+ steps appending to the a file. The name of the file is
contructed from a pattern set by the user via the
\verb+save_file_pattern+ entry, by replacing \verb+%d+ 
by \verb+0+ ``\`a la'' \verb+printf()+. For instance, if
\verb+save_file_pattern+ i set to \verb+file%d.out+ 
then the state vectors are appended to \verb+file0.out+.  When the
number of written states reach the \verb+nrec+ count, the file is
reset to 0, and the saving continues from the start of the
file. However, if \verb+nfile+ is greater than one, then the state
vector are continued to be stored in file \verb+file1.out+ and so
on. When the number of files \verb+nfile+ is reached, the saving
continues in file '0'.

More precisely, the saving mechanism is described by the following
pseudo-code:

% %\begin{latexonly}
% % algorithmic environment is not correctly processed by Latex2HTML
% \begin{algorithmic}
% \STATE read state vector from \verb+initial_state+ file into $x^0$, $n\gets 0$.
% \FOR{$i$=0 to {\tt nstep}}
%   \STATE advance $x^n$ to $x^{n+1}$.
%   \IF{$n$ \% {\tt nsaverot} == 0}
%     \STATE $j \gets n$/{\tt nsaverot}
%     \STATE $k \gets j$ \% {\tt nrec}
%     \STATE $l \gets j$/{\tt nrec}
%     \IF{$k=0$}\STATE rewind file $l$ \ENDIF
%     \STATE append state vector to file $l$
%   \ENDIF
% \ENDFOR
% \end{algorithmic}
% %\end{latexonly}
%%%<>%%%<>%%%<>%%%<>%%%<>%%%<>%%%<>%%%<>%%%<>%%%<>%%%<>%%%<>%%%<> 
% algorithmic environment is not correctly processed by Latex2HTML
% \begin{htmlonly}
\begin{verbatim}
Read state vector from `initial_state' file into x^0, n=0;
for (i=0; i<nstep; i++) {
   advance x^n to x^{n+1};
   if (n % nsaverot == 0) {
           j <- n/nsaverot;
           k <- j % nrec;
           l <- j / nrec;
           if (k==0) { rewind file l; }
           append state vector to file l;
   }
}
\end{verbatim}
% \end{htmlonly}
%end_section

%<*>---<*>---<*>---<*>---<*>---<*>---<*>---<*>---<*>---<*>---<*> 
%section print_some
\label{sec:print_some}
The ``print some'' mechanism allows the user to store the variables of
some set of nodes with some frequency. The nodes are entered in a
separate file whose name is given by a \verb+print_some_file+ entry in
the general options, one node per line. The entry \verb+nsome+
indicates the frequency (in steps) at which the data is saved and
\verb+save_file_some+ the name of the file to save in. 
%end_section

%<*>---<*>---<*>---<*>---<*>---<*>---<*>---<*>---<*>---<*>---<*> 
%section newton_relaxation_factor
Relaxation parameter for Newton iteration. Several
values may be entered in the form
%
\begin{verbatim} 
newton_relaxation_factor w1 n1 w2 n2 .... wn
\end{verbatim}
%
that means: Take relaxation factor \verb+w1+
for the first \verb+n1+ steps, \verb+w2+ for the following \verb+n2+ steps
and so on until \verb+w_{n-1}+. \verb+wn+ is taken for all subsequent 
steps. Normally one takes a conservative (said 0.5) relaxation
factor for the first steps and then let full Newton (i.e. \verb+w=1+)
for the rest. For instance, the line  
%
\begin{verbatim} 
newton_relaxation_factor 0.5 3 1.
\end{verbatim}
%
means: take $w=0.5$ for the first 3 steps, and then use $w=1$. 
%end_section

%<*>---<*>---<*>---<*>---<*>---<*>---<*>---<*>---<*>---<*>---<*> 
%section free_surface_damp
\label{sec:free_surface}
The equation of the free surface is
%
\begin{equation} 
\dtots\eta t = w
\end{equation}
%
where $\eta$ 
elevation, and $w$ is the velocity component normal to the free
surface. We modify this as follows 
%
\begin{equation} \label{eq:fsmod}  
\Cnst{eq}\,\dtot\eta t 
   + \Cnst{lf} \bar{\eta} - \Cnst{damp} \, \Delta \eta = w
\end{equation}
%
Where $\bar\eta$ is the average value of eta on the free surface, and:
$\Cnst{damp}={\tt free\_surface\_damp}$ smoothes the free surface
adding a Laplacian filter.  Note that if only the temporal derivative
and the Laplace term are present in (\ref{eq:fsmod}) then the equation
is a heat equation. A null value (which is the default) means no
filtering. A high value means high filtering. (Warning: A high value
may result in unstability).  $\Cnst{damp}$ has dimensions of $L^2/T$
(like a diffusivity). One possibility is to scale with mesh parameters
like $h^2/\Delta t$, other is to scale with $h^{1.5} \,
g^{0.5}$. Currently, we are using $\Cnst{damp} = \Cnst{damp}' h^{1.5}
g^{0.5}$ with $\Cnst{damp}'\approx 2$.
%end_section

%<*>---<*>---<*>---<*>---<*>---<*>---<*>---<*>---<*>---<*>---<*> 
%section dx_line
List of indices of the nodes to be passed to DX as connectivity
table. This has mainly two uses. First, note that the order the nodes
are entered in DX is not the same as in PETSc-FEM. For instance, in
PETSc-FEM the nodes of a quad are entered counter-clockwise, and, if
they are (in that order) 1, 2, 3, 4, then for DX they are entered in
the order: 1, 2, 4, 3. Secondly, DX has only four basic interpolations
(quads, cubes, triangles and tetras) so that other interpolations, as
prisms must be split in some number of these basic DX geometries.
For the more usual interpolations this is automatically generated.
The line has the following syntax, {\tt node-list dx-geometry}.  Where
{\tt node-list} is a list of (1-based) nodes and {\tt dx-geom} the
corresponding gometry. For instance, for the {\tt cartesian2d}
PETSc-FEM geometry, one could use the line {\tt 1 2 4 3 quads}. 
That element is equivalent to one quad, with a node
renumbering. (However this conversion is made automatically by
PETSc-FEM). If we wanted to split the quad in two triangles, say
triangle (1 2 3) and (3 4 1), we should enter {\tt 1 2 3 3 4 1
triangles}. As another example, the prism element may be split in 3
tetras according to the rule. {\tt 1 2 3 4 5 4 6 2 2 6 3 4
tetrahedra}. If you need to split the element in different geometries,
you can use the syntax {\tt node-list-1 dx-geom-1 node-list-2
dx-geom-2 ... node-list-n dx-geom-n}. 
% If the element may be split in only one type of
% geometry, then The number of nodes in #node-list-j# must be a multiple
% of the number of nodes in each #dx_type_j# is of the form #subel_1
% subel_2 ...  dx_type#, #dx_type# may be #quads#, #cubes#, with #k =
% nsubelem * subnel#. For instance a prism may be slit in 3 tetras with
% a line like #tetrahedra 3 4 1 2 3 4 5 4 6 2 2 6 3 4#.  For instance,
% for quads one should enter {\tt dx_indices 1 2 4 3 quads} and for
% cubes {\tt dx_indices 1 2 4 3 5 6 8 7}. [Note: DX wants 0-based
% (C-sytle) node numbers, whereas PETSc-FEM uses 1-based (Fortran
% style).  However they must be entered 1-based for this option.]
%end_section
