%-----<>-----<>-----<>-----<>-----<>-----<>-----<>-----<>-----<>-----<>
% DON'T EDIT MANUALLY THIS FILE !!!!!!
% This files automatically generated by odoc.pl from 
% source file ""
%-----<>-----<>-----<>-----<>-----<>-----<>-----<>-----<>-----<>-----<>
\index{atol@\verb+atol+}
\item\verb+double atol+ {\rm(default=\verb|1e-6|)}:

Absolute tolerance when solving a consistent matrix

%----<>----<>----<>----<>----<>----<>----<>----<>----<>----<>
\index{rtol@\verb+rtol+}
\item\verb+double rtol+ {\rm(default=\verb|1e-3|)}:

Relative tolerance when solving a consistent matrix

%----<>----<>----<>----<>----<>----<>----<>----<>----<>----<>
\index{dtol@\verb+dtol+}
\item\verb+double dtol+ {\rm(default=\verb|1e+3|)}:

Divergence tolerance when solving a consistent matrix

%----<>----<>----<>----<>----<>----<>----<>----<>----<>----<>
\index{maxits@\verb+maxits+}
\item\verb+int maxits+ {\rm(default=\verb|150|)}:

Maximum iterations when solving a consistent matrix

%----<>----<>----<>----<>----<>----<>----<>----<>----<>----<>
\index{print\_internal\_loop\_conv@\verb+print_internal_loop_conv+}
\item\verb+int print_internal_loop_conv+ {\rm(default=\verb|0|)}:

Prints the convergence history when solving a consistent matrix

%----<>----<>----<>----<>----<>----<>----<>----<>----<>----<>
\index{measure\_performance@\verb+measure_performance+}
\item\verb+int measure_performance+ {\rm(default=\verb|0|)}:

Measure performance of the \verb+comp_mat_res+ jobinfo. 

%----<>----<>----<>----<>----<>----<>----<>----<>----<>----<>
\index{nsave@\verb+nsave+}
\item\verb+int nsave+ {\rm(default=\verb|10|)}:

Save state vector frequency (in steps)

%----<>----<>----<>----<>----<>----<>----<>----<>----<>----<>
\index{nsaverot@\verb+nsaverot+}
\item\verb+int nsaverot+ {\rm(default=\verb|100|)}:

Save state vector frequency with the ``rotary save''
mechanism. (see \ref{sec:rotary_save})

%----<>----<>----<>----<>----<>----<>----<>----<>----<>----<>
\index{nrec@\verb+nrec+}
\item\verb+int nrec+ {\rm(default=\verb|1000000|)}:

Sets the number of states saved in a given file
in the ``rotary save'' mechanism (see \ref{sec:rotary_save}

%----<>----<>----<>----<>----<>----<>----<>----<>----<>----<>
\index{nfile@\verb+nfile+}
\item\verb+int nfile+ {\rm(default=\verb|1|)}:

Sets the number of files in the ``rotary save'' mechanism. 
(see \ref{sec:rotary_save})

%----<>----<>----<>----<>----<>----<>----<>----<>----<>----<>
\index{nstep@\verb+nstep+}
\item\verb+int nstep+ {\rm(default=\verb|10000|)}:

The number of time steps. 

%----<>----<>----<>----<>----<>----<>----<>----<>----<>----<>
\index{nstep\_cpu\_stat@\verb+nstep_cpu_stat+}
\item\verb+int nstep_cpu_stat+ {\rm(default=\verb|10|)}:

Output CPU time statistics for frequency in time steps. 

%----<>----<>----<>----<>----<>----<>----<>----<>----<>----<>
\index{print\_linear\_system\_and\_stop@\verb+print_linear_system_and_stop+}
\item\verb+int print_linear_system_and_stop+ {\rm(default=\verb|0|)}:

After computing the linear system prints Jacobian and
right hand side and stops.. 

%----<>----<>----<>----<>----<>----<>----<>----<>----<>----<>
\index{solve\_system@\verb+solve_system+}
\item\verb+int solve_system+ {\rm(default=\verb|1|)}:

Solve system before \verb+print\_linear_system_and_stop+

%----<>----<>----<>----<>----<>----<>----<>----<>----<>----<>
\index{nsome@\verb+nsome+}
\item\verb+int nsome+ {\rm(default=\verb|10000|)}:

Sets the save frequency in iterations for the ``print some''
mechanism. (see doc in the Navier-Stokes module)

%----<>----<>----<>----<>----<>----<>----<>----<>----<>----<>
\index{print\_some\_file@\verb+print_some_file+}
\item\verb+string print_some_file+ {\rm(default=\verb||)}:

Name of file where to read the nodes for the ``print some'' 
feature. 

%----<>----<>----<>----<>----<>----<>----<>----<>----<>----<>
\index{save\_file\_some@\verb+save_file_some+}
\item\verb+string save_file_some+ {\rm(default=\verb|outvsome.out|)}:

Name of file where to save node values for the ``print some'' 
feature. 

%----<>----<>----<>----<>----<>----<>----<>----<>----<>----<>
\index{report\_option\_access@\verb+report_option_access+}
\item\verb+int report_option_access+ {\rm(default=\verb|1|)}:

Print, after execution, a report of the times a given option
was accessed. Useful for detecting if an option was used or not.

%----<>----<>----<>----<>----<>----<>----<>----<>----<>----<>
\index{use\_iisd@\verb+use_iisd+}
\item\verb+int use_iisd+ {\rm(default=\verb|0|)}:

Use IISD (Interface Iterative Subdomain Direct) or not.

%----<>----<>----<>----<>----<>----<>----<>----<>----<>----<>
\index{solver@\verb+solver+}
\item\verb+string solver+ {\rm(default=\verb|petsc|)}:

Type of solver. May be \verb+iisd+ or \verb+petsc+. 

%----<>----<>----<>----<>----<>----<>----<>----<>----<>----<>
\index{consistent\_supg\_matrix@\verb+consistent_supg_matrix+}
\item\verb+int consistent_supg_matrix+ {\rm(default=\verb|0|)}:

Uses consistent SUPG matrix for the temporal term or not. 

%----<>----<>----<>----<>----<>----<>----<>----<>----<>----<>
\index{auto\_time\_step@\verb+auto_time_step+}
\item\verb+int auto_time_step+ {\rm(default=\verb|1|)}:

Chooses automatically the time step from the 
selected Courant number

%----<>----<>----<>----<>----<>----<>----<>----<>----<>----<>
\index{local\_time\_step@\verb+local_time_step+}
\item\verb+int local_time_step+ {\rm(default=\verb|1|)}:

Chooses a time step that varies locally. (Only makes sense
when looking for steady state solutions. 

%----<>----<>----<>----<>----<>----<>----<>----<>----<>----<>
\index{Courant@\verb+Courant+}
\item\verb+double Courant+ {\rm(default=\verb|0.6|)}:

The Courant number.

%----<>----<>----<>----<>----<>----<>----<>----<>----<>----<>
\index{Dt@\verb+Dt+}
\item\verb+double Dt+ {\rm(default=\verb|0.|)}:

Time step. 

%----<>----<>----<>----<>----<>----<>----<>----<>----<>----<>
\index{steady@\verb+steady+}
\item\verb+int steady+ {\rm(default=\verb|0|)}:

Flag if steady solution or not (uses Dt=inf). If \verb+steady+
is set to 1, then the computations are as if $\Dt=\infty$. 
The value of \verb+Dt+ is used for printing etc... If \verb+Dt+
is not set and \verb+steady+ is set then \verb+Dt+ is set to one.

%----<>----<>----<>----<>----<>----<>----<>----<>----<>----<>
\index{alpha@\verb+alpha+}
\item\verb+double alpha+ {\rm(default=\verb|0.|)}:

The parameter of the trapezoidal rule
for temporal integration. 

%----<>----<>----<>----<>----<>----<>----<>----<>----<>----<>
\index{nnwt@\verb+nnwt+}
\item\verb+int nnwt+ {\rm(default=\verb|3|)}:

Number of iterations in the Newton loop. (
for the implicit method. 

%----<>----<>----<>----<>----<>----<>----<>----<>----<>----<>
\index{start\_time@\verb+start_time+}
\item\verb+double start_time+ {\rm(default=\verb|0.|)}:

Counts time from here.

%----<>----<>----<>----<>----<>----<>----<>----<>----<>----<>
\index{tol\_mass@\verb+tol_mass+}
\item\verb+double tol_mass+ {\rm(default=\verb|1e-3|)}:

Tolerance when solving with the mass matrix. 

%----<>----<>----<>----<>----<>----<>----<>----<>----<>----<>
\index{tol\_linear@\verb+tol_linear+}
\item\verb+double tol_linear+ {\rm(default=\verb|1e-3|)}:

Tolerance when solving the sublinear problem
at each iteration.

%----<>----<>----<>----<>----<>----<>----<>----<>----<>----<>
\index{tol\_newton@\verb+tol_newton+}
\item\verb+double tol_newton+ {\rm(default=\verb|1e-3|)}:

Tolerance when solving the non-linear problem
for the implicit case.

%----<>----<>----<>----<>----<>----<>----<>----<>----<>----<>
\index{tol\_steady@\verb+tol_steady+}
\item\verb+double tol_steady+ {\rm(default=\verb|0.|)}:

Tolerance when solving for a steady state

%----<>----<>----<>----<>----<>----<>----<>----<>----<>----<>
\index{omega\_newton@\verb+omega_newton+}
\item\verb+double omega_newton+ {\rm(default=\verb|1.|)}:

Relaxation factor for the Newton iteration

%----<>----<>----<>----<>----<>----<>----<>----<>----<>----<>
\index{preco\_type@\verb+preco_type+}
\item\verb+string preco_type+ {\rm(default=\verb|jacobi|)}:

Chooses the preconditioning operator. 

%----<>----<>----<>----<>----<>----<>----<>----<>----<>----<>
\index{save\_file\_pattern@\verb+save_file_pattern+}
\item\verb+string save_file_pattern+ {\rm(default=\verb|outvector%d.out|)}:

The pattern to generate the file name to save in for
the rotary save mechanism.

%----<>----<>----<>----<>----<>----<>----<>----<>----<>----<>
\index{save\_file@\verb+save_file+}
\item\verb+string save_file+ {\rm(default=\verb|outvector.out|)}:

Filename for saving the state vector.

%----<>----<>----<>----<>----<>----<>----<>----<>----<>----<>
%-----<>-----<>-----<>-----<>-----<>-----<>-----<>-----<>-----<>-----<>
% DON'T EDIT MANUALLY THIS FILE !!!!!!
% This files automatically generated by odoc.pl from 
% source file ""
%-----<>-----<>-----<>-----<>-----<>-----<>-----<>-----<>-----<>-----<>
