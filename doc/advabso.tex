%__INSERT_LICENSE__
% $Id: advabso.tex,v 1.3 2005/02/04 19:56:32 mstorti Exp $ 

\SSection{Absorbing boundary conditions} 

Once you write the flux function for a particular advective-diffusive
problem you have an associated absorbing boundary condition. There are
basically two types of absorbing boundary conditions

\begin{itemize}
\compactlist 
\item Linear, based on the Jacobian of the flux function, assuming
  small perturbations about a reference value. 
\item Based on Riemman invariants (require the writter of the flux
  function to provide the Riemman invariants for the flux function).
\end{itemize}
 
Starting with the conservation form of an advective system
(\ref{eq:advec-eq}), and assuming small perturbations about a mean
fluid state $!U(!x,t) = !U_0 + !U'(!x,t)$, and no source term, then we
obtain the linearized form
%
\begin{equation} \label{eq:advec-lin}  
  \dep{!U'}t + !A_0\dep {!U'}x =0. 
\end{equation}
%
where we assume further, that the flow only depends on $x$ the
direction normal to the boundary. Let $!S$ be the matrix of
right eigenvectors of $!A_0$ so that
%
\begin{equation} 
  !A_0!S = !S\Lambda
\end{equation}
%
where $\bzL = \diag\{\lambda_1,...,\lambda_\ndof\}$ are the
eigenvalues of $A_0$. Multiplying (\ref{eq:advec-lin}) at left byt
$!S\muno$ and defining $V=!S\muno !U'$ we obtain a decoupled system of
equations
%
\begin{equation}
  \dep{!V}t + \bzL\dep {!V}x =0. 
\end{equation}
%
Now, the equation for each component $v_j$ of $!V$ is a simple linear
transport equation with constant transport velocity $\lambda_j$
%
\begin{equation} '
  \dep{v_j}t + \lambda_j\dep {v_j}x =0. 
\end{equation}
%
so that the absorbing boundary condition is almost evident. Assuming
that we want to solve the equation on $-\infty<x<0$, so that $x=0$ is
a boudnary, then the corresponding absorbing boundary condition is
%
\begin{equation} 
  \begin{cases}
    v_j(0) = 0; & \text{ if $\lambda_j\le 0$ (ingoing boundary)}\\
    v_j \textrm{extrapolated from interior}; & \text{(outgoing boundary)}\\
  \end{cases}
\end{equation}
%
This can be summarized as
%
\begin{equation} 
  \bzP_V^- !V(x=0) = 0
\end{equation}
%
where 
%
\begin{equation} 
  \bzP_V^- = \diag\{(1-\sign(\lambda_j))/2\}
\end{equation}
%
is the projection matrix onto the space of incoming waves. As the
$\bzP_V^-$ is a diagonal matrix, with diagonal elements 1 or 0, it
trivially satisfies the projection condition $\bzP_V^- \bzP_V^- =
\bzP_V^-$. 

Coming back to the $!U$ basis, we obtain the following first-order,
linear absorbing boundary condition
%
\begin{equation} 
  \bzP_U^-(!U^0) \, (!U(x=0)-!U_0) = 0. 
\end{equation}
%
This condition is perfectly absorbing for small amplitude waves around
the state $!U_0$. The main problem with it is that as the state that
the state at the boundary is not known a priori, and so 



% Local Variables: *
% mode: latex *
% tex-main-file: "petscfem.tex" *
% End: *
