%__INSERT_LICENSE__
% $Id: advabso.tex,v 1.11 2005/02/23 03:14:42 mstorti Exp $ 

\SSection{Absorbing boundary conditions} 

Absorbing boundary conditions are a very useful feature for the
solution of advective diffusion problems. They allow the user to put
artificial boundaries closer to the interest region, and also
accelerate convergence to steady solutions, since provide the highest
rate of energy dissipation through the boundaries. 

In PETSc-FEM, once you write the flux function for a particular
advective-diffusive problem you get absorbing boundary conditions with
none or little extra work. There are basically two types of absorbing
boundary conditions

\begin{itemize}
\compactlist 
\item Linear, based on the Jacobian of the flux function, assuming
  small perturbations about a reference value. 
\item Based on Riemann invariants (require the writer of the flux
  function to provide the Riemann invariants for the flux
  function). (Needs the user to write the Riemann invariants and 
\end{itemize}

%<*>---<*>---<*>---<*>---<*>---<*>---<*>---<*>---<*>---<*>---<*>
\SSSection{Linear absorbing boundary conditions} 
 
Starting with the conservation form of an advective system
(\ref{eq:advec-eq}), and assuming small perturbations about a mean
fluid state $!U(!x,t) = !U_0 + !U'(!x,t)$, and no source term, then we
obtain the linearized form
%
\begin{equation} \label{eq:advec-lin}  
  \dep{!U'}t + !A_0\dep {!U'}x =0. 
\end{equation}
%
where we assume further, that the flow only depends on $x$ the
direction normal to the boundary. Let $!S$ be the matrix of
right eigenvectors of $!A_0$ so that
%
\begin{equation} 
  !A_0!S = !S\Lambda
\end{equation}
%
where $\bzL = \diag\{\lambda_1,...,\lambda_\ndof\}$ are the
eigenvalues of $A_0$. Assuming that the system is
\emph{``hyperbolic''}, then such a diagonal decomposition is possible,
for any state $!U_0$, with real eigenvalues and a non singular matrix
$!S$. Multiplying (\ref{eq:advec-lin}) at left by $!S\muno$ and
defining $V=!S\muno !U'$ we obtain a decoupled system of equations
%
\begin{equation}
  \dep{!V}t + \bzL\dep {!V}x =0. 
\end{equation}
%
\index{characteristic component}
Now, the equation for each \emph{``characteristic component''} $v_j$ of $!V$ is a
simple linear transport equation with constant transport velocity
$\lambda_j$
%
\begin{equation} \label{eq:abc-chareq}  
  \dep{v_j}t + \lambda_j\dep {v_j}x =0. 
\end{equation}
%
so that the absorbing boundary condition is almost evident. Assuming
that we want to solve the equation on the semiplane $x\ge $, so
that $x=0$ is a boundary, then the corresponding absorbing boundary
condition is
%
\begin{equation} \label{eq:v-abc}  
  \begin{cases}
    v_j(0) = 0; & \text{ if $\lambda_j\ge 0$ (ingoing boundary)}\\
    v_j \textrm{extrapolated from interior}; & \text{otherwise, (outgoing boundary)}\\
  \end{cases}
\end{equation}
%
This can be summarized as
%
\begin{equation} 
  \bzP_V^+ !V_0 = 0
\end{equation}
%
where 
%
\begin{equation} 
  \bzP_V^+ = \diag\{(1+\sign(\lambda_j))/2\}
\end{equation}
%
is the projection matrix onto the space of incoming waves. As the
$\bzP_V^+$ is a diagonal matrix, with diagonal elements 1 or 0, it
trivially satisfies the projection condition $\bzP_V^+ \bzP_V^+ =
\bzP_V^+$. 

Coming back to the $!U$ basis, we obtain the following first-order,
linear absorbing boundary condition
%
\begin{equation} \label{eq:lin-abc}  
  \bzP_U^+(!U_0) \, (!U(0)-!U_0) = 0,
\end{equation}
%
where
%
\begin{equation} \label{eq:proj-p}  
  \bzP_U^\pm  = !S \, \bzP_V^\pm !S\muno
\end{equation}
%
This condition is perfectly absorbing for small amplitude waves around
the state $!U_0$. The main problem with it is that, as the limit state
at the boundary $!U_\infty$ is not known a priori, we have to use some
\emph{a priori} chosen state $!U_0\neq !U_\infty$ and then, the
projection matrix used $\bzP_U^+(!U^0)$ will not be equal to
$\bzP_U^+(!U_\infty)$, and then not fully absorbing.  We call $!U_0$
the reference state for the absorbing boundary condition. It can
even happen that the eigenvalues for the actual state at the boundary
change sign with respect to the reference state. 

%<*>---<*>---<*>---<*>---<*>---<*>---<*>---<*>---<*>---<*>---<*> 
\SSSection{Riemann based absorbing boundary conditions} 

Let $n^+$ ($n^-)$ be the number of incoming (outgoing) waves, i.e. the
number of positive (negative) eigenvalues of $A_0$, and assume that
the eigenvalues are decreasingly ordered, i,e,
$\lambda_j\ge\lambda_k$, if $j<k$. So that the positive eigenvalues
are the first $n^+$ ones. The boundary conditions for the incoming
waves (\ref{eq:v-abc}) can be written as
%
\begin{equation} \label{eq:l-abc}  
  !l_j.(!U-!U_0) = 0, \ \ \ j=1,\dots,n^+
\end{equation}
%
where $!l_j$ is a row of $!S\muno$, i.e. a \emph{``left eigenvalue''}
of $!A_0$. If $!U$ is close to $!U_0$ we can write (\ref{eq:l-abc}) as 
%
\begin{equation} \label{eq:abc-ri-dif}  
  !l_j(!U).\di!U = 0, \ \ \ j=1,\dots,n^+.
\end{equation}
%
If this differential forms were \emph{``exact differentials''},
i.e. if 
%
\begin{equation} \label{eq:abc-ri-def}   
  !l_j(!U).\di!U = \di w_j(!U), 
  \text{ for all } \ \ \ j=1,\dots,\ndof,
\end{equation}
%
for some functions $w_j(!U)$, then we could impose as absorbing
boundary conditions
%
\begin{equation} \label{eq:ri-abc}  
  w_j(!U) = w_j(!U_0), \ \ \ j=1,\dots,n^+. 
\end{equation}
%
Let's call to these $w_j$ functions \emph{``invariants''}.  As there
are $\ndof$ invariants, we can define as a new representation of the
internal state the $!w$ variables.  Note that, as
$\deps{!w}{!U}=S\muno$, and $!S$ is non-singular, due to the
hyperbolicity of the system, the correspondence between $!w$ and $!U$
is one-to-one.  

Assume that the value at the boundary reaches a steady limit value of
$!U_\infty$, i.e.
%
\begin{equation} 
!U(0,t)\to!U_\infty,  \text{ for } \ \ \ t\to\infty
\end{equation}
%
If all the waves were incoming ($n^+=\ndof$), then the set of boundary
conditions (\ref{eq:ri-abc}) would be a set of $\ndof$ non-linear
equations on the value $!U_\infty$. As the correspondence $!U\to!w$ is
one to one, the boundary conditions would mean $!w(!U_\infty) =
!w(!U_0)$, and then $!U_\infty=!U_0$. But if the number of incoming
waves is $n^+<\ndof$, then it could happen that $!U_\infty\neq
!U_0$. In fact, for a given $U_0$, the limit value $!U_\infty$ would
belong to a curved $n^-$-dimensional curvilinear manifold. Even if the
limit state $!U_\infty!=!U_0$, we can proved to be perfectly
absorbing, since, as $!U\to!U_\infty$ at the boundary, we can expand
each of the conditions around $!U_\infty$ and it would result in an
equation similar to (\ref{eq:abc-ri-dif}) but centered about
$!U_\infty$, 
%
\begin{equation}
  !l_j(!U_\infty).(!U-!U_\infty) = 0, \ \ \ j=1,\dots,n^+.
\end{equation}

The problem is that in general the differentials are not
exact. \emph{``Riemann invariants''} are functions that satisfy
(\ref{eq:abc-ri-def}) under some restrictions on the flow. For
instance, Riemann invariants can be computed for compressible gas flow
if we assume that the flow is isentropic. They are
%
\begin{equation} 
  \begin{aligned} 
    w_1 &= s = \log (p/\rho^\gamma), & \lambda_1&=u, & &\text{(entropy)}; \\
    w_{2,3} &= u\pm\frac{2a}{\gamma-1}, & \lambda_{2,3}&=u\pm a, &
    &\text{(acoustic waves)};\\
    w_{4,5} &= !u\cdot{\versor t}_{1,2};  & \lambda_{4,5}&=u, &
    &\text{(vorticity waves)}.\\
  \end{aligned}
\end{equation}
%
Boundary conditions based on Riemann invariants are, then, absorbing
in some particular cases. 

%<*>---<*>---<*>---<*>---<*>---<*>---<*>---<*>---<*>---<*>---<*> 
\SSSection{Absorbing boundary conditions based on last state} 

Another possibility is to linearize around the last state $!U^n$, i.e.
%
\begin{equation} \label{eq:lin-abc-n}  
  \bzP_U^+(!U(0,t^n)) \, (!U(0,t^{n+1})-!U(0,t^n)) = 0. 
\end{equation}
%
This equation is always perfectly absorbing in the limit, because we
are always linearizing about a state that, in the limit, will tend to
$!U_\infty$ and doesn't need the computation of Riemann invariants
(which could be unknown for certain problems). Also, this boundary
condition is fully absorbing even in the case of inversion of the
sense of propagation of waves. 

The drawback is that we have no external control on the internal
states, i.e. the limit internal state does not depend on some external
value (as the $!U_0$ used for the Riemann based absorbing boundary
conditions), but on the internal state. That means, for instance, that
if the internal computational scheme tends to produce some error (due
to non conservativity, or rounding errors), the internal state would
drift constantly.

A good compromise may be to use Riemann based (\ref{eq:ri-abc}) or
linear absorbing boundary conditions (\ref{eq:lin-abc}) at
inlet and absorbing boundary conditions based on last state
(\ref{eq:lin-abc-n}) at outlet. As the error tends to propagate more
intensely towards the outlet boundary, it is preferably to use
strongly absorbing boundary conditions there, whereas the linearly
absorbing or Riemann invariant boundary conditions upstream avoid the
drift.

%<*>---<*>---<*>---<*>---<*>---<*>---<*>---<*>---<*>---<*>---<*> 
\SSSection{Finite element setup} 

Assume that the problem is 1D, with a constant mesh size $h$ and time
step $\Dt$, so that nodes are located at positions $x_k=kh$,
$k=0,\dots,\infty$. Let $!U_j^n$ be the state at node $j$, time step
$n$. FEM discretisation of the system of equations with
\emph{``natural''} boundary conditions leads to a system of the form
%
\begin{equation} \label{eq:fem-eq}  
  \begin{aligned} 
    !F_0(!U_0^{n+1},!U_1^{n+1}) &= !R_0^{n+1}\\
    !F_1(!U_0^{n+1},!U_1^{n+1},!U_2^{n+1}) &= !R_1^{n+1}\\
    \vdots \ \ \ \ &= \ \ \ \ \vdots\\
    !F_k(!U_{k-1}^{n+1},!U_k^{n+1},!U_{k+1}^{n+1}) &= !R_k^{n+1}\\
    \vdots \ \ \ \ &= \ \ \ \ \vdots\\
  \end{aligned}
\end{equation}
%
where the $!F_k()$ are non-linear functions and the $!R_k$ possibly
depends on the previous values $!U_k^n$. \emph{``Imposing boundary conditions''}
means to replace some of the equations in the first row ($k=0$) for
other equations. Recall that, in order to balance the number of
equations and unknowns, we must specify which of the equations are
discarded for each equations that is added. For instance, when
imposing conditions on primitive variables it is usual to discard the
energy equation if pressure is imposed, to discard the continuity
equation if density is imposed and to discard the $j$-th component of
the momentum equation if the $j$-th component of velocity is imposed. 
On solid walls, the energy equation is discarded if temperature is
imposed. Some of these \emph{``pairings''} equation/unknown are more
clear than others. 

So that, when imposing absorbing boundary conditions we have not only
to specify the new conditions, but also which equations are
discarded. Note that this is not necessary if all the variables are
imposed at the boundary, for instance in a supersonic outlet. This
suggests to generate appropriate values even for the outgoing waves,
for instance, by extrapolation from the interior values. 

%<*>---<*>---<*>---<*>---<*>---<*>---<*>---<*>---<*>---<*>---<*> 
\SSSection{Extrapolation form interiors} 

For a linear system in characteristic variables (\ref{eq:abc-chareq})
we could replace all the first row of equations by
%
\begin{equation} 
    v^{n+1}_{j0} = 
    \begin{cases} 
      0;& j=1,..,n^+\\
      \sum_{p=0}^m c_p v^n_{jp}; &j=n^+ + 1,\dots,\ndof\\
    \end{cases};
\end{equation}
%
which can be put in matricial form as
%
\begin{equation} 
  \begin{aligned} 
    \bzP_V^+!V^{n+1}_0 &= 0\\
    \bzP_V^-(!V^{n+1}_0 - \sum_{p=0}^m c_p !V^n_p)&= 0\\
  \end{aligned}
\end{equation}
%
where the $c_p$'s are appropriate coefficients that provide an
extrapolation to the value $!v^{n+1}_0$ from the values at time
$!v^{n+1}_0$ Note that these represents $2\ndof$ equations, but as
$\bzP_V^\pm$ have rank $n^\pm$ there are, in total, $\ndof$ linearly
independent equations. As the $n^++1\le j \le\ndof $ rows in the first
row of equations (corresponding to incoming waves) are null and
\emph{vice versa} for the outgoing waves, we can add both blocks of
equations to add a single set of $\ndof$ equations
%
\begin{equation} 
    \bzP_V^+!V^{n+1}_0 +
    \bzP_V^-(!V^{n+1}_0 - \sum_{p=0}^m c_p !V^n_p)= 0
\end{equation}
%
and, coming back to the $!U$ basis
%
\begin{equation} 
    \bzP_U^+(!U^{n+1}_0 - !U_0) +
    \bzP_U^-(!U^{n+1}_0 - \sum_{p=0}^m c_p !U^n_p)= 0
\end{equation}
%
The modified version of the FEM system of equations (\ref{eq:fem-eq})
that incoroporates the absorbing boundary conditions is, then 
%
\begin{equation} 
  \begin{aligned} 
    \bzP_U^+(!U^{n+1}_0 - !U_0) +
    \bzP_U^-(!U^{n+1}_0 - \sum_{p=0}^m c_p !U^n_p)&= 0\\
    !F_1(!U_0^{n+1},!U_1^{n+1},!U_2^{n+1}) &= !R_1^{n+1}\\
    \vdots \ \ \ \ &= \ \ \ \ \vdots\\
    !F_k(!U_{k-1}^{n+1},!U_k^{n+1},!U_{k+1}^{n+1}) &= !R_k^{n+1}\\
    \vdots \ \ \ \ &= \ \ \ \ \vdots\\
  \end{aligned}
\end{equation}

%<*>---<*>---<*>---<*>---<*>---<*>---<*>---<*>---<*>---<*>---<*> 
\SSSection{Avoiding extrapolation} 

For linear systems, equation (\ref{eq:fem-eq}) is of the form
%
\begin{equation} 
  \begin{aligned} 
    \frac{!U_0^{n+1}-!U_0^{n}}{\Dt} +
    !A\,\frac{!U_1^{n+1}-!U_{0}^{n}}{h} &= 0; \\
    \frac{!U_k^{n+1}-!U_k^{n}}{\Dt} +
    !A\,\frac{!U_{k+1}^{n+1}-!U_{k-1}^{n}}{2h} &= 0, \ \ \ k\ge 1\\
  \end{aligned}
\end{equation}
%
We have made a lot of simplifications here, no source or upwind terms,
and a simple discretization based on centered finite
differences. Alternatively, it can be thought as a pure Galerkin FEM
discretization with mass lumping. In the base of the characteristic
variables $!V$ this could be written as
%
\begin{equation} \label{eq:abc-char}  
  \begin{aligned} 
    \frac{!V_0^{n+1}-!V_0^{n}}{\Dt} +
    \bzL\,\frac{!V_1^{n+1}-!V_{0}^{n}}{h} &= 0; \\
    \frac{!V_k^{n+1}-!V_k^{n}}{\Dt} +
    \bzL\,\frac{!V_{k+1}^{n+1}-!V_{k-1}^{n}}{h} &= 0,\ \ \ k\ge 1. \\
  \end{aligned}
\end{equation}
%
For the linear absorbing boundary conditions (\ref{eq:lin-abc}) we
should impose
%
\begin{equation} 
  \bzP_V^+(!V_\reff) \, (!V_0-!V_\reff) = 0. 
\end{equation}
%
while discarding the equations corresponding to the incoming waves in
the first rows of (\ref{eq:abc-char}). Here $!U_\reff/!V_\reff$ is the state
about which we make the linearization. This can be done, via Lagrange
multipliers in the following way
%
\begin{equation} \label{eq:abc-char-v}  
  \begin{aligned} 
    \bzP_V^+(!V_\reff) \, (!V_0-!V_\reff) 
    + \bzP_V^-(!V_\reff) \, !V_{lm}&= 0,\\
    \frac{!V_0^{n+1}-!V_0^{n}}{\Dt} +
    \bzL\,\frac{!V_1^{n+1}-!V_{0}^{n}}{h}  
    +\bzP_V^+ \, !V_{lm}&= 0; \\
    \frac{!V_k^{n+1}-!V_k^{n}}{\Dt} +
    \bzL\,\frac{!V_{k+1}^{n+1}-!V_{k-1}^{n}}{2h} &= 0,\ \ \ k\ge 1. \\
  \end{aligned}
\end{equation}
%
where $!V_{lm}$ are the Lagrange multipliers for imposing the new
conditions. Note that, if $j$ is an incoming wave ($\lambda_j>=0$),
then the equation is of the form
%
\begin{equation} 
  \begin{aligned} 
    v_{j0}-v_{\reff0} &= 0\\
    \frac{v_{j0}^{n+1}-v_{j0}^{n}}{\Dt} 
    + \lambda_j \frac{v_{j1}^{n+1}-v_{j0}^{n}}{h} + 
    v_{j,lm} &= 0\\
    \frac{v_{jk}^{n+1}-v_{jk}^{n}}{\Dt} 
    + \lambda_j \frac{v_{j,k+1}^{n+1}-v_{jk}^{n}}{2h} &= 0 ,\ \ \
    k\ge 1
  \end{aligned}
\end{equation}
%
Note that, due to the $v_{j,lm}$ Lagrange multiplier, we can solve for
the $v_{jk}$ values from the first last rows, the value of the
multiplier $v_{j,lm}$ \emph{``adjusts''} itself in order to relax the
equations in the second row. 

On the other hand, for the outgoing waves ($\lambda_j<0$), we have 
%
\begin{equation} 
  \begin{aligned} 
    v_{j,lm} &= 0\\
    \frac{v_{j0}^{n+1}-v_{j0}^{n}}{\Dt} 
    + \lambda_j \frac{v_{j1}^{n+1}-v_{j0}^{n}}{h} &= 0\\
    \frac{v_{jk}^{n+1}-v_{jk}^{n}}{\Dt} 
    + \lambda_j \frac{v_{j,k+1}^{n+1}-v_{jk}^{n}}{2h} &= 0 ,\ \ \
    k\ge 1
  \end{aligned}
\end{equation}
%
So that the solution coincides with the unmodified orginal FEM
equation, and $v_{j,lm}=0$. 

Coming back to the $!U$ basis, we have
%
\begin{equation} \label{eq:abc-char-u}  
  \begin{aligned} 
    \bzP_U^+(!U_\reff) \, (!U_0-!U_\reff) 
    + \bzP_U^-(!U_\reff) \, !U_{lm}&= 0,\\
    \frac{!U_0^{n+1}-!U_0^{n}}{\Dt} +
    !A\,\frac{!U_1^{n+1}-!U_{0}^{n}}{h}  
    +\bzP_U^+ \, !U_{lm}&= 0; \\
    \frac{!U_k^{n+1}-!U_k^{n}}{\Dt} +
    !A\,\frac{!U_{k+1}^{n+1}-!U_{k-1}^{n}}{2h} &= 0,\ \ \ k\ge 1. \\
  \end{aligned}
\end{equation}
%
And finally, coming back to the FEM equations (\ref{eq:fem-eq}),
%
\begin{equation} \label{eq:fem-eq-abso}  
  \begin{aligned} 
    \bzP_U^+(!U_\reff) \, (!U_0-!U_\reff) 
    + \bzP_U^-(!U_\reff) \, !U_{lm}&= 0,\\
    !F_0(!U_0^{n+1},!U_1^{n+1}) +\bzP_U^+ \, !U_{lm}
    &= !R_0^{n+1}\\
    !F_1(!U_0^{n+1},!U_1^{n+1},!U_2^{n+1}) &= !R_1^{n+1}\\
    \vdots \ \ \ \ &= \ \ \ \ \vdots\\
    !F_k(!U_{k-1}^{n+1},!U_k^{n+1},!U_{k+1}^{n+1}) &= !R_k^{n+1}\\
    \vdots \ \ \ \ &= \ \ \ \ \vdots\\
  \end{aligned}
\end{equation}

In conclusion, in this setup we do not need to make extrapolations to
the variables, and then there is no need to have a structured line of
nodes near the boundary. It's only required to have an additional
fictitious node at the boundary in order to hold the Lagrange
multiplier unknowns $U_{lm}$, and to add the absorbing boundary
equation (first row of (\ref{eq:fem-eq-abso}) for these nodes.  


%<*>---<*>---<*>---<*>---<*>---<*>---<*>---<*>---<*>---<*>---<*> 
\SSSection{Flux functions with enthalpy. } 

When the flux function has an enthalpy term that is not the identity,
then the expressions for the change of basis are somewhat modified,
and also the projectors. An advective diffusive-system with a
\emph{``generalized enthalpy function''} $\Hent(!U)$ is an extension
of the form (\ref{eq:advec-eq}) and can be written as
%
\begin{equation} \label{eq:advec-eq-h}  
 \dep{}t \, \Hent(!U)+\dep{\F_i(U)}{x_i} = 0
\end{equation}
%
The heat equation can be naturally put in this way. Also, the gas
dynamics equations for compressible flow can be put in this form if we
put the equations in \emph{``conservative form''} but use the
\emph{``primitive variables''} $!U=[\rho,!u,p]^T$ as the main main
variables for the code. This has the advantage of using a conservative
form of the equations and, at the same time, allows an easy imposition
of Dirichlet boundary conditions that are normally set in terms of the
primitive variables. In this case $!U$ are the primitive variables,
and the generalized enthalpy $\Hent(!U)$ is the vector of conservative
variables. We call the generalized \emph{``heat content matrix''}
$!C_p$ as 
%
\begin{equation} 
  !C_p = \dep{\Hent(!U)}{!U}
\end{equation}
%
and (\ref{eq:advec-eq}) can be put in quasi-linear form as
%
\begin{equation} \label{eq:advec-eq-cp}  
 !C_p \, \dep{!U}t+!A_i \, \dep{U}{x_i} = 0
\end{equation}
%
Note that this can be brought to the quasi-linear form
(\ref{eq:advec-lin}) (i.e., without the $!C_p$) if we multiply the
equation at left by $!C_p\muno$ and define new flux Jacobians as 
%
\begin{equation} \label{eq:abc-modif-jaco}  
  \tilde{!A_i}= !C_p\muno!A_i,
\end{equation}
%
So that, basically, the extension to systems with generalized enthalpy
is to replace the Jacobians, by the modified Jacobians
(\ref{eq:abc-modif-jaco}). The modified expression for the projectors
is 
%
\begin{equation} \label{eq:proj-p-cp}  
  \bzP_U^\pm  = !C_p \, !S \, \bzP_V^\pm !S\muno
\end{equation}

%<*>---<*>---<*>---<*>---<*>---<*>---<*>---<*>---<*>---<*>---<*>
\SSSection{Absorbing boundary conditions available} 

At the moment of writing this, we have three possible combinations of
boundary conditions. 

%<*>---<*>---<*>---<*>---<*>---<*>---<*>---<*>---<*>---<*>---<*> 
\SSSSection{Using extrapolation from the interior. }
%
These is the elemset \ttsl{<system>}{\tt \_abso}. The number of nodes
per element $\nel$ must be not lower than 4. The first $\nel-2$ nodes
are used for a second order extrapolation of the outgoing wave. The
$\nel-1$-th node is the node with Lagrange multipliers, and the
$\nel$-th node is used to set the reference value. For instance, for
an absorbing elment of $\nel=5$ nodes, we would have 3 internal nodes,
and the data would loke like this
%
 
%<*>---<*>---<*>---<*>---<*>---<*>---<*>---<*>---<*>---<*>---<*> 
\begin{figure*}[htb]
\centerline{\includegraphics{./OBJ/abso.\EPSPDF}}
\caption{Absorbing element. }
\label{fg:abso}
\end{figure*}
%<*>---<*>---<*>---<*>---<*>---<*>---<*>---<*>---<*>---<*>---<*> 

\begin{alltt} 
elemset gasflow_abso 5
normal \Arg{nx} \Arg{ny}
__END_HASH__
\Arg{n1} \Arg{n2} \Arg{n3} \Arg{n4} \Arg{n5}
...
__END_ELEMSET__

end_elemsets

fixa
\Arg{n4} 1 \Arg{rho\_ref}
\Arg{n4} 2 \Arg{u\_ref}
\Arg{n4} 3 \Arg{v\_ref}
\Arg{n4} 4 \Arg{p\_ref}
__END_FIXA__
\end{alltt}

\begin{itemize}
\compactlist 
\item Each element has 5 nodes, first three are real nodes (i.e. not
  fictitious) numbered from the boundary to the interior. Fourth node
  is reserved for Lagrange multipliers, and the fifth node is set to
  the reference value. 
\item The \verb+normal+ option is used to define the normal to the
  boundary. It can be set as a constant vector per elemset (as in the
  example above), or as a per element value. In this last case we
  would have something like this

\begin{alltt} 
elemset gasflow_abso 5
props normal[2]
normal \Arg{nx} \Arg{ny}
__END_HASH__
\Arg{n1} \Arg{n2} \Arg{n3} \Arg{n4} \Arg{n5} \Arg{nx} \Arg{ny}
...
__END_ELEMSET__
\end{alltt}

\item Note the the {\tt fixa} section for the values of the reference
  node. In this case (gas dynamics, elemset \verb+gasflow+) we set the
  four fields ($\ndim=2$) to the reference values. 
\item The reference values can be made time dependent in an explicit
  way by using a \verb+fixa_amplitude+ section instead of a
  \verb+fixa+ section. 
\item Using this absorbing boundary condition requires the flow to
  have implemented the \verb+Riemman_Inv()+ method. If this is not the
  case, then the program will stop with a message like 
\item \emph{For the {\tt gasflow} elemset:} If the option
  \verb+linear_abso+ is set to false (0), then the Riemman invariants
  for gas dynamics are used, and the state reference value is used for
  computing the reference Riemman invariants. If \verb+linear_abso+ is
  set to true (1), then the linear absorbing boundary conditions are
  imposed. 
\end{itemize}
 
%<*>---<*>---<*>---<*>---<*>---<*>---<*>---<*>---<*>---<*>---<*> 
\SSSSection{Not using extrapolation from the interior. }

These is the elemset \ttsl{<system>}{\tt \_abso2}. The number of nodes
per element $\nel$ is 3. The first node is the node at the boundary. The
second node is the node with Lagrange multipliers, and the
third node is used to set the reference value. 
%
 
%<*>---<*>---<*>---<*>---<*>---<*>---<*>---<*>---<*>---<*>---<*> 
\begin{figure*}[htb]
\centerline{\includegraphics{./OBJ/abso.\EPSPDF}}
\caption{Absorbing element. }
\label{fg:abso}
\end{figure*}
%<*>---<*>---<*>---<*>---<*>---<*>---<*>---<*>---<*>---<*>---<*> 




% Local Variables: *
% mode: latex *
% latex-occur-section-regexp: "^\\\\S*ection" *
% tex-main-file: "petscfem.tex" *
% End: *
