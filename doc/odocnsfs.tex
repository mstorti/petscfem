%-----<>-----<>-----<>-----<>-----<>-----<>-----<>-----<>-----<>-----<>
% DON'T EDIT MANUALLY THIS FILE !!!!!!
% This files automatically generated by odoc.pl from 
% source file "../../doc/nsdoc.tex"
%-----<>-----<>-----<>-----<>-----<>-----<>-----<>-----<>-----<>-----<>
\index{LES@\verb+LES+}
\item\verb+int LES+ {\rm(default=\verb|0|)}:

Add LES for this particular elemset.

%----<>----<>----<>----<>----<>----<>----<>----<>----<>----<>
\index{fs\_eq\_factor@\verb+fs_eq_factor+}
\item\verb+double fs_eq_factor+ {\rm(default=\verb|1.|)}:

$\Cnst{eq}$={\tt fs\_eq\_factor} (see doc for {\tt
free\_surface\_damp} option) is a factor that scales the free
surface ``rigidity''. $\Cnst{eq}=1$ (which is the default) means
no scaling, a zero value means infinitely rigid (as for an
inifinite gravity).

%----<>----<>----<>----<>----<>----<>----<>----<>----<>----<>
\index{free\_surface\_damp@\verb+free_surface_damp+}
\item\verb+double free_surface_damp+ {\rm(default=\verb|0.|)}:

$\Cnst{lf}=$\altt{free\_surface\_set\_level\_factor} tries to
keep the free surface level constant by adding a term $\propto
\bar\eta$ to the free surface level.  (see doc //for {\tt
free\_surface\_damp}})
\label{sec:free_surface}
The equation of the free surface is
%
\begin{equation} 
\dtots\eta t = w
\end{equation}
%
where $\eta$ 
elevation, and $w$ is the velocity component normal to the free
surface. We modify this as follows 
%
\begin{equation} \label{eq:fsmod}  
\Cnst{eq}\,\dtotd\eta t 
   + \Cnst{lf} \bar{\eta} + \Cnst{damp} \, \Delta \eta = w
\end{equation}
%
Where $\bar\eta$ is the average value of eta on the free surface, and:
$\Cnst{damp}={\tt free\_surface\_damp}$ smoothes the free surface
adding a Laplacian filter.  Note that if only the temporal derivative
and the Laplace term are present in (\ref{eq:fsmod}) then the equation
is a heat equation. A null value (which is the default) means no
filtering. A high value means high filtering. (Warning: A high value
may result in unstability).  $\Cnst{damp}$ has dimensions of $L^2/T$
(like a diffusivity). One possibility is to scale with mesh parameters
like $h^2/\Delta t$, other is to scale with $h^1.5 \,
g^0.5$. Currently, we are using $\Cnst{damp} = \Cnst{damp}' h^1.5 *
g^0.5$ with $\Cnst{damp}'\approx 2$.

%----<>----<>----<>----<>----<>----<>----<>----<>----<>----<>
\index{free\_surface\_set\_level\_factor@\verb+free_surface_set_level_factor+}
\item\verb+double free_surface_set_level_factor+ {\rm(default=\verb|0.|)}:

This adds a $\Cnst{lf}\eta$ term in the free surface equation
in order to have the total meniscus volume constant. 

%----<>----<>----<>----<>----<>----<>----<>----<>----<>----<>
\index{npg@\verb+npg+}
\item\verb+int npg+ {\rm(default=\verb|none|)}:

Number of Gauss points.

%----<>----<>----<>----<>----<>----<>----<>----<>----<>----<>
\index{geometry@\verb+geometry+}
\item\verb+string geometry+ {\rm(default=\verb|cartesian2d|)}:

Type of element geometry to define Gauss Point data

%----<>----<>----<>----<>----<>----<>----<>----<>----<>----<>
%-----<>-----<>-----<>-----<>-----<>-----<>-----<>-----<>-----<>-----<>
% DON'T EDIT MANUALLY THIS FILE !!!!!!
% This files automatically generated by odoc.pl from 
% source file "../../doc/nsdoc.tex"
%-----<>-----<>-----<>-----<>-----<>-----<>-----<>-----<>-----<>-----<>
