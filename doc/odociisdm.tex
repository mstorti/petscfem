%-----<>-----<>-----<>-----<>-----<>-----<>-----<>-----<>-----<>-----<>
% DONT EDIT MANUALLY THIS FILE !!!!!!
% This files automatically generated by odoc.pl from 
% source file ""
%-----<>-----<>-----<>-----<>-----<>-----<>-----<>-----<>-----<>-----<>
\index{asm\_lblocks@\verb+asm_lblocks+}
\item\verb+int asm_lblocks+ {\rm(default=\verb|1|)}:

Chooses the number of local blocks in ASM
 (found in file: \verb+iisdmat.cpp+)
%----<>----<>----<>----<>----<>----<>----<>----<>----<>----<>
\index{asm\_overlap@\verb+asm_overlap+}
\item\verb+int asm_overlap+ {\rm(default=\verb|1|)}:

Chooses the overlap of blocks in ASM
 (found in file: \verb+iisdmat.cpp+)
%----<>----<>----<>----<>----<>----<>----<>----<>----<>----<>
\index{asm\_sub\_ksp\_type@\verb+asm_sub_ksp_type+}
\item\verb+string asm_sub_ksp_type+ {\rm(default=\verb|preonly|)}:

Chooses the preconditioning for block problems in ASM method.
 (found in file: \verb+iisdmat.cpp+)
%----<>----<>----<>----<>----<>----<>----<>----<>----<>----<>
\index{asm\_sub\_preco\_type@\verb+asm_sub_preco_type+}
\item\verb+string asm_sub_preco_type+ {\rm(default=\verb|ilu|)}:

Chooses the preconditioning for block problems in ASM method.
 (found in file: \verb+iisdmat.cpp+)
%----<>----<>----<>----<>----<>----<>----<>----<>----<>----<>
\index{asm\_type@\verb+asm_type+}
\item\verb+string asm_type+ {\rm(default=\verb|restrict|)}:

Chooses the restriction/extension type in ASM
 (found in file: \verb+iisdmat.cpp+)
%----<>----<>----<>----<>----<>----<>----<>----<>----<>----<>
\index{atol@\verb+atol+}
\item\verb+double atol+ {\rm(default=\verb|1e-6|)}:

Absolute tolerance to solve the monolithic linear
system (Newton linear subiteration).
 (found in file: \verb+iisdmat.cpp+)
%----<>----<>----<>----<>----<>----<>----<>----<>----<>----<>
\index{compact\_profile\_graph\_chunk\_size@\verb+compact_profile_graph_chunk_size+}
\item\verb+int compact_profile_graph_chunk_size+ {\rm(default=\verb|0|)}:

Size of chunk for the dynamic vector used in computing the
mstrix profile. 
 (found in file: \verb+iisdmat.cpp+)
%----<>----<>----<>----<>----<>----<>----<>----<>----<>----<>
\index{dtol@\verb+dtol+}
\item\verb+double dtol+ {\rm(default=\verb|1e+3|)}:

Divergence tolerance to solve the monolithic linear
system (Newton linear subiteration).
 (found in file: \verb+iisdmat.cpp+)
%----<>----<>----<>----<>----<>----<>----<>----<>----<>----<>
\index{gmres\_orthogonalization@\verb+gmres_orthogonalization+}
\item\verb+string gmres_orthogonalization+ {\rm(default=\verb|modified_gram_schmidt|)}:

Orthogonalization method used in conjunction with GMRES. 
May be  {\tt unmodified\_gram\_schmidt},
 \verb+modified_gram_schmidt+  or {\tt ir\_orthog} (default). (Iterative refinement).
See PETSc documentation. 
 (found in file: \verb+iisdmat.cpp+)
%----<>----<>----<>----<>----<>----<>----<>----<>----<>----<>
\index{Krylov\_dim@\verb+Krylov_dim+}
\item\verb+int Krylov_dim+ {\rm(default=\verb|50|)}:

Krylov space dimension in solving the monolithic linear
system (Newton linear subiteration) by GMRES.
 (found in file: \verb+iisdmat.cpp+)
%----<>----<>----<>----<>----<>----<>----<>----<>----<>----<>
\index{KSP\_method@\verb+KSP_method+}
\item\verb+string KSP_method+ {\rm(default=\verb|fgmres|)}:

Defines the KSP method
 (found in file: \verb+iisdmat.cpp+)
%----<>----<>----<>----<>----<>----<>----<>----<>----<>----<>
\index{maxits@\verb+maxits+}
\item\verb+int maxits+ {\rm(default=\verb|Krylov_dim|)}:

Maximum iteration number in solving the monolithic linear
system (Newton linear subiteration).
 (found in file: \verb+iisdmat.cpp+)
%----<>----<>----<>----<>----<>----<>----<>----<>----<>----<>
\index{preco\_side@\verb+preco_side+}
\item\verb+string preco_side+ {\rm(default=\verb|<ksp-dependent>|)}:

Uses right or left preconditioning. Default is  \verb+right+  for
GMRES. 
 (found in file: \verb+iisdmat.cpp+)
%----<>----<>----<>----<>----<>----<>----<>----<>----<>----<>
\index{preco\_type@\verb+preco_type+}
\item\verb+string preco_type+ {\rm(default=\verb|jacobi|)}:

Chooses the preconditioning operator. 
 (found in file: \verb+iisdmat.cpp+)
%----<>----<>----<>----<>----<>----<>----<>----<>----<>----<>
\index{print\_fsm\_transition\_info@\verb+print_fsm_transition_info+}
\item\verb+int print_fsm_transition_info+ {\rm(default=\verb|0|)}:

Print Finite State Machine transitions for PFPETScMat matrices.
1: print inmediately, 2: gather events (non immediate printing). 
 (found in file: \verb+iisdmat.cpp+)
%----<>----<>----<>----<>----<>----<>----<>----<>----<>----<>
\index{print\_internal\_loop\_conv@\verb+print_internal_loop_conv+}
\item\verb+int print_internal_loop_conv+ {\rm(default=\verb|0|)}:

Prints convergence in the solution of the GMRES iteration. 
 (found in file: \verb+iisdmat.cpp+)
%----<>----<>----<>----<>----<>----<>----<>----<>----<>----<>
\index{rtol@\verb+rtol+}
\item\verb+double rtol+ {\rm(default=\verb|1e-3|)}:

Relative tolerance to solve the monolithic linear
system (Newton linear subiteration).
 (found in file: \verb+iisdmat.cpp+)
%----<>----<>----<>----<>----<>----<>----<>----<>----<>----<>
\index{use\_compact\_profile@\verb+use_compact_profile+}
\item\verb+int use_compact_profile+ {\rm(default=\verb|LINK_GRAPH|)}:

Choice representation of the profile graph. Possible values are:
0) Adjacency graph classes
based on STL map+set, demands too much memory, CPU time OK.
1) Based on dynamic vector of pair of indices with resorting,
demands too much CPU time, RAM is OK
2) For each vertex wee keep a linked list of cells containing the
adjacent nodes. Each insertion is $O(m^2)$ where $m$ is the average
number of adjacent vertices. This seems to be optimal for
FEM connectivities.
 (found in file: \verb+iisdmat.cpp+)
%----<>----<>----<>----<>----<>----<>----<>----<>----<>----<>
%-----<>-----<>-----<>-----<>-----<>-----<>-----<>-----<>-----<>-----<>
% DONT EDIT MANUALLY THIS FILE !!!!!!
% This files automatically generated by odoc.pl from 
% source file ""
%-----<>-----<>-----<>-----<>-----<>-----<>-----<>-----<>-----<>-----<>
