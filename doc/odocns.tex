%-----<>-----<>-----<>-----<>-----<>-----<>-----<>-----<>-----<>-----<>
% DON'T EDIT MANUALLY THIS FILE !!!!!!
% This files automatically generated by odoc.pl from 
% source file "../../doc/nsdoc.tex"
%-----<>-----<>-----<>-----<>-----<>-----<>-----<>-----<>-----<>-----<>
\index{stdout\_file@\verb+stdout_file+}
\item\verb+string stdout_file+ {\rm(default=\verb||)}:

If set, redirect output to this file.

%----<>----<>----<>----<>----<>----<>----<>----<>----<>----<>
\index{fractional\_step@\verb+fractional_step+}
\item\verb+int fractional_step+ {\rm(default=\verb|0|)}:

Use fractional step or TET algorithm

%----<>----<>----<>----<>----<>----<>----<>----<>----<>----<>
\index{reuse\_mat@\verb+reuse_mat+}
\item\verb+int reuse_mat+ {\rm(default=\verb|0|)}:

Use fractional step or TET algorithm

%----<>----<>----<>----<>----<>----<>----<>----<>----<>----<>
\index{activate\_debug@\verb+activate_debug+}
\item\verb+int activate_debug+ {\rm(default=\verb|0|)}:

Activate debugging

%----<>----<>----<>----<>----<>----<>----<>----<>----<>----<>
\index{activate\_debug\_print@\verb+activate_debug_print+}
\item\verb+int activate_debug_print+ {\rm(default=\verb|0|)}:

Activate printing in debugging

%----<>----<>----<>----<>----<>----<>----<>----<>----<>----<>
\index{activate\_debug\_memory\_usage@\verb+activate_debug_memory_usage+}
\item\verb+int activate_debug_memory_usage+ {\rm(default=\verb|0|)}:

Activate report of memory usage

%----<>----<>----<>----<>----<>----<>----<>----<>----<>----<>
\index{ndim@\verb+ndim+}
\item\verb+int ndim+ {\rm(default=\verb|3|)}:

Dimension of the problem.

%----<>----<>----<>----<>----<>----<>----<>----<>----<>----<>
\index{displ\_factor@\verb+displ_factor+}
\item\verb+double displ_factor+ {\rm(default=\verb|0.1|)}:

Scales displacement for ALE-like mesh relocation. 

%----<>----<>----<>----<>----<>----<>----<>----<>----<>----<>
\index{nnwt@\verb+nnwt+}
\item\verb+int nnwt+ {\rm(default=\verb|1|)}:

Number of inner iterations for the global non-linear
Newton  problem. 

%----<>----<>----<>----<>----<>----<>----<>----<>----<>----<>
\index{update\_jacobian\_steps@\verb+update_jacobian_steps+}
\item\verb+int update_jacobian_steps+ {\rm(default=\verb|0|)}:

Update jacobian each $n$-th time step. 

%----<>----<>----<>----<>----<>----<>----<>----<>----<>----<>
\index{update\_jacobian\_start\_steps@\verb+update_jacobian_start_steps+}
\item\verb+int update_jacobian_start_steps+ {\rm(default=\verb|INF|)}:

Update jacobian each $n$-th time step. 

%----<>----<>----<>----<>----<>----<>----<>----<>----<>----<>
\index{tol\_newton@\verb+tol_newton+}
\item\verb+double tol_newton+ {\rm(default=\verb|1e-8|)}:

Tolerance to solve the non-linear system (global Newton).

%----<>----<>----<>----<>----<>----<>----<>----<>----<>----<>
\index{update\_jacobian\_iters@\verb+update_jacobian_iters+}
\item\verb+int update_jacobian_iters+ {\rm(default=\verb|1|)}:

Update jacobian only until n-th Newton subiteration. 
Don't update if null. 

%----<>----<>----<>----<>----<>----<>----<>----<>----<>----<>
\index{update\_jacobian\_start\_iters@\verb+update_jacobian_start_iters+}
\item\verb+int update_jacobian_start_iters+ {\rm(default=\verb|INF|)}:

Update jacobian each $n$-th Newton iteration

%----<>----<>----<>----<>----<>----<>----<>----<>----<>----<>
\item\verb+ vector<double>+ \verb+ newton_relaxation_factor+ {\rm(default=\verb| (none)|)}:

Relaxation parameter for Newton iteration. Several
values may be entered in the form
%
\begin{verbatim} 
newton_relaxation_factor w1 n1 w2 n2 .... wn
\end{verbatim}
%
that means: Take relaxation factor \verb+w1+
for the first \verb+n1+ steps, \verb+w2+ for the following \verb+n2+ steps
and so on until \verb+w_{n-1}+. \verb+wn+ is taken for all subsequent 
steps. Normally one takes a conservative (said 0.5) relaxation
factor for the first steps and then let full Newton (i.e. \verb+w=1+)
for the rest. For instance, the line  
%
\begin{verbatim} 
newton_relaxation_factor 0.5 3 1.
\end{verbatim}
%
means: take $w=0.5$ for the first 3 steps, and then use $w=1$. 
%----<>----<>----<>----<>----<>----<>----<>----<>----<>----<>
\index{print\_linear\_system\_and\_stop@\verb+print_linear_system_and_stop+}
\item\verb+int print_linear_system_and_stop+ {\rm(default=\verb|0|)}:

After computing the linear system solves it and prints Jacobian,
right hand side and solution vector, and stops. 

%----<>----<>----<>----<>----<>----<>----<>----<>----<>----<>
\index{print\_residual@\verb+print_residual+}
\item\verb+int print_residual+ {\rm(default=\verb|0|)}:

Print the residual each \verb+nsave+ steps. 

%----<>----<>----<>----<>----<>----<>----<>----<>----<>----<>
\index{solve\_system@\verb+solve_system+}
\item\verb+int solve_system+ {\rm(default=\verb|1|)}:

Solve system before \verb+print\_linear_system_and_stop+

%----<>----<>----<>----<>----<>----<>----<>----<>----<>----<>
\index{measure\_performance@\verb+measure_performance+}
\item\verb+int measure_performance+ {\rm(default=\verb|0|)}:

Measure performance of the 'comp\_mat\_res' jobinfo. 

%----<>----<>----<>----<>----<>----<>----<>----<>----<>----<>
\index{nsave@\verb+nsave+}
\item\verb+int nsave+ {\rm(default=\verb|10|)}:

Sets the save frequency in iterations 

%----<>----<>----<>----<>----<>----<>----<>----<>----<>----<>
\index{nsaverot@\verb+nsaverot+}
\item\verb+int nsaverot+ {\rm(default=\verb|100|)}:

Sets the frequency save for the ``rotary save'' mechanism. 
\label{sec:rotary_save}
Sometimes it is interesting to save the state vector with a certain
frequency in a ``append'' manner, i.e. appending the state vector at
the end of the file. However, this posses the danger of storing too
much amount of data if the user performs a very long run. The ``rotary
save'' mechanism allows writing only a certain amount of the recent
states. The mechanism basically saves the state vector each
\verb+nsaverot+ steps appending to the a file. The name of the file is
contructed from a pattern set by the user via the
\verb+save_file_pattern+ entry, by replacing \verb+%d+ 
by \verb+0+ ``\`a la'' \verb+printf()+. For instance, if
\verb+save_file_pattern+ i set to \verb+file%d.out+ 
then the state vectors are appended to \verb+file0.out+.  When the
number of written states reach the \verb+nrec+ count, the file is
reset to 0, and the saving continues from the start of the
file. However, if \verb+nfile+ is greater than one, then the state
vector are continued to be stored in file \verb+file1.out+ and so
on. When the number of files \verb+nfile+ is reached, the saving
continues in file '0'.

More precisely, the saving mechanism is described by the following
pseudo-code:

% %\begin{latexonly}
% % algorithmic environment is not correctly processed by Latex2HTML
% \begin{algorithmic}
% \STATE read state vector from \verb+initial_state+ file into $x^0$, $n\gets 0$.
% \FOR{$i$=0 to {\tt nstep}}
%   \STATE advance $x^n$ to $x^{n+1}$.
%   \IF{$n$ \% {\tt nsaverot} == 0}
%     \STATE $j \gets n$/{\tt nsaverot}
%     \STATE $k \gets j$ \% {\tt nrec}
%     \STATE $l \gets j$/{\tt nrec}
%     \IF{$k=0$}\STATE rewind file $l$ \ENDIF
%     \STATE append state vector to file $l$
%   \ENDIF
% \ENDFOR
% \end{algorithmic}
% %\end{latexonly}
%%%<>%%%<>%%%<>%%%<>%%%<>%%%<>%%%<>%%%<>%%%<>%%%<>%%%<>%%%<>%%%<> 
% algorithmic environment is not correctly processed by Latex2HTML
% \begin{htmlonly}
\begin{verbatim}
Read state vector from `initial_state' file into x^0, n=0;
for (i=0; i<nstep; i++) {
   advance x^n to x^{n+1};
   if (n % nsaverot == 0) {
           j <- n/nsaverot;
           k <- j % nrec;
           l <- j / nrec;
           if (k==0) { rewind file l; }
           append state vector to file l;
   }
}
\end{verbatim}
% \end{htmlonly}

%----<>----<>----<>----<>----<>----<>----<>----<>----<>----<>
\index{nrec@\verb+nrec+}
\item\verb+int nrec+ {\rm(default=\verb|1000000|)}:

Sets the number of states saved in a given file
in the ``rotary save'' mechanism (see \ref{sec:rotary_save}

%----<>----<>----<>----<>----<>----<>----<>----<>----<>----<>
\index{nfile@\verb+nfile+}
\item\verb+int nfile+ {\rm(default=\verb|1|)}:

Sets the number of files in the ``rotary save'' mechanism. 
(see \ref{sec:rotary_save})

%----<>----<>----<>----<>----<>----<>----<>----<>----<>----<>
\index{nsome@\verb+nsome+}
\item\verb+int nsome+ {\rm(default=\verb|10000|)}:

Sets the save frequency in iterations for the ``print some''
mechanism. 
\label{sec:print_some}
The ``print some'' mechanism allows the user to store the variables of
some set of nodes with some frequency. The nodes are entered in a
separate file whose name is given by a \verb+print_some_file+ entry in
the general options, one node per line. The entry \verb+nsome+
indicates the frequency (in steps) at which the data is saved and
\verb+save_file_some+ the name of the file to save in. 

%----<>----<>----<>----<>----<>----<>----<>----<>----<>----<>
\index{nstep@\verb+nstep+}
\item\verb+int nstep+ {\rm(default=\verb|10000|)}:

The number of time steps. 

%----<>----<>----<>----<>----<>----<>----<>----<>----<>----<>
\index{Dt@\verb+Dt+}
\item\verb+double Dt+ {\rm(default=\verb|0.|)}:

The time step.

%----<>----<>----<>----<>----<>----<>----<>----<>----<>----<>
\index{steady@\verb+steady+}
\item\verb+int steady+ {\rm(default=\verb|0|)}:

Flag if steady solution or not (uses Dt=inf). If \verb+steady+
is set to 1, then the computations are as if $\Dt=\infty$. 
The value of \verb+Dt+ is used for printing etc... If \verb+Dt+
is not set and \verb+steady+ is set then \verb+Dt+ is set to one.

%----<>----<>----<>----<>----<>----<>----<>----<>----<>----<>
\index{alpha@\verb+alpha+}
\item\verb+double alpha+ {\rm(default=\verb|1.|)}:

Trapezoidal method parameter. \verb+alpha=1+:
Backward Euler. \verb+alpha=0+: Forward Euler.
\verb+alpha=0.5+: Crank-Nicholson. 

%----<>----<>----<>----<>----<>----<>----<>----<>----<>----<>
\index{ngather@\verb+ngather+}
\item\verb+int ngather+ {\rm(default=\verb|0|)}:

Number of ``gathered'' quantities.

%----<>----<>----<>----<>----<>----<>----<>----<>----<>----<>
\index{gather\_file@\verb+gather_file+}
\item\verb+string gather_file+ {\rm(default=\verb|"gather.out"|)}:

Print values in this file 

%----<>----<>----<>----<>----<>----<>----<>----<>----<>----<>
\index{LES@\verb+LES+}
\item\verb+int LES+ {\rm(default=\verb|0|)}:

Use the LES/Smagorinsky turbulence model. 

%----<>----<>----<>----<>----<>----<>----<>----<>----<>----<>
\index{A\_van\_Driest@\verb+A_van_Driest+}
\item\verb+int A_van_Driest+ {\rm(default=\verb|0|)}:

If {\tt A\_van\_Driest=0} then the van Driest
   damping factor is not used 

%----<>----<>----<>----<>----<>----<>----<>----<>----<>----<>
\index{use\_iisd@\verb+use_iisd+}
\item\verb+int use_iisd+ {\rm(default=\verb|0|)}:

Use IISD (Interface Iterative Subdomain Direct) or not.

%----<>----<>----<>----<>----<>----<>----<>----<>----<>----<>
\index{solver@\verb+solver+}
\item\verb+string solver+ {\rm(default=\verb|petsc|)}:

Type of solver. May be \verb+iisd+ or \verb+petsc+. 

%----<>----<>----<>----<>----<>----<>----<>----<>----<>----<>
\index{solver\_mom@\verb+solver_mom+}
\item\verb+string solver_mom+ {\rm(default=\verb|petsc|)}:

Type of solver for the projection and momentum steps
(fractional-step). May be \verb+iisd+ or \verb+petsc+.

%----<>----<>----<>----<>----<>----<>----<>----<>----<>----<>
\index{save\_file\_pattern@\verb+save_file_pattern+}
\item\verb+string save_file_pattern+ {\rm(default=\verb|outvector%d.out|)}:

The pattern to generate the file name to save in for
the rotary save mechanism.

%----<>----<>----<>----<>----<>----<>----<>----<>----<>----<>
\index{save\_file@\verb+save_file+}
\item\verb+string save_file+ {\rm(default=\verb|outvector.out|)}:

The name of the file to save the state vector. 

%----<>----<>----<>----<>----<>----<>----<>----<>----<>----<>
\index{print\_some\_file@\verb+print_some_file+}
\item\verb+string print_some_file+ {\rm(default=\verb||)}:

Name of file where to read the nodes for the ``print some'' 
feature. 

%----<>----<>----<>----<>----<>----<>----<>----<>----<>----<>
\index{save\_file\_some@\verb+save_file_some+}
\item\verb+string save_file_some+ {\rm(default=\verb|outvsome.out|)}:

Name of file where to save node values for the ``print some'' 
feature. 

%----<>----<>----<>----<>----<>----<>----<>----<>----<>----<>
\index{save\_file\_some\_append@\verb+save_file_some_append+}
\item\verb+int save_file_some_append+ {\rm(default=\verb|1|)}:

Access mode to the ``some'' file. If 0 rewind file. If 1 
 append to previous  results.

%----<>----<>----<>----<>----<>----<>----<>----<>----<>----<>
\index{report\_option\_access@\verb+report_option_access+}
\item\verb+int report_option_access+ {\rm(default=\verb|1|)}:

Print, after execution, a report of the times a given option
was accessed. Useful for detecting if an option was used or not.

%----<>----<>----<>----<>----<>----<>----<>----<>----<>----<>
%-----<>-----<>-----<>-----<>-----<>-----<>-----<>-----<>-----<>-----<>
% DON'T EDIT MANUALLY THIS FILE !!!!!!
% This files automatically generated by odoc.pl from 
% source file "../../doc/nsdoc.tex"
%-----<>-----<>-----<>-----<>-----<>-----<>-----<>-----<>-----<>-----<>
