%__INSERT_LICENSE__
% $Id$ 
\let\verbb=\Verb

\SSSection{Absorbing/wall boundary conditions} 

In some cases we want to have a boundary condition that switches
between absorbing and wall. This is useful for instance for simulating
a sliding hole in a boundary. This kind of b.c. is implemented with
the \verbb+gasflow_abso_wall+ elemset. It has properties similar to
that elemset, in particular the \verbb+normal+ and \verbb+vmesh+
properties. The difference is that for each boundary node, and at each
time step the b.c. may switch from absorbing to wall
(i.e. $!v=0$ or $!v\cdot\nor=0$). This is activated by implmenting the
virtual function 

\medskip
\begin{Verbatim}
int AdvectiveAbsoWall::
turn_wall_fun(int elem,int node, FastMat2 &x,double t);
\end{Verbatim}
\medskip

At run time this function is called for each boundary element and if
the function returns a true value (1) then the wall boundary condition
is enforced through Lagrange multipliers or penalization (currently
only with Lagrange multipliers). If the result returned is a false
value (0) then the boundary condition is absorbing. 
The arguments passed to the routine can used by the user to compute
the desired value, 
%
\begin{itemize}
\compactlist 
\item \verbb+elem+: the element index (1 base)
\item \verbb+node+: the element index (1 base)
\item \verbb+x+: the coordinates of the boundary node (an \verbb+ndim+
  vector) 
\item \verbb+t+: the current time. 
\end{itemize}
 
For instance if the computational domain is the $0\le x,y\le 1$ square
and the outlet boundary is the segment $x=1, 0\le y\le 1$ then the
following code makes the upper half of the boundary to be wall and the
half lower to be absorbing:

\medskip
\begin{Verbatim}
int AdvectiveAbsoWall::
turn_wall_fun(int elem,int node,
              FastMat2 &x,double t) {
  return x.get(2)>0.5;
}
\end{Verbatim}
\medskip

(See example \verbb+petscfem-cases/gasflow/dyna-abso-wall3+). 

The \verbb+node+ and \verbb+elem+ indices may be used to perform more
elaborate tasks, as loading tables from files, or seek per-element
properties. 

\SSSection{Related Options}

\begin{itemize}
\input odocabsow
\end{itemize}

% Local Variables: *
% mode: latex *
% latex-occur-section-regexp: "^\\\\S*ection" *
% tex-main-file: "petscfem.tex" *
% End: *
