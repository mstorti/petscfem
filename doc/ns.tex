%__INSERT_LICENSE__
\Section{The Navier-Stokes module}

\SSection{LES implementation}

The Smagorisky LES model for the Navier-Stokes module follows ...
%\cite{el libro del %chino }
The implementation under \pfem{} has presents the following
particularities
%
\begin{itemize}
\item Wall boundary conditions are implemented as \emph{``mixed
type''}. 

\item The van Driest damping factor introduces non-localities in
the sense that the turbulent viscosity at a volume element depends on
the state of the fluid at a wall.
\end{itemize}

\SSSection{The wall elemset}

Wall boundary conditions have been implemented via a \verb+wall+
elemset. This is a surface element that computes, given the velocities
at the nodes, the tractions corresponding to this velocities, for a
given law of wall. Also, this shear velocities as stored internally in
the element, so that the volume elements can get them and compute the
van Driest damping factor. This requires to find, for each volume
element, the nearest wall element. This is done \emph{before} the time
loop, with the ANN (Approximate Nearest Neighbor) library. 

\SSSection{The mixed type boundary condition}

The contribution to the momentum equations from the wall element is
%
\begin{equation} 
   R_{ip} = \int_\Se t_p N_i \dS \label{eq:wallres} 
\end{equation}
%
where $R_{ip}$ is the contribution to the residual of the $p$-th
momentum equation of the node $i$. $N_i$ is the shape function of node
$i$ and $t_p$ are the tractions on the surface of the element
$\Se$. The wall law is in general of the form
%
\begin{equation} 
  \frac{u}{\ustar} = f(y^+) \label{eq:walllaw} 
\end{equation}
%
where $u$ is the tangent velocity, $\ustar$ the shear velocity $\ustar
= \sqrt{\tau_w/\rho}$, and $y^+ = y\ustar/\nu$, the non-dimensional
distance to the wall. We have several possibilities regarding the
positioning of the computational boundary. We first discuss the
simplest, that is to set the computational boundary at a fixed $y^+$
position. Note, that this means that the real position od the boundary
$y^+$ changes during iteration. In this case (\ref{eq:walllaw}) can be
rewritten as
%
\begin{equation} 
   \tau_w = g(u) \,  u
\end{equation}
%
where
%
\begin{equation}
  \tau_w = g(u) \, u = \rho \LL(\frac u{f(y^+)}\RR)^2
\end{equation}
%
or
%
\begin{equation} 
   g(u) = \frac\rho{f(y^+)^2} u
\end{equation}
%
The traction on the wall element is assumed to be parallel to the wall
and in opposite direction to the velocity vector, that is
%
\begin{equation} 
  t_p = -g(u) u_p
\end{equation}
%
Replacing in (\ref{eq:wallres}) the residual term is
%
\begin{equation} 
   R_{ip} = -\int_\Se  g(u) u_p N_i \dS
\end{equation}
%
The Jacobian of the residual with respect to the state variables,
needed for the Newton-Raphson algorithm is
%
\begin{equation}
\begin{aligned}
   J_{ip,jq} &= - \dep{R_{ip}}{u_{jq}} = 
                 \int_\Se  \dep{}{u_{jq}} (g(u) u_p) \,  N_i \dS \\
       &\int_\Se  \LL(g(u) \, \dep{u_p}{u_{jq}} + g'(u) \, u_p \,
                 \dep{u}{u_{jq}}\RR)  \,  N_i \dS  \label{eq:integ} 
\end{aligned}
\end{equation}
%
but
%
\begin{equation} 
   u_p = \sum_l u_{lp} N_l
\end{equation}
%
so that
%
\begin{equation}
\begin{aligned}
   \dep{u_p}{u_{jq}} &= \sum_l \dep{u_{lp}}{u_{jq}} N_l\\
                     &= \sum_l \delta_{lj} \, \delta_{pq} N_l\\
                     &= \delta_{pq} \, N_j
\end{aligned}
\end{equation}
%
Similarly,
%
\begin{equation} 
   u^2 = u_p\,u_p = u_{lp} N_l \, u_{mp} N_m
\end{equation}

%
and
%
\begin{equation} 
   2u \, \dep{u}{u_{jq}}  = 2 u_p \dep{u_p}{u_{jq}}
\end{equation}
%
so that
%
\begin{equation} 
   \dep{u}{u_{jq}}  = \frac{u_p}{u} N_j
\end{equation}
%
Replacing in (\ref{eq:integ}),
%
\begin{equation} 
   J_{ip,jq} = \int_\Se  \LL(g(u) \, \delta_{pq} + \frac{g'(u)}u \, u_p \,
                 u_q\RR)  \,  N_i N_j\dS
\end{equation}
  
\index{van Driest}
\SSSection{The van Driest damping factor. Programming notes}

\index{wall element}
\index{ANN}
\index{shear velocity}
\index{octree}
%
This is a non-standard issue, since the computation of one volume
element requires information of other (wall) elements.  First we
compute the wall element that is associated to each volume
element. \verb+assemble()+ is called with
\verb+jobinfo="build_nneighbor_tree"+. This jobinfo is acknowledged
only by the wall elemsets which compute their geometrical center and
put them in the \verb+data_pts+ STL array.  Then, this is passed to
the ANN package which computes the octree. All this is cached in the
constructor of a \verb+WallData+ class. After this a call to \verb+assemble()+
with \verb+jobinfo="get_nearest_wall_element"+ is acknowledged by all
the volume elemsets, that compute for each volume element the neares
wall element. This is stored as an \emph{``integer per element
property''} in the volume elemsets. In order to reduce memory
requirements only an index in the \verb+data_pts+ array is stored. As
several wall elemsets may exists, an array of
\verb+pair<int,elemset *>+ is used to store pointers to the
\verb+data_pts+ array in order to know to which wall elemset the given
index in \verb+data_pts+ belongs. 

\begin{verbatim}
vector<double> *data_pts_ = new vector<double>;
vector<ElemToPtr> *elemset_pointer = new vector<ElemToPtr>;
WallData *wall_data;
if (LES) {

  VOID_IT(argl);
  argl.arg_add(data_pts_,USER_DATA);
  argl.arg_add(elemset_pointer,USER_DATA);
  Elemset *elemset=NULL;
  argl.arg_add(elemset,USER_DATA);
  ierr = assemble(mesh,argl,dofmap,
                   "build_nneighbor_tree",&time); CHKERRA(ierr); 
  PetscPrintf(PETSC_COMM_WORLD,"After nearest neighbor tree.\n");

  wall_data = new WallData(data_pts_,elemset_pointer,ndim);

  //---:---<*>---:---<*>---:---<*>---:---<*>---:---<*>---:---<*>---: 
  // Find nearest neighbor for each volume element
  VOID_IT(argl);
  argl.arg_add(wall_data,USER_DATA);
  ierr = assemble(mesh,argl,dofmap,"get_nearest_wall_element",
      	    &time); CHKERRA(ierr); 
}
\end{verbatim}

In the \verb+jobinfo="build_nneighbor_tree"+ call to \verb+assemble()+
a loop over all the elements in the elemset, ignoring to what
processor it belongs, must be performed. Otherwise, each processor
loads in \verb+data_pts+ only the coordinates of the elements that
belong to him. A possible solution is, after the loop, to exchange the
information among the processors, but the simplest solution id to
simply bypass the element selection with \verb+compute_this_elem()+
with a call like
%
\begin{verbatim}
  for (int k=el_start; k<=el_last; k++) {
    if (!(build_nneighbor_tree ||
          comp_shear_vel ||
          compute_this_elem(k,this,myrank,iter_mode))) continue;
...
\end{verbatim}
%
That means that for \verb+jobinfo="build_nneighbor_tree"+ and
\verb+"comp_shear_vel"+ the normal element selection is bypassed. 

\SSection{Options}

\latex{\begin{flushleft}}
\begin{description}
%<*>---<*>---<*>---<*>---<*>---<*>---<*>---<*>---<*>---<*>---<*> 
\item[General options:] ~

\begin{itemize}
\input odocns
\end{itemize}
%<*>---<*>---<*>---<*>---<*>---<*>---<*>---<*>---<*>---<*>---<*> 
\item[Elemset ``{\tt nsitetlesfm2}'':] ~

\begin{itemize}
\input odocnse
\end{itemize}

%<*>---<*>---<*>---<*>---<*>---<*>---<*>---<*>---<*>---<*>---<*> 
\item[Elemset ``{\tt bcconv\_ns\_fm2}'':] ~

\begin{itemize}
\input odocnsb
\end{itemize}

%<*>---<*>---<*>---<*>---<*>---<*>---<*>---<*>---<*>---<*>---<*> 
\item[Elemset ``{\tt wall}'':] ~

\begin{itemize}
\input odocnsw
\end{itemize}

%<*>---<*>---<*>---<*>---<*>---<*>---<*>---<*>---<*>---<*>---<*> 
\item[Elemset ``{\tt ns\_sup\_g}'':] ~

This element imposes a linearized  free surface boundary condition. 

\begin{itemize}
\input odocnsfs
\end{itemize}

\end{description}
\latex{\end{flushleft}}

\SSection{Mesh movement}

Sometimes one has a mesh (connectivity and node coordinates) and wants
a mesh for a slightly modified domain. If $!u$ is the displacement of
the boundaries, then we can pose the problem of finding a mesh
topologically equal to the original one, but with a slight
displacement of the nodes so that the boundaries are in the new
position. This is termed \emph{``mesh relocation''}. One way to do this
is to solve an elasticity problem where the displacements at the
boundaries are the prescribed displacement of the boundary. 
This problem has been included by convenience in the Navier-Stokes
module, even if at this stage it has little to do with the
Navier-Stokes eqs.

Once the displacement is computed by a standard finite element
computation, we can compute the new mesh node coordinates by adding
simply the computed displacement to the original node coordinate. The
elastic constants can be chosen arbitrarily. If an isotropic material
is considered, then the unique relevant non-dimensional parameter is
the Poisson ratio, controlling the incompressibility of the fictituos
material. However, if the distortion is too large, this linear simple
strategy can break down, when some element collapses and has a negative
Jacobian. A simple idea to fix this is to somewhat rigidize the
elastic constants of the fictituos material in order to minimize the
distortion of the elements. Designing this nonlinear material
behaviour should guarantee a unique solution and a relatively easy way
to compute the Jacobian. A possibility is to have an hyperelastic
material, i.e. to define an \emph{``energy functional''} $F(\epsilon_{ij})$ as
function of the strain tensor. One should guarantee that this
functional is convex, and one should have an easy way to compute its
derivatives and second derivatives.

A related approach, implemented in PETSc-FEM, is to compute (for
simplices) the transformation from a regular master element to the
actual element, and to define the energy functional to be minimized as
a function of the associated matrix tensor. If $J_{ij} = \deps
{x_i}{\xi_j}$ where $x_i$ are the spatial coordinates and $\xi_j$ the
master coordinates, then the metric tensor is 
%
\begin{equation} \label{eq:gij}  
G_{ij} = \dep{x_k}{\xi_i}  \, \dep{x_k}{\xi_j}.
\end{equation}
%
The Jacobian of the transformation may be computed by differentiating
the interpolated displacements
%
\begin{equation} 
  x_i(\xi_j) = \sum_\mu x_{\mu i} N_\mu(\xi_j),
\end{equation}
%
with respect to the master coordinates, so that
%
\begin{equation} 
  J_{ij} = \dep {x_i}{\xi_j} = \sum_\mu x_{\mu i} \dep{N_\mu}{\xi_j}.
\end{equation}
%
We want to compute the distortion function to be minimized as a
function of the eigenvalues of the metric tensor $G_{ij}$. The
eigenvalues $!v_q$ of $!G$ are such that
%
\begin{equation} 
  G_{ij} v_{qj} = \lambda_q v_{qj} 
\end{equation}
%
so that,
%
\begin{equation} \label{eq:eigg}  
  \lambda_q  = v_{qi} \, G_{ij} \, v_{qj} 
\end{equation}
%
(no sum over $q$) since the $!v_q$ are orthogonal. Now if the
functional $F$ is a function of the element node coordinates through
the eigenvalues $\lambda_q$ then we can compute the new coordinates
$x'$ as
%
\begin{equation} \label{eq:nonli}  
  \dep F {x_{\mu j}} = 0.
\end{equation}
%
A possible distortion functional is
%
\begin{equation} 
   F(\{\lambda_q\}) = \LL(\Pi_q \lambda_q \RR)^{-2/\ndim} \, 
      \sum_{qr} (\lambda_q-\lambda_r)^2
\end{equation}
%
This functional has several nice features. Is minimal whenever all the
eigenvalues are equal (the distortion is minimal). It is
non-dimensional, so that an isotropic dilatation or contraction
doesn't produce a change in the functional.  The non-linear problem
(\ref{eq:nonli}) can be solved by a Newton strategy, by computing the
first and second derivatives of $F$ with respect to the node
displacements (the residual and Jacobian of the system of equations)
by finite difference approximations. However, this turns to be too
costly in CPU time for the second derivatives, since we should compute
for each second derivative three evaluations of the functional, and
there are $\nen(\nen+1)/2$ (where $\nen=\ndof\nel$ is the number of
unknowns per element, $\nel$ is the number of nodes per element, and
$\ndof$ the number of unknowns per node, here $\ndof=\ndim$) second
derivatives to compute. Each evaluation of the functional amounts to
the computation of the tensor metrics $G$ and to solve the associated
eigenvaue problem, so that an analytical expression to, at least, the
first derivatives, is desired. The derivatives of the distortion
functional can be computed as
%
\begin{equation} 
  \dep F {x_j} = \dep F {\lambda_q} \dep{\lambda_q}{x_{\mu j}}
\end{equation}
%
and then the derivatives with respect to the eigenvalues can be
computed still numerically, while we will show that the derivatives of
the eigenvalues with respect to the node coordinates (which are the
most expensive part) can be computed analytically. In this way, we can
compute the derivatives of the functional with the solution of only
one eigenvalue problem. The second derivatives can be computed
similarly as
%
\begin{equation} 
  \dep {^2F} {x_{\mu i} x_{\nu j}} = 
  \dep {^2F}{\lambda_q\lambda_r} \, \dep{\lambda_q}{x_{\mu i}} \,
  \dep{\lambda_r}{x_{\mu j}} + 
     \dep F {\lambda_q} \, \dep{^2\lambda_q}{x_{\mu i} x_{\nu j}}. 
\end{equation}
%
The first and second derivatives of $F$ with respect to the
eigenvalues are still computed numerically, whilst the second 
derivatives of the eigenvalues can be computed by differentiating
numerically the first derivatives. This amounts to $O(\nen)$ eigenvalue
problems for computing the first and second derivatives of the
distortion functional (the residual and Jacobian). This cost may be
further reduced by noting that the eigenvalues are invariant under
rotations and translations, and simply scaled by a dilatation or
contraction, so that, from the $\nen=\ndim(\ndim+1)$ displacements,
only $\ndim(\ndim+1)/2-1$ should be really computed, but this is not
implemented yet. For tetras in 3D this implies a reduction from 12
distortion functional computation to only 5. 

We will show now how the derivatives of the eigenvalues are computed
analytically. It can be shown that 
%
\begin{equation} 
  \dep{\lambda_q}{x_{\mu k}}  = v_{qi} \, \dep{G_{ij}}{x_{\mu k}} \, v_{qj}.
\end{equation}
%
Note that this is as differentiating (\ref{eq:eigg}) but only keeping
in the right hand side 
the change in the matrix $G$ and discarding the rate of change of the
eigenvectors. It can be shown that the conttribution of the other two
terms is an antisymmetric matrix, so that the contibution to the rate
of change of the eigenvalue is null. 
Now, from (\ref{eq:gij})
%
\begin{equation} 
\dep{G_{ij}}{x_{\mu l}} = J_{li} \, \dep {N_\mu}{\xi_j} + 
        J_{lj} \, \dep {N_\mu}{\xi_i}
\end{equation}
%
so that
%
\begin{equation} 
  \dep{\lambda_q}{x_{\mu l}}  = \LL(v_{ql} \, J_{li}\RR) \, 
                \LL(\dep{N_\mu}{\xi_j} \, v_{qj}\RR),
\end{equation}
%
which is the expression used. 

% Local Variables: *
% mode: latex *
% tex-main-file: "petscfem.tex" *
% End: *

