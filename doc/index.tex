%__INSERT_LICENSE__
\documentclass[11pt]{article}
\usepackage{makeidx,epsf}
\usepackage[spanish]{babel}
\usepackage{html,htmllist,heqn}

\makeindex
\sloppy

\input version.tex
\begin{document}
\latex{\hsize 15truecm
\vsize 24.7truecm
\advance \hoffset by -1truecm
\advance \voffset by -3truecm}
\sloppy

\title{PETSc-FEM: A General Purpose, Parallel, Multi-Physics FEM Program.}
\maketitle

\begin{abstract}
%fixme:= incluir referencia a PETSc
This is the documentation root page for \pfem (current version 
{\tt petscfem-\petscfemversion}), a general purpose, parallel,
multi-physics FEM program for CFD applications based on
PETSc. \pfem{} comprises both a library that allows the user to
develop FEM (or FEM-like, i.e. non-structured mesh oriented) programs, and
a suite of application programs. It is written in the C++ language
with an OOP (Object Oriented Programming) philosophy, but always
keeping in mind the scope of efficiency. 
\end{abstract}

The documentation for PETSc-FEM is divided in 

\begin{itemize}
\item A \htmladdnormallink{user's guide}{petscfem/petscfem.html} (written
in \LaTeX and also available in PostScript and HTML).

\item The \htmladdnormallink{reference
documentation}{manual/html/index.html} for all the routines in the
library and the applications, stripped from the sources with Doc++.
\end{itemize}

\end{document}
