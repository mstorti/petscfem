%__INSERT_LICENSE__
\Section{Generic load elemsets}

Generic load elemsets account for surface contributions which
represent constant terms in the governing equations or either a
function of the state at the surface. Typical terms that can be
represented in this way are
%
\begin{itemize}
\compactlist 
\item External heat loads (like a constant radiation load) in thermal
  problems: $q=\bar q$
\item A linear Newtonian cooling term $q= -h(T-\Tinfty)$. 
\item A nonlinear Newtonian cooling term $q= f(T,\Tinfty)$. 
\end{itemize}
 
\SSection{Linear generic load elemset}

%<*>---<*>---<*>---<*>---<*>---<*>---<*>---<*>---<*>---<*>---<*> 
\begin{figure*}[htb]
\centerline{\includegraphics[scale=0.8]{./OBJ/genloadsl}}
\caption{Generic load element (single layer)}
\label{fg:genloadsl}
\end{figure*}
%<*>---<*>---<*>---<*>---<*>---<*>---<*>---<*>---<*>---<*>---<*> 

%<*>---<*>---<*>---<*>---<*>---<*>---<*>---<*>---<*>---<*>---<*> 
\begin{figure*}[htb]
\centerline{\includegraphics[scale=0.8]{./OBJ/genloaddl}}
\caption{Generic load element (double layer)}
\label{fg:genloaddl}
\end{figure*}
%<*>---<*>---<*>---<*>---<*>---<*>---<*>---<*>---<*>---<*>---<*> 

In the simplest case the load is of the form
%
\begin{equation} 
  \begin{aligned} 
    !q &= -!h !U + \bar{!q},&&\text{(single layer)}\\
    !q &= !h (!U_\mathrm{out}-!U) + \bar{!q},&&\text{(double layer)}\\
  \end{aligned}
\end{equation}
% 
This is implemented by the \verb+lin_gen_load+ elemset. This elemset
may be \emph{``single-layer''} or \emph{``double-layer''} (see
figures~\ref{fg:genloadsl} and \ref{fg:genloaddl}). Double layer
elements can represent a lumped thermal resistance, for instance a
very thin layer of air inside a material of higher conductivity. In
the double layer case the number of nodes is twice than in the
single-layer case. Double layer is activated either by including the
\verb+double_layer+ option or either if the number of nodes is twice
that one specified by the geometry.

The options for the \verb+lin_gen_load+ elemset are
%
\begin{itemize}
\input odocgenl
\end{itemize}

\SSection{Functional extensions of the elemset}

The generic elemset is \verb+GenLoad+. There is an instantiation for
the Generic load elemsets can be extended by the user by deriving the
base class \verb+GenLoad+ and the most important task is to implement
the methods \verb+q(u,flux,jac)+ for the single layer case, and
\verb+q(u_in.u_out,flux_in,flux_out,jac)+ in the double layer case.

\SSection{The flow reversal elemset}

This instantiation of \verb+GenLoad+ is useful for avoiding
instabilities caused by inversion of the flow at an outlet
boundary. Assume that the Navier-Stokes module computes a velocity
field $!v$ and some scalar $\phi$ is being advected with this velocity
field. On an outlet boundary one usually leaves the temperature free,
that is, no Dirichlet condition is imposed on $\phi$. But if the flow
is reversed, that is $!v\cdot\nor<0$ ($\nor$ is the external normal to
the boundary) at some instant, then this boundary condition becomes
ill-posed and the simulation may diverge. One solution is to switch
from Neumann to Dirichlet boundary conditions depending on the sign of
$!v\cdot\nor$, i.e.
%
\begin{equation} 
  !q = \begin{cases}
    0,& \textrm{ if $!v\cdot\nor\ge0$},\\
    h(\bar \phi-\phi),& \textrm{ if $!v\cdot\nor>0$},\\
  \end{cases}
\end{equation}
%
where $h$ is a film coefficient and $\bar phi$ is the value to be
imposed at the boundary when the flow is reversed. Note that this
amounts to switch to a Dirichlet condition by \emph{penalization}. For
$h$ large enough, the value of $\phi$ will converge to
$\bar\phi$. However, if $h$ is too large, this can affect the
conditioning of the linear system. 

The options for this elemset are
%
\begin{itemize}
\input odocflrv
\end{itemize}

\SSection{Examples of use of flow reversal elemset}

Typical use in a thermal convection problem associated with an element
like \verb+nsi_les_ther+ is as follows
%
\medskip
\begin{Verbatim}
elemset flow_reversal 2
npg 2
geometry cartesian1d
dofs 4
coefs <PENALIZATION-COEFFICIENT>
refvals <INLET-TEMPERATURE>
data <SURFACE-CONNECTIVITY-FILE>
__END_HASH__
\end{Verbatim}
\medskip
%
Note that \verb+dof=4+ corresponds to the temperature field.
\verb+<INLET-TEMPERATURE>+ is the temperature that the user wants to be
imposed at the boundary if the flow is reverted.
\verb+<PENALIZATION-COEFFICIENT>+ should be large enough so that the
temperature approaches \verb+<INLET-TEMPERATURE>+. However, a large
value has the disadvantage of increasing the condition number of the
linear system. 

In 3D with a surface of triangular panels the input file section would
be
%
\medskip
\begin{Verbatim}
elemset flow_reversal 3
npg 4
geometry triangles
dofs 5
coefs <PENALIZATION-COEFFICIENT>
refvals <INLET-TEMPERATURE>
data <SURFACE-CONNECTIVITY-FILE>
__END_HASH__
\end{Verbatim}
\medskip

This term can be used also with the dofs of the velocity itself, so
that a lumped Darcy term that tends to prevent the reversal of the
flow is added. In this case the input file section should be
%
\medskip
\begin{Verbatim}
elemset flow_reversal 3
npg 4
geometry triangles
# Penalize temperature AND velocity (lumped Darcy term)
dofs 1 2 3 5
coefs COEFV COEFV COEFV COEFT
refvals 0. 0. 0. <TEMPERATURA-DE-ENTRADA>
data <ARCHIVO-DE-CONECTIVIDADES-DE-SUPERFICIE>
__END_HASH__
\end{Verbatim}
\medskip
where \verb+COEFV+ is a penalization coefficient for velocity and
\verb+COEFT+ is a penalization coefficient for temperature. The
reference value for velocity are null, i.e. if the flow is reverted
then the penalization term tends make it as small as possible. 

\iffalse
El ejemplo es un rectangulo de Lx=2 por Ly=1. Es un flujo Poiseuille
con condiciones periodicas y con un G_body de [1,0] o
[-1,0]. Dependiendo del signo el flujo va para un lado o para el
otro. En los contornos left y right (x=0 o x=2) se pone la condicion
de `flow_reversal' con un refval de 1.0 de manera que cuando G_body =
[1,0] el flujo es entrante en x=0 y por lo tanto ahi el flow_reversal
impone T=1. Cuando se pone G_body = [-1,0] al reves, y el
flow_reversal impone T=1 en x=2. Ojo el coeficiente de penalizacion
esta bajo de manera que se ven temperaturas en la pared que son del
orden de 0.9. No lo probe pero aumentando el coeficiente de
penalizacion la velocidad deberia acercarse mas a uno. 

Repito, en una de esas esto tambien actua de Darcy, es lo siguiente
que voy a probar. 
\fi
