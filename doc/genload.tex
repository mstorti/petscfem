%__INSERT_LICENSE__
\Section{Generic load elemsets}

Generic load elemsets account for surface contributions which
represent constant terms in the governing equations or either a
function of the state at the surface. Typical terms that can be
represented in this way are
%
\begin{itemize}
\compactlist 
\item External heat loads (like a constant radiation load) in thermal
  problems: $q=\bar q$
\item A linear Newtonian cooling term $q= -h(T-\Tinfty)$. 
\item A nonlinear Newtonian cooling term $q= f(T,\Tinfty)$. 
\end{itemize}
 
\SSection{Linear generic load elemset}

In the simplest case the load is of the form
%
\begin{equation} 
  !q = !h !U + \bar{!q}
\end{equation}
% 


This elemset may be \emph{``single-layer''} or
\emph{``double-layer''} (see figures~\ref{fg:genloadsl} and
\ref{fg:genloaddl}). Double layer elements can represent a
lumped thermal resistance, for instance a very thin layer of air
inside a material of higher conductivity. 


The generic elemset is \verb+GenLoad+. There is an instantiation for
the 
Generic load elemsets can be extended by the user by deriving the base
class \verb+GenLoad+ and implement the methos \verb+q(u,flux,jac)+ for
the single layer case, and \verb+q(u_in.u_out,flux_in,flux_out,jac)+
in the double layer case. 

%<*>---<*>---<*>---<*>---<*>---<*>---<*>---<*>---<*>---<*>---<*> 
\begin{figure*}[htb]
\centerline{\includegraphics[scale=0.8]{./OBJ/genloadsl}}
\caption{Generic load element (single layer)}
\label{fg:genloadsl}
\end{figure*}
%<*>---<*>---<*>---<*>---<*>---<*>---<*>---<*>---<*>---<*>---<*> 

%<*>---<*>---<*>---<*>---<*>---<*>---<*>---<*>---<*>---<*>---<*> 
\begin{figure*}[htb]
\centerline{\includegraphics[scale=0.8]{./OBJ/genloaddl}}
\caption{Generic load element (double layer)}
\label{fg:genloaddl}
\end{figure*}
%<*>---<*>---<*>---<*>---<*>---<*>---<*>---<*>---<*>---<*>---<*> 

% Local Variables: *
% mode: latex *
% tex-main-file: "petscfem.tex" *
% End: *

