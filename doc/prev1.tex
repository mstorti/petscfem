%__INSERT_LICENSE__
\Section{The \pfem{} philosophy} 

\SSection{The three levels of interaction with \pfem} 

\index{PETSc-FEM!programmers}
\index{PETSc-FEM!application writers}
\index{PETSc-FEM!users}
%
As stated in the \pfem{} description, it is both a library and an
application suite. That means that some applications as Navier-Stokes,
Euler (inviscid flow), shallow water, general advective linear systems
and the Laplace/Poisson equation, come bundled with it, whereas the
library is an abstract interface that allows people to write other
applications. So that we distinguish between the \emph{``user''} for
which the interaction with \pfem{} is limited to writing data files
for the bundled applications, from the \emph{``application writers''}
that is people that uses the library to develop new
applications. Usually, application writers write a \verb+main()+
routine that use routine calls to the \pfem{} library in order to
assemble PETSc vectors and matrices and perform algebraic operations
among them via calls to PETSc routines.  In addition, they also have
to code \emph{``element routines''} that compute vectors and matrices
at the element level. \pfem{} is the code layer that is in charge of
assembling the individual contributions in the global vectors or
matrices, taking into account fixations, etc...  Finally, there is the
\emph{``\pfem{} programmers''}, that is people that write code for the
core library.

\begin{figure*}[htb]
\centerline{\includegraphics{./OBJ/petscfem.\EPSPDF}}
\caption{Typical structure of a \pfem{} application}
\label{fg:petscfem}
\end{figure*}

\SSection{The elemset concept}

\index{elemset}
\pfem{} is written in the C++ language and sticks to the
Object-Oriented Programming (OOP) philosophy. In OOP, data is stored
in \emph{``objects''} and the programmer access them via an abstract
interface. A first approach to writing a Finite Element program with
OOP philosophy, is to define each element or node as an
object. However, this is both time and memory consuming, since
accessing each element or each node is performed by passing through
the whole code layer of the element or node class. As one of the
primary objectives of \pfem{} is the efficiency, we solved this by
defining the basic objects as a whole set of elements of nodes that
share almost all the properties, aside element connectivities or node
coordinates. This is very common in CFD, where for each problem almost
all the elements share the same physical properties (viscosity,
density, etc...) and options (number of integration Gauss points,
parameters for the FEM formulation, etc...). Thus, for each problem
the user defines a \verb+nodedata+ object, and one or several
\verb+elemset+ objects. 

Each elemset groups elements of the same type, i.e. for which
residuals and matrices are to be computed by the same routine.
Usually, in CFD all the elements are processed by the same routine, so
that one may ask for what it may serve to have several elemsets.
First, some boundary conditions (constant flux or mixed boundary
conditions for heat conduction, absorbing boundary conditions for
advective systems) may be imposed more conveniently through an elemset
abstraction.  Also, several elemsets may be used for reducing the
space required for storing varying physical properties.  If some
physical property is shared by all the elements, for instance
viscosity or specific heat, then it is defined once for all the
elemset. If the quantity varies continuously in the region, but it is
known a priori, then it can be passed as a ``per-element'' property
(see~\S\ref{sec:per_elem_prop}), but that means storage of a double
(or the amount of memory needed for that property) for each
element. If it is not the same on all the mesh, but is constant over
large regions, then it may be convenient to divide the mesh in several
elemsets, where the given property has the same value over the
elements of each elemset.


% Local Variables: *
% mode: latex *
% tex-main-file: "petscfem.tex" *
% End: *
