%__INSERT_LICENSE__

\Section{The ``idmap'' class}

\SSection{Permutation matrices} 

\index{idmap}
\index{permutation matrices}
%
This class represents sparse matrices that are ``close'' to a
permutation. Assume that the $n\times n$ matrix \lfrc{$Q$} is a permutation
matrix, so that if \lfrc{$!y$}, \lfrc{$!x$} are vectors of dimension \lfrc{$n$} and
%
\begin{equation} 
   !y = !Q \, !x
\end{equation}
%
then
%
\begin{equation} 
  y_j = x_{P(j)}
\end{equation}
%
where \lfrc{$P$} is the permutation associated with \lfrc{$Q$}. For instance if
$P=\{1,2,3,4\}\to\{2,4,3,1\}$ then
%
\begin{equation} 
  !y = !Q \, !x = \begin{sqmat}{c} 
                        x_2\\ 
                        x_4\\
                        x_3\\
                        x_1
                  \end{sqmat}
\end{equation}
%
This matrix can be efficiently stored as an array of integers of
dimension ${\rm ident[n]}$ such that ${\rm ident[j-1]} = P(j)$. Also,
the inverse permutation can be stored in another integer array ${\rm
iident[n]}$, so that both \lfrc{$Q$} and $Q\muno$ can be stored at the cost
of  \lfrc{$n$} integer values for each. Also, both multiplication and
inversion operations as $!y = !Q\,!x$, $!y = !Q^T\,!x$, $!x =
!Q\muno\,!y$ and $!x = !Q^{-T}\,!y$, can be done with $n$-addressing
operations. 

\SSection{Permutation matrices in the FEM context}

\begin{figure*}[htb]
\centerline{\includegraphics{./OBJ/idmap.\EPSPDF}}
\caption{Node/field to dof map. Example for NS or Euler in a duct.}
\label{fg:idmap}
\end{figure*}
 
\index{degree of freedom}
\index{dof}
Permutation matrices are very common in FEM applications for
describing the relation between node/field pairs and degrees of
freedom (abbrev. \emph{``dof''}'s). For instance consider a flow NS or
Euler application in a mesh like that shown in
figure~\ref{fg:idmap}. At each node we have a set of unknown
``fields'' (both components of velocity and pressure, \lfrc{$u$},
\lfrc{$v$} and \lfrc{$p$}).  In a first description, we may arrange
the vector of unknowns as
%
\begin{equation} 
   !U_\NF = \begin{sqmat}{c}
            u_1\\
            v_1\\
            p_1\\
            u_2\\
            v_2\\
            p_2\\
            \vdots\\
            u_\Nnod\\
            v_\Nnod\\
            p_\Nnod
        \end{sqmat}
\end{equation}
%
The length of this vector is $\Nnod\times\ndof$ and this may be called
the ``node/field'' (this accounts for the \lfrc{$NF$} subindex) description
of the vector of unknowns.  However, we can not take this vector as
the vector of unknowns for actual computations due to a series of
facts,

\begin{itemize}

\item Not all of them are true unknowns, since some of them may be
imposed to a given value by a Dirichlet boundary condition.

\item There may be some constraints between them, for instance, in structural
mechanics two material points linked by a rigid bar, or a node that is
constrained to move on a surface or line. In CFD these constraints
arise when periodic boundary conditions are considered 

\item Some reordering may be necessary, either for reducing band-width
if a direct solver is used, or either due to mesh partitioning, if the
problem is solved in parallel. 

\item Also, non-linear constraints may arise, for instance when
considering absorbing boundary conditions or non-linear restrictions. 

\end{itemize}

So that we assume that we have a relation
%
\begin{equation} 
  !U_\NF = !Q \, !U + \barQ \, \barU   \label{eq:urep} 
\end{equation}
%
where \lfrc{$!U$} is the ``reduced'' array of unknowns, $\barU$ 
representing the externally fixed values, and  \lfrc{$Q$}, $Q^*$ appropriated
matrices. Follows some common cases 

\begin{description}
\item[Dirichlet boundary conditions:] If the velocity  components of node 1 are
fixed, then we have
%
\begin{equation} 
\begin{aligned}
   u_1 = \bar u_1\\
   v_1 = \bar v_1
\end{aligned}
\end{equation}
%
so that the corresponding file in \lfrc{$!Q$} is null and the file in $\barQ$
has a ``1'' in the entry corresponding to the barred values. 

\begin{figure*}[htb]
\centerline{\includegraphics{./OBJ/slip.\EPSPDF}}
\caption{Slip boundary condition}
\label{fg:slip}
\end{figure*}

\item[Slip boundary conditions:] For the Euler equations, if node $j$
is on a slip wall and $\nor_j$ is the corresponding normal, then the
normal component of velocity is null and the tangential component is
free. (In 3D there are two free tangential components and one
prescribed normal component). The normal component may be prescribed
to some non-null value if transpirations fluxes are to be imposed. The
corresponding equations may look like this
%
\begin{equation} 
\begin{aligned}
   u_j &= u_{jt} t_x + \bar u_{jn} n_x\\
   v_j &= u_{jt} t_y + \bar u_{jn} n_y
\end{aligned}
\end{equation}
%
where $u_{jt}$ is the (free) tangential component and $\bar u_{jn}$
the prescribed normal component. 

\begin{figure*}[htb]
\centerline{\includegraphics{./OBJ/symme.\EPSPDF}}
\caption{Symmetrical arrangement of foils with specular symmetry.}
\label{fg:symme}
\end{figure*}
 
\begin{figure*}[htb]
\centerline{\includegraphics{./OBJ/nonsym.\EPSPDF}}
\caption{Non-symmetrical arrangement of foils.}
\label{fg:nonsym}
\end{figure*}
 
\index{periodic boundary conditions}
\index{boundary conditions!periodic}
%
\item[Periodic boundary conditions:] These kind of b.c.'s are useful
when treating geometries with repetitive components, as is very common
for rotating machinery. In some cases this can be done as slip
boundary conditions (see figure~\ref{fg:symme}). Here the foils are
symmetric about their centerline, and then the whole geometry not only
posses symmetry of rotation about angles multiple of $2\pi/\nfoils$
but in addition possess reflection symmetry about the centerline of a
foil, like $BC$ and the centerline between two consecutive foils like
$AB$. In consequence it suffices the domain $ADEC$ with inlet/outlet
b.c.'s in $DE$ and $AC$, slip boundary conditions on those parts of
$AD$ and $EFGHC$. On the foil (boundary $FGH$) pressure should be
free, while on $AD$, $EF$ and $HC$ we can impose $\deps pn=0$. For
Navier-Stokes we may impose solid wall b.c.'s on the foil, normal null
velocity component on $AD$, $EF$ and $HC$, null tangential shear
stress ($\deps {u_t}n=0$), and $\deps pn=0$. 

However, if the foils are non-symmetric or they are not disposed
symmetrically about a line passing by the rotation axis (see
figure~\ref{fg:nonsym}) then there are no predefined streamlines, but
given two corresponding point like $j$ and $j'$ that are obtained
through rotation of $\alpha=2\pi/\nfoils$, then we can impose
%
\begin{equation} 
\begin{aligned}
        u_{j'} &= \cos\alpha \, u_j + \sin\alpha \, v_j\\
        v_{j'} &=-\sin\alpha \, u_j + \cos\alpha \, v_j\\
        p_{j'} &= p_j\\
\end{aligned}
\end{equation}

\index{absorbing boundary conditions}
\index{boundary conditions!absorbing}
%
\item[Absorbing boundary conditions] The basic concept about these
b.c.'s is to impose the in-going components from outside (reference)
values while leaving the outgoing components free. If \lfrc{$!w=[u,v,p]$} is
the state vector of the fluid at some node \lfrc{$j$} on the outlet boundary
(see figure~\ref{fg:idmap}), then \lfrc{$!V!w = !u$} are the
eigen-components, where \lfrc{$!V$} is the change of basis matrix.
The absorbing b.c. may be written as
%
\begin{equation} 
  !u = !V\begin{sqmat}{c} \bar w^1\\ w^2 \\ w^3 \end{sqmat}
          \label{eq:abso}  
\end{equation}
%
% where $\bPi^+$ is a \emph{projection} matrix (i.e. $\bPi^+\bPi^+
% =\bPi^+$) onto the ingoing components and $!w_\refer$ represents the
% outer fluid state. $\bPi^+$ is computed through an eigenvalue
% decomposition of the normal projection of the flux jacobians. $\bPi^+$
% has a rank equal to the number $n^+$ of ingoing components, so that
% only $n^+$ of the $\ndof$ equations in (\ref{eq:abso}) are linearly
% independent. 
Where $\bar w^1$ is the in-going component (taken from the reference
value $!w_\refer$) and the other components $w^{2,3}$ are free,
i.e. they go in \lfrc{$!U$}. 

\index{boundary conditions!non-linear Dirichlet}
\index{Non-linear Dirichlet boundary conditions}
%
\item[Non-linear Dirichlet boundary conditions:] In some cases,
Dirichlet boundary conditions are not expressed in terms of the state
variables used in the computations, but on a non-linear combination of
them, instead. For instance, consider the transport of moisture and
heat through a porous media, and choose temperature $T$ and moisture
content $H$ as the state variables. On an external boundary, we impose
that the partial pressure of water should be equal to its external
value $P_w = P_{w,\atm}$. As the partial pressure of water (which may
be related to relative humidity) is a complex non-linear function of
$T$ and $H$ through the sorption isotherms of the porous media and the
saturation pressure of water, it results in a non-linear link of the
form $P_w(T,H) = P_{w,\atm}$. By the moment we consider only a linear
relation, since the non-linear case doesn't fit in the representation
% fixme:= 
(\ref{eq:urep}). The non linear case will be considered later on.

\end{description}

\SSection{A small example} 

\begin{figure*}[htb]
\centerline{\includegraphics{./OBJ/mesh.\EPSPDF}}
\caption{A small example showing boundary conditions.}
\label{fg:mesh}
\end{figure*}
 
Consider the region shown in figure~\ref{fg:mesh}, composed of 8
triangle elements and 9 nodes. We have $9\times 3=27$ node/field
values, but periodic boundary conditions on side 1-4-7 to side 3-6-9
eliminates the 9 unknowns on these last three nodes. In addition, at
the outlet boundary (nodes 1-2-3) there is only one in-going component
so that the unknowns here are only $w^2_1$, $w^3_1$, $w^2_2$ and
$w^3_2$. On the other hand, on the inlet boundary $u$, and $v$ are
imposed so the vector of unknowns is
%
\begin{equation} 
\begin{aligned}
     U_1  &=   w^2_1\\
     U_2  &=   w^3_1\\
     U_3  &=   w^2_2\\
     U_4  &=   w^3_2\\
     U_5  &=   u_4\\
     U_6  &=   v_4\\
     U_7  &=   p_4\\
     U_8  &=   u_5\\
     U_9  &=   v_5\\
     U_{10} &=   p_5\\
     U_{11} &=   p_7\\
     U_{12} &=   p_8
\end{aligned}
\end{equation}
%
while the prescribed values are
%
\begin{equation} 
\begin{aligned}
  \bU_1  &=  \bar w^1_1\\
  \bU_2  &=  \bar w^1_2\\
  \bU_3  &=   u_7\\
  \bU_4  &=   v_7\\
  \bU_5  &=   u_8\\
  \bU_6  &=   v_8
\end{aligned}
\end{equation}
%
And the relation defining matrices \lfrc{$!Q$} and $\barQ$ are
%
\begin{equation} 
\begin{aligned}
   u_1 &= V^1_{11} \, \bU_1 + V^1_{12} \, U_1 + V^1_{13} \, U_2\\
   v_1 &= V^1_{21} \, \bU_1 + V^1_{22} \, U_1 + V^1_{23} \, U_2\\
   p_1 &= V^1_{21} \, \bU_1 + V^1_{22} \, U_1 + V^1_{23} \, U_2\\
%
   u_2 &= V^2_{11} \, \bU_2 + V^2_{12} \, U_3 + V^2_{13} \, U_4\\
   v_2 &= V^2_{21} \, \bU_2 + V^2_{22} \, U_3 + V^2_{23} \, U_4\\
   p_2 &= V^2_{21} \, \bU_2 + V^2_{22} \, U_3 + V^2_{23} \, U_4\\
%
   u_3 &= \cos\alpha \, (V^1_{11} \, \bU_1 + V^1_{12} \, U_1 + V^1_{13} \,
             U_2) + \sin\alpha \, (V^1_{21} \, \bU_1 + V^1_{22} \, U_1 + V^1_{23} \, U_2) \\
   v_3 &=-\sin\alpha \, (V^1_{11} \, \bU_1 + V^1_{12} \, U_1 + V^1_{13} \,
             U_2) + \cos\alpha \, (V^1_{21} \, \bU_1 + V^1_{22} \, U_1 + V^1_{23} \, U_2) \\
   p_3 &= V^1_{21} \, \bU_1 + V^1_{22} \, U_1 + V^1_{23} \, U_2\\
%
   u_4 &= U_5\\
   v_4 &= U_6\\
   p_4 &= U_7\\
%
   u_5 &= U_8\\
   v_5 &= U_9\\
   p_5 &= U_10\\
%
   u_6 &= \cos\alpha \, U_5 + \sin\alpha \, U_6\\
   v_6 &=-\sin\alpha \, U_5 + \cos\alpha \, U_6\\
   p_6 &= U_7\\
%
   u_7 &= \bU_3\\
   v_7 &= \bU_4\\
   p_7 &= U_{11} \\
%
   u_8 &= \bU_5\\
   v_8 &= \bU_6\\
   p_8 &= U_{12}\\
%
   u_9 &= \bU_3\\
   v_9 &= \bU_4\\
   p_9 &= U_{11} \\
%
\end{aligned}
\end{equation}
%
As we can see the boundary conditions result in sparse \lfrc{$!Q$} and
$\barQ$ matrices. Moreover, in real (large) problems most of the rows
correspond to interior nodes (such as node 5 here) so that \lfrc{$!Q$} is
very close to a permutation matrix. If we think at accessing the
elements of \lfrc{$!Q$} by row, then could store \lfrc{$!Q$} as a sparse matrix, but
in this case we need an average of two integers (for pointers) and a
double (for the value) per unknown, whereas a permutation matrix can
be stored with only one integer per unknown.  One possibility is to
think at these matrices as permutations followed by a sequence of
operations depending on each kind of boundary conditions, but it may
happen also that several kind of b.c.'s superposes, as int the case of
node 3 (periodic + absorbing) and node 9 (periodic+ Dirichlet).

\SSection{Inversion of the map} 

It is necessary sometimes to invert relation (\ref{eq:urep}),
i.e. given $!U_\NF$ and $\barU$ to find \lfrc{$!U$}, for instance when 
initializing a temporal problem. Of course, this may not be possible
for arbitrary  $!U_\NF$ and $\barU$, since \lfrc{$!Q$} is in general
rectangular, but we assume that $!Q^T!Q$ is non singular and solve
(\ref{eq:urep}) in a least square sense. After, we may verify if the given
data ($!U_\NF$ and $\barU$) is \emph{``compatible''} by evaluating the
residual of the equation. This operation should be performed in $O(N)$
operations. 

\SSection{Design and efficiency restrictions} 

The class of matrices representing \lfrc{$!Q$} and $\barQ$ should have the
following characteristics. 
%
\begin{itemize}
\item Be capable of storing arbitrary matrices.

\item Should be efficient for permutation like matrices, i.e. require
as storage (order of) 2 integers per unknown and constant time access
by row and column.
\end{itemize}

\SSection{Implementation} 

An arbitrary  permutation $P$ of $N$ objects can be stored as two integer
vectors ${\tt ident}[N]$ and ${\tt iident}[N]$ such that if $P(j) = k$, then
%
\begin{equation}
\begin{aligned}
   {\tt ident}[j-1] = k\\
   {\tt iident}[k-1] = j
\end{aligned}
\end{equation}
%
We will consider a slight modification to this. A row is 
%
\begin{description}
\item[void:] If all its elements are null
\item[normal:] If one element is one and the rest null (like in a
permutation matrix). 
\item[special: ] otherwise.
\end{description}
%
Then we set
%
\begin{equation} 
  {\tt ident}[j-1] = \begin{cases}
      0;& \textrm{ if row $j$ is void}\\
      k;& \textrm{ if row $j$ is normal and the non-null element is in
                  position $k$}\\
      -1;& \textrm{ if row $j$ is special}
  \end{cases}
\end{equation}
%
We need now how to access the coefficients of the special rows. 
Rows are stored as a ``map'', in the STL language, from integer values to doubles, i.e.
%
\begin{verbatim}
typedef map<int,double> row_t;
\end{verbatim}
%
and we keep in class {\tt idmap} a map from row indices to pointers to
rows of the form
%
\begin{verbatim}
map<int,row_t*> row_map;
\end{verbatim}
%
So that, for a given $j$ we can get the corresponding row by checking
the value of ${\tt inode}[j-1]$. If the row is void or normal we
return the corresponding row and if it is special, then we look for
the pointer in \verb+row_map+ and this returns the desired row. Similar
information is stored by columns, but in this case it is not necessary
to store the values for the special rows so that the columns are of the form
%
\begin{verbatim}
typedef set<int> col_t;
\end{verbatim}
%
and class {\tt idmap} contains a private member {\tt col\_map} of the form:
%
\begin{verbatim}
map<int,col_t*> col_map;
\end{verbatim}

\SSection{Block matrices} 

When solving a linear (\ref{eq:urep}) for \lfrc{$!U$}, it is a key point to
take into account that most part of \lfrc{$!Q$} are independent blocks, so
that the inversion may be done with a minimum computational effort. We
say that a given row and column indices $i,j$ we say that they are
\emph{``directly connected''} if the element $Q_{ij}$ is non null. We
can then define that two row indices $i,i'$ are \emph{``indirectly
connected''} if they are connected by a path $i\to j_1\to i_1\to j_2
\to \ldots \to j_n\to i'$ where the arrow means ``directly
connected''. The definition applies also to pairs of column indices
and row/column, column/row  pairs. 
The indirect connection defines an equivalence class that
splits the rows in disjoint sets.

\SSSection{Example:} 

Consider matrix
%
\begin{equation} 
  !Q = 
\begin{sqmat}{rcccccccccc} 
\textrm{row}\to  &1  & 2 & 3 & 4 & 5 & 6 & 7 & 8 & 9 & 10\\
\textrm{col}\downarrow
          1      & . & . & . & . & . & . & . & * & * & . \\   
          2      & . & . & . & . & . & . & . & . & . & .\\    
          3      & . & . & . & . & . & . & . & * & * & .\\    
          4      & . & . & . & . & . & . & . & . & . & *\\    
          5      & . & . & . & * & * & . & * & . & . & .\\    
          6      & . & . & . & * & * & . & * & . & . & .\\    
          7      & * & . & . & . & . & . & . & . & . & .\\    
          8      & . & . & . & . & . & * & . & . & . & .\\    
          9      & . & . & . & . & . & . & . & . & . & .\\    
         10      & . & . & . & . & . & . & . & * & * & .      
\end{sqmat} 
\end{equation}
%
where the asterisks means a non-null element. Starting with row index
1, we see that its corresponding connecting set is rows 1,3,10 and
columns 8,9. So that the linear transformation $!y = !Q!x$ can be
represent as
%
\begin{equation}
\begin{aligned}
 y_2&=0\\
 y_9&=0\\
%
 \begin{sqmat}{c} y_1 \\ y_3 \\ y_{10} \end{sqmat} 
   & = !Q^1 \begin{sqmat}{c} x_8 \\ x_9 \end{sqmat} \\
%
 y_4 & = Q^{4,10} x_{10}\\ 
%
 \begin{sqmat}{c} y_5 \\ y_6 \end{sqmat} 
   & = !Q^3 \begin{sqmat}{c} x_4 \\ x_5\\ x_7 \end{sqmat} \\
\end{aligned}
\end{equation}
%
So that \lfrc{$!Q$} decomposes in three blocks of $3\times 2$, $1\times 1$,
and $2\times 3$, two void rows (2,9) and two row columns (2,3). One
basic operation of class idmap is to compute the row connected to a
given column index. 

\SSection{Temporal dependent boundary conditions} 

Temporal dependent boundary conditions are treated in \pfem{} assuming
that the boundary condition on a set of nodes $J=\{j_1,j_2,j_3,\dots,j_N\}$
is of the form 
%
\begin{equation}
\begin{aligned}
   \phi_{j_1}(t) &= a_1 \phi(t) \\
   \phi_{j_2}(t) &= a_2 \phi(t) \\
        &\vdots \\
   \phi_{j_N}(t)& = a_N \phi(t) 
\end{aligned}
\end{equation}
%
where the $a_k$ are spatial amplitudes and the function $\phi(t)$ is
the temporal part of the dependency. For instance consider the
solution of a problem of heat conduction coupledt with diffusion of a
component in a rectangular region, as depicted in
figure~\ref{fg:tempbc}. The boundary condition on side $AD$ is of the
form
%
\begin{equation} 
   T(x,t) = 100\gC \, [4x(L-x)/L^2] \, \sin (5 t)
\end{equation}
%
where $L$ is the distance $AD$. 

\begin{figure*}[htb]
\centerline{\includegraphics{./OBJ/tempbc.\EPSPDF}}
\caption{Example with time dependent boundary conditions.}
\label{fg:tempbc}
\end{figure*}

The nodes on side $AD$ are $\{1,7,13,19,25,31,37\}$ and
we assume that concentration and temperature are fields 1 and 3
respectively. Here we can take
%
\begin{equation}
\begin{aligned}
   a_j &= 4x_j(L-x_j)/L^2], \textrm{ for  } j \textrm { in } J\\
   \phi(t) = 100\gC \, \sin 5 t
\end{aligned}
\end{equation}
%
Boundary conditions depending on time, as in this example are entered
in PETSc-FEM with a \verb+fixa_amplitude+ section. The general form is
%
\begin{alltt}
fixa_amplitude  \Arg{function}
\Arg{param1} \Arg{val1}
\Arg{param2} \Arg{val2}
...
__END_HASH__
\Arg{node1} \Arg{dof1} \Arg{val1}
\Arg{node2} \Arg{dof2} \Arg{val2}
...
\Arg{nodeN} \Arg{dofN} \Arg{valN}
__END_FIXA__
\end{alltt}
%
For instance, for the example above it may look like this
%
\begin{verbatim}
fixa_amplitude sine
omega 5.
amplitude 1.
__END_HASH__
  0.00000 2
  0.55556 2
  0.88889 2
  1.00000 2
  0.88889 2
  0.55556 2
  0.00000 2
__END_FIXA__
\end{verbatim}

\SSSection{Built in temporal functions} 

The following temporal functions are currently available in \pfem{}:
%
\begin{description}

%<*>---<*>---<*>---<*>---<*>---<*>---<*>---<*>---<*>---<*>---<*> 
\item\funhead{ramp}{Double ramp function}
%
See~\ref{fg:ramp}. 
%
\begin{equation} 
  \phi(t) = \begin{cases} 
                \phi_1 &; \textrm{ if } t\le t_1\\
                \phi_2 &; \textrm{ if } t\ge t_2\\
                \phi_1+s (t-t_1) &; \textrm{ if } t_1<t<t_2;
            \end{cases}
\end{equation}
%
where the slope $s$ is
%
\begin{equation} 
  s = \frac{\phi_2-\phi_1}{t_2-t_1}
\end{equation}
%
The parameters are
\begin{itemize}
\item\verb+start_time+=$t_1$. (default 0.) The starting time of the ramp. 
\item\verb+start_value+=$\phi_1$. (default 0.) The starting value of the ramp.
\item\verb+slope+=$s$. (default 0.) The slope in the ramp region. 
\item\verb+end_time+=$t_2$. (default =\verb+start_time+) The end time of the ramp. 
\item\verb+end_value+=$\phi_2$. (default =\verb+start_value+)  
                The end value of the ramp.
\end{itemize}
%
Only one of \verb+slope+ and \verb+end_value+ must be specified. If
the end values are not defined, then they are assumed to be equal to
the starting ones. If the end time is equal to the starting time, then
the end time is taken as $\infty$, i.e. a single ramp starting at
\verb+start_time+. 
%
\begin{figure*}[htb]
\centerline{\includegraphics{./OBJ/ramp.\EPSPDF}}
\caption{Ramp function}
\label{fg:ramp}
\end{figure*}

%<*>---<*>---<*>---<*>---<*>---<*>---<*>---<*>---<*>---<*>---<*> 
\item\funhead{smooth\_ramp}{Smooth double 
              ramp function (hyperbolic tangent)}
%
See figure~\ref{fg:smramp}. 
%
\begin{equation} 
  \phi(t) = \frac{\phi_1+\phi_0}2 + \frac{\phi_1-\phi_0}2 \, \tanh \frac{t-t_s}\tau
\end{equation}
%
This function is somewhat analogous to \verb+ramp+ in the sense that
it goes from a starting constant value to another final one, but 
smoothly. Beware that the start and end values are reached only for
$t\to\pm\infty$. 

\begin{itemize}
\item\verb+switch_time+=$t_s$. (default 0.) The time at which $\phi$
passes by the mean value $\bar\phi$ between $\phi_0$ and $\phi_1$. 

\item\verb+time_scale+=$\tau$. (default: none) This somewhat represents
the duration of the change from the starting value to the end
value. During the interval from $t_s-\tau$ to $t_s+\tau$ (i.e. a total
duration of $2\tau$) $\phi$ goes from
$\phi_0+0.12\Delta\phi$ to $\phi_0+0.88\Delta\phi$. 

\item\verb+start_value+=$\phi_0$. (default 0.) The limit valu for
$t\to-\infty$.
 
\item\verb+end_value+=$\phi_1$. (default 1.) The limit valu for
$t\to+\infty$. 
\end{itemize}

\begin{figure*}[htb]
\centerline{\includegraphics{./OBJ/smramp.\EPSPDF}}
\caption{Smooth ramp with the hyperbolic tangent function}
\label{fg:smramp}
\end{figure*}

%<*>---<*>---<*>---<*>---<*>---<*>---<*>---<*>---<*>---<*>---<*> 
\item\funhead{sin}{Trigonometric sine function}\label{sec:sine}  
%
\begin{equation} 
  \phi(t) = \phi_0 + A \, \sin(\omega t +\varphi)
\end{equation}
%

\begin{figure*}[htb]
\centerline{\includegraphics{./OBJ/sine.\EPSPDF}}
\caption{Sine function.}
\label{fg:sine}
\end{figure*}
 

The parameters are
%
\begin{itemize}
\item\verb+mean_val+=$\phi_0$. (default 0.) 
\item\verb+amplitude+=$A$. (default 1.) 
\item\verb+omega+=$\omega$. (default: none) 
\item\verb+frequency+=$\omega/2\pi$. (default: none) 
\item\verb+period+=$T = 2\pi/\omega$. (default: none) 
\item\verb+phase+=$\varphi$. (default 0.) 
\end{itemize}
%
Only one of \verb+omega+,  \verb+frequency+ or \verb+period+ must be
specified. 

%<*>---<*>---<*>---<*>---<*>---<*>---<*>---<*>---<*>---<*>---<*> 
\item\funhead{cos}{Trigonometric cosine function}\label{sec:cosine}  
%
\begin{equation} 
  \phi(t) = \phi_0 + A \, \cos(\omega t +\varphi)
\end{equation}
%
The parameters are the same as for \verb+sin+ (see~\ref{sec:sine})

%<*>---<*>---<*>---<*>---<*>---<*>---<*>---<*>---<*>---<*>---<*> 
\item\funhead{piecewise}{Piecewise linear function}\label{sec:piecewise}  
%
This defines a piecewise linear interpolation for a given 
series of time instants $t_j$ (ordered such that $t_j<t_{j+1}$) 
and amplitudes $\phi_j$ ($1<j<n$) entered by
the user. The interpolation is
%
\begin{equation} 
  \phi(t) = \begin{cases}
          0 ;& \textrm{ if } t<t_0 \textrm{ or } t>t_n \\
          \phi_k + \frac{\phi_{k+1}-\phi_k}{t_{k+1}-t_k} \, {t-t_k} ;&
                        \textrm{ if } t_k<t<t_{k+1}
            \end{cases}
\end{equation}
%
If \verb+final_time+ is defined, then th function is extended
periodically with period $t_n-t_1$. The parameters are

\begin{itemize}
\item\verb+npoints+ (integer) The number $n$ of points to be entered.
\item\verb+t+ ($n$ doubles) The instants $t_j$
\item\verb+f+ ($n$ doubles) The function values $\phi_j$
\item\verb+final_time+ (double) The function is extended from $t_n$ to
this instant with period $t_n-t_1$. 
\end{itemize}

%<*>---<*>---<*>---<*>---<*>---<*>---<*>---<*>---<*>---<*>---<*> 
\item\funhead{spline}{Spline interpolated function}\label{sec:spline}  
%
This is similar to \verb+piecewise+ but data is smoothly interpolated
using splines. 
%
\begin{itemize}
\item\verb+npoints+ (integer) The number $n$ of points to be entered.
\item\verb+t+ ($n$ doubles) The instants $t_j$
\item\verb+f+ ($n$ doubles) The function values $\phi_j$
\item\verb+final_time+ (double) The function is extended from $t_n$ to
this instant with period $t_n-t_1$. 
\end{itemize}

%<*>---<*>---<*>---<*>---<*>---<*>---<*>---<*>---<*>---<*>---<*> 
\item\funhead{spline\_periodic}
      {Spline interpolated periodic function}\label{sec:splineperiodic}  
%
Spline interpolation with \verb+piecewise+ may give poor results,
specially if you try to match smoothly the beginning and end of the
period. \verb+spline_periodic+  may give better results, at the cost
of being restricted to enter the data function in an equally spaced
grid ($t_{j+1}-t_j=\Delta t=\cnst$). 
%
\begin{itemize}\label{eq:cosinfun} 
\item\verb+npoints+ (integer) The number $n$ of points to be
entered. 
\item\verb+period+ (double) The period $T$ of the function. 
\item\verb+start_time+ (double, default = 0.) The first time instant $t_1$. The
remaining instants are then defined as $t_j = t_1 +  [(j-1)/(n-1)]\, T$. 
\item\verb+f+ ($n$ doubles) The function values $\phi_j$ at times
           $\{t_j\}$. 
\end{itemize}
%
\paragraph{Example:} The lines
%
\begin{verbatim}
fixa_amplitude spline_periodic
period 0.2
npoints 5
start_time 0.333
ampl_vals  1            0        -1         0.     1.
\end{verbatim}
%
Defines a function with period $T=0.2$ that takes the values 
%
\begin{equation}
\begin{aligned}
\phi(0.333+n T) &= 1\\
\phi(0.333+(n+\sfr14) T) &= 0\\
\phi(0.333+(n+\sfr12) T) &= -1\\
\phi(0.333+(n+\sfr34) T) &= 0
\end{aligned}
\end{equation}
%
Actually, the resulting interpolated function is simply
%
\begin{equation} 
  \phi(t) = \cos \LL(2\pi \, \frac{t-0.333}T\RR)
\end{equation}

%# Current line ===========  

\paragraph{Implementation details:} If we define the phase $\theta$ as 
%
\begin{equation} 
   \theta = 2\pi \, \frac{t-t_1}T
\end{equation}
%
then $\phi(\theta)$ is periodic with period $2\pi$. We can decompose
it in two even functions $\phi^\pm(\theta)$ as
%
\newcommand{\phip}{{\phi^+}}
\newcommand{\phim}{{\phi^-}}
\newcommand{\Dtheta}{{\Delta\theta}}
%
\begin{equation}
\begin{aligned}
   \phip(\theta) &= \frac{\phi(\theta)+\phi(-\theta)}2 \\
   \phim(\theta) &= \frac{\phi(\theta)-\phi(-\theta)}{2\sin(\theta)}
\end{aligned}
\end{equation}
%
so that
%
\begin{equation} 
   \phi(\theta) = \phip(\theta) + \sin(\theta) \, \phim(\theta)
\end{equation}
%
As $\phi^\pm$ are even function, they may be put in terms of
$x=(1-\cos\theta)/2$. So that $x_j$ and $\phi^\pm_j$ are computed and
by construction, only one half of the values (say $j=1$ to
$j=m=(n-1)/2+1$) are relevant. 
The values $\phi^-_0$ and $\phi^-_m$ have to be computed
specially, since $\sin(\phi_j)=0$ for them. If we take the limit 
%
\begin{equation} 
  \phi^-_0 = \lim_{\theta\to0}
  \frac{\phi(\theta)-\phi(-\theta)}{2\sin\theta}
\end{equation}
%
then, if linear interpolation is assumed in each interval, it can be
shown that 
%
\begin{equation} 
  \frac{\phi(\theta)-\phi(-\theta)}2 = 
        \frac{\phi_2-\phi_{n-1}}{\Dtheta} \, \theta,\ \ \textrm{ for }
        |\theta|<\Dtheta
\end{equation}
%
Then 
%
\begin{equation} \label{eq:philim}  
  \lim_{\theta\to0}
       \frac{\phi(\theta)-\phi(-\theta)}{2\sin\theta} = 
             \frac{\phi_2-\phi_{n-1}}{2\Dtheta}
\end{equation}
%
We could take then this value for $\phi^-_0$, however, this introduces
some noise. For instance, if $\phi(\theta) = \sin(\theta)$ then
$\phi^+\equiv=0$ and $\phi^-_j=1$ for $j\neq 1,m$, and
(\ref{eq:philim}) gives 
%
\begin{equation} 
  \lim_{\theta\to0}
       \frac{\phi(\theta)-\phi(-\theta)}{2\sin\theta} = 
             \frac{\sin\Dtheta}{\Dtheta}
\end{equation}
%
We introduce, then a \emph{``correction factor''} $\frac{\Dtheta}{\sin\Dtheta}$
so that we define
%
\begin{equation} 
  \phi^-_0 = \frac{\phi_2-\phi_{n-1}}{2\sin\Dtheta}
\end{equation}
%
Analogously, we define
%
\begin{equation} 
  \phi^-_m = \frac{\phi_{m-1}-\phi_{m+1}}{2\sin\Dtheta}
\end{equation}
%
\end{description}

\SSSection{Implementation details} 

Temporal boundary conditions are stored in an object of class
\verb+Amplitude+. This objects contains a key string identifying the
function and a text hash table containing the parameters for that
function (for instance \verb+omega+$\to$3.5, \verb+amplitude+$\to$2.3,
etc...). Recall that each node/field pair may depend on some free
degrees of freedom (those in $U$ in (\ref{eq:urep})) and some others
fixed (those in $\bar U$). For those fixed, there is an array
containing both an amplitude (i.e. a double) and a pointer to an
\verb+Amplitude+ object. If the condition don't depends on time, then
the pointer is null. 

\SSSection{How to add a new temporal function} 

If none of the built-in time dependent functions fit your needs, then
you can add you own temporal function. Suppose you want a function of
the form 
%
\begin{equation} 
   \phi(t) = \begin{cases}
                 0                  &; t<0;\\
                 \frac t{(1+t/\tau)}&; t>=0;
             \end{cases}
\end{equation}
%
Follow these steps
%
\begin{enumerate}
\item Give a name to it ending in \verb+_function+. We will use
\verb+my_own_function+ in the following. 
%
\item Declare it with a line like
%
\begin{verbatim}
AmplitudeFunction my_own_function;
\end{verbatim}
%
(In the core \pfem{} this is done in \verb+dofmap.h+.)

\item Write the definition. (You can find typical definitions in file
\verb+tempfun.cpp+). You can use the macro 
%
\begin{alltt}
SGETOPTDEF(\Arg{type},\Arg{name},\Arg{default\_value});
\end{alltt}
%
for retrieving values from the hash table defined by the user in the
data file. 

\item Register it in the temporal function table with a call such
as the following, preferably after the call to
\verb+Amplitude::initialize_function_table+ in the \verb+main()+. For
instance

%
\begin{alltt}
  Amplitude::initialize_function_table();
  Amplitude::add_entry("smooth_ramp",
             &smooth_ramp_function); // \vrbsl{\textless- add  this line}
\end{alltt}
%
(In the core \pfem{} this is done inside
the \verb+Amplitude::initialize_function_table+.)

\item Use it in your data files as follows
%
\begin{verbatim}
amplitude_function my_own
tau 3.0
__END_HASH__
1 2 3.
...
\end{verbatim}
\end{enumerate}

\SSSection{Dynamically loaded amplitude functions} 

Another possibility to add new amplitude functions is through loading
functions at run-time. This avoids re-linkage and modification of the
sources. (Note: You need to have the module compiled with the
appropriate flag (either the compilation flag \verb+-DUSE_DLEF+ or,
the Makefile variable
\verb+USE_DYNAMICALLY_LOADED_EXTENDED_FUNCTIONS = yes+, either in the
\verb+Makefile.base+ or the \verb+../Makefile.defs+ file.) 

These amplitude functions are written in C++ in source files, say
\verb+fun.cpp+. The compiled files have extension \verb+.efn+ (from
\emph{``extension function''}) for instance \verb+fun.efn+ in this
example.  The \verb+Makefile.base+ file provides the appropriate rules
to generate the compiled \verb+.efn+ functions from the source. For
each amplitude you want to define you have to write three functions
%
\begin{alltt}
extern "C" void \Arg{prefix}init_fun(TextHashTable *thash,void *&fun_data);
extern "C" double \Arg{prefix}eval_fun(double t,void *fun_data);
extern "C" void \Arg{prefix}clear_fun(void *fun_data) ; // \vrbsl{this is optional}
\end{alltt}
%
Prefix may be an empty string, or a non-empty string ending in
underscore. Use of non-empty prefix is explained below. 
The macros \verb+INIT_FUN+, \verb+EVAL_FUN+ and \verb+CLEAR_FUN+
expand to these definitions. 
The second function \verb+eval_fun(..)+ is that one that defines the
relation between the time and the corresponding amplitude. The first
\verb+init_fun(...)+ one can be used to make some initialization,
normally it is called only once. The last one \verb+clear_fun(...)+ is
called after all calls to \verb+eval_fun(...)+. For simple functions
you may not need the init and clear functions, for instance if you
want a linear ramp from 0 at $t=0$ to 1 at $t=1$, then it can be done
with a simple file \verb+fun0.cpp+ this:
%
\begin{verbatim}
// File: fun0.cpp

#include <math.h>
#include <src/ampli.h>

INIT_FUN {}

EVAL_FUN {
  if (t < 0) return 0.;
  else if (t > 1.) return 1.;
  else return t;
}

CLEAR_FUN {}
\end{verbatim}
%
You then have to do a {\alltt \$ make fun0.efn} that will create the
\verb+fun0.efn+ shared object file. Dynamically loaded functions can
be used using the \verb+fixa_amplitude dl_generic+ clause and then
giving the name of the file with the \verb+ext_filename+ option. 
For instance,
%
\begin{alltt}
\Arg{... previous lines here ...}
fixa_amplitude dl_generic
ext_filename "./fun0.efn"
__END_HASH__
 1 1 1.
\Arg{... more node/dof/val combinations here ...}
__END_FIXA__
\end{alltt}
%
You can also have dynamically loaded functions that use parameters
loaded via the table of options at run time. For this you are passed
the \verb+TextHashTable *+ object entered by the user to the
\verb+init+ function. You then can read parameters from these but
in order that the
\emph{``init''} function do anything useful it has to be able to pass
data to the \emph{``eval''} function. 
% This is done by passing some
% data through the generic \emph{``fun\_data''} pointer. 
Normally you define a \emph{``struct''} or class to hold the data,
create it with \verb+new+ or \verb+malloc()+ put the data in it, and
then pass the address via the \emph{``fun\_data''} pointer. Later, in
subsequent calls to \verb+eval_fun(...)+ it is passed the same pointer
so that you can recover the information, previous a static cast. Of
course, in order to avoid memory leak you have to free the allocated
memory somewhere, this is done in the \verb+clear(...)+ function. For
instance, the following file defines a ramp function where youcan set
the four values $t_0, f_0,t_1, f_1$
%
\begin{verbatim}
#include <math.h>

#include <src/fem.h>
#include <src/getprop.h>
#include <src/ampli.h>

struct  MyFunData {
  double f0, f1, t0, t1, slope;
};

INIT_FUN {
  int ierr;
  // Read values from option table
  TGETOPTDEF(thash,double,t0,0.);
  TGETOPTDEF(thash,double,t1,1.);
  TGETOPTDEF(thash,double,f0,0.);
  TGETOPTDEF(thash,double,f1,1.);

  // create struct data and set values
  MyFunData *d = new MyFunData;
  fun_data = d;
  d->f0 = f0;
  d->f1 = f1;
  d->t0 = t0;
  d->t1 = t1;
  d->slope = (f1-f0)/(t1-t0);
}

EVAL_FUN {
  MyFunData *d = (MyFunData *) fun_data;
  // define ramp function
  if (t < d->t0) return d->f0;
  else if (t > d->t1) return d->f1;
  else return d->f0 + d->slope *(t - d->t0);
}

CLEAR_FUN {
  // clear allocated memory
  MyFunData *d = (MyFunData *) fun_data;
  delete d;
  fun_data=NULL;
}
\end{verbatim}
%
Another possibility, perhaps more simple, would be to use a global
\verb+MyFunData+ object but if several \verb+fixa_amplitude+ entries
that use the same function are used then the data created by the first
entry would be overwritten by the second entry. 

\SSSection{Use of prefixes} 

Several functions can be written in the same \verb+.cpp+ file using a
prefix ending in underscore, for instance if you want to define a
\verb+gaussian+ function then you define functions with
{\alltt\Arg{prefix}=gaussian\_}, for instance
%
\begin{alltt}
extern "C" void gaussian_init_fun(TextHashTable *thash,void *&fun_data);
extern "C" double gaussian_eval_fun(double t,void *fun_data);
extern "C" void gaussian_clear_fun(void *fun_data) ; // \vrbsl{this is optional}
\end{alltt}
%
The macros \verb+INIT_FUN1(name)+, \verb+EVAL_FUN1(name)+ and
\verb+CLEAR_FUN1(name)+ do it for you, where \verb+name+ is the
prefix, with the underscore removed, for instance
%
\begin{alltt}
\allttbraces%
\Arg{... headers ...}

struct  MyGaussFunData \{
   \Arg{... your data here ...}
\};

INIT_FUN1(gaussian) \{
   \Arg{... read data...}
   MyGaussFunData *d = new MyGaussFunData;
   fun_data = d;
   \Arg{... set data in {\tt d} ...}
\}

EVAL_FUN(gaussian) \{
  MyGaussFunData *d = (MyGaussFunData *) fun_data;
  \Arg{... use data in {\tt d} and define amplitude ...}
\}

CLEAR_FUN(gaussian) \{
  // clear allocated memory
  MyGaussFunData *d = (MyGaussFunData *) fun_data;
  delete d;
  fun_data=NULL;
\}
\end{alltt}
%
Finally, yet another approach is to have a \emph{``wrapper''} class
with methods, \verb+init+ and \verb+eval+. The macro
\verb+DEFINE_EXTENDED_AMPLITUDE_FUNCTION(class_name)+ is in charge of
creating and destroying the object. For instance the following file
\verb+fun3.cpp+ defines two functions \verb+linramp+ and
\verb+tanh_ramp+.
%
\begin{verbatim}
// File fun3.cpp
#include <src/ampli.h>

class  linramp {
public:
  double f0, f1, t0, t1, slope;
  void init(TextHashTable *thash);
  double eval(double);
};

void linramp::init(TextHashTable *thash) {
  int ierr;
  TGETOPTDEF_ND(thash,double,t0,0.);
  TGETOPTDEF_ND(thash,double,t1,1.);
  TGETOPTDEF_ND(thash,double,f0,0.);
  TGETOPTDEF_ND(thash,double,f1,1.);
  slope = (f1-f0)/(t1-t0);
}

double  linramp::eval(double t) {
  if (t < t0) return f0;
  else if (t > t1) return f1;
  else return f0 + slope *(t - t0);
}

DEFINE_EXTENDED_AMPLITUDE_FUNCTION(linramp);

//---:---<*>---:---<*>---:---<*>---:---<*>---:---<*>---:---<*>---: 
class tanh_ramp {
public:
  double A, t0, delta, base;
  void init(TextHashTable *thash);
  double eval(double);
};

void tanh_ramp::init(TextHashTable *thash) {
  int ierr;
  TGETOPTDEF_ND(thash,double,base,0.);
  TGETOPTDEF_ND(thash,double,A,1.);
  TGETOPTDEF_ND(thash,double,t0,0.);
  TGETOPTDEF_ND(thash,double,delta,1.);
  assert(delta>0.);
}

double  tanh_ramp::eval(double t) {
  if (t < t0) return base;
  else return base + A * tanh((t-t0)/delta);
}

DEFINE_EXTENDED_AMPLITUDE_FUNCTION(tanh_ramp);
\end{verbatim}

\SSSection{Time like problems.} 

The mechanism of passing time to boundary conditions and elemsets may
be used to pass any other kind of data, since it is passed as a
generic pointer \verb+void *time-data+. This may be used to treat
other problems where an external parameter exists, for instance 

\begin{description}
\item[Continuation problems] For instance a NS solution at high Reynolds
number may be obtained by continuation in Re, and then we can use it
as the external (tome-like) parameter. In general, we can perform
continuation in a set of parameters, so that the \verb+time_data+
variable should be an array of scalars. 

\item[Multiple right hand sides] Here the time like variable may be an
integer: the number of right hand side case. 
\end{description}


% Local Variables: *
% mode: latex *
% tex-main-file: "petscfem.tex" *
% End: *
