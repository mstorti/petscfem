%__INSERT_LICENSE__
\documentclass[a4paper,11pt]{article} 
%---:---<*>---:---<*>---:---<*>---:---<*>---:---<*>---:---<*>---: 
% HOW TO ADD MATERIAL HERE
%
% 1./ Do not use environment 'align'
% 2./ Be careful with calls to $!a$ for vector a. Sometimes if
%     the material inside an equation is too simple latex2html may
%     not process it.
% 3./ Do not use the `algorithmic' and 'algorithm. environments.
%     Latex2Html doesn't like them. 
% 4./ After adding documentation, please check that it process
%     OK and that equation numbering is OK in the HTML output. Also that
%     image processing in the generation of the manual is OK. 
%
%---:---<*>---:---<*>---:---<*>---:---<*>---:---<*>---:---<*>---: 
%Section{
%<: if (!$html) { :>
\newif\ifpdf 
\ifx\pdfoutput\undefined
  \pdffalse % Not running PDFLaTeX 
  \usepackage[dvips]{graphicx}
  \def\EPSPDF{eps}
\else
  \pdfoutput=1 % Running PDFLaTeX 
  \pdftrue 
  \usepackage[pdftex]{graphicx} \pdfcompresslevel=9
  \def\EPSPDF{pdf}
\fi
%<: } else { :>
\usepackage[dvips]{graphicx}
\newcommand{\EPSPDF}{eps}
%<: } :>
%\usepackage{showkeys}   %__COMMENT_IN_FINAL_VERSION__
\usepackage{makeidx}
%\usepackage[dvips]{graphicx}
%\usepackage[dvips]{changebar}
\usepackage{html,htmllist,heqn}
%<: if (!$html) { :>
\usepackage{algorithm,algorithmic}
%<: } :>
\usepackage{amsmath,alltt,mstorti,moreverb,url}
\usepackage{hyperref,nameref}
%
\scrollmode
\sloppy
\bibliographystyle{plain}
%
\latex{
\setlength{\textheight}{22truecm}
\setlength{\textwidth}{15truecm}
\advance\hoffset by -1truecm
\advance\voffset by -2truecm
}
%<*>---<*>---<*>- MATH DEFINITIONS --<*>---<*>---<*>---<*>---<*> 
\newcommand{\Nnod}{{N_{\mathrm{nod}}}}
\newcommand{\dof}{{\mathrm{d.o.f.}}}
\newcommand{\conv}{{\mathrm{conv}}}
\newcommand{\diff}{{\mathrm{diff}}}
\newcommand{\Ndof}{{N_\dof}}
\newcommand{\ndof}{{n_\dof}}
\newcommand{\ndim}{{n_d}}
\newcommand{\nel}{{n_{\mathrm{el}}}}
\newcommand{\NF}{{\mathrm{NF}}}
\newcommand{\barU}{{\bar{!U}}}
\newcommand{\bU}{{\bar U}}
\newcommand{\barQ}{{\bar{!Q}}}
\newcommand{\Nelem}{{n_\mathrm{elem}}}
\newcommand{\nprops}{{n_\mathrm{props}}}
\newcommand{\nfoils}{{n_\mathrm{foils}}}
\newcommand{\refer}{{\mathrm{ref}}}
\newcommand{\atm}{{\mathrm{atm}}}
\newcommand{\bPi}{\boldsymbol{\Pi}}
\newcommand{\btau}{\boldsymbol{\tau}}
\newcommand{\pfem}{PETSc-FEM}
\newcommand{\PETSc}{PETSc}
\newcommand{\F}{\mathcal{F}}
%\newcommand{\Hen}{\mathcal{H}} % enthalpy
\newcommand{\D}{\mathcal{D}}
\newcommand{\cU}{{\mathcal{U}}}
\newcommand{\bD}{\bar{\D}}
\newcommand{\bA}{{\bar A}}
\newcommand{\Psupg}{{P_\mathrm{SUPG}}}
\newcommand{\deltasc}{{\delta_\mathrm{sc}}}
\newcommand{\swf}{s_{\mathrm{wf}}}
\newcommand{\cswf}{\factorial \swf}
\newcommand{\bswf}{\bar {s_{\mathrm{wf}}}}
\newcommand{\bsupg}{\beta_{\mathrm{SUPG}}}
\newcommand{\taufac}{\tau_{\mathrm{sel}}}
% For LES
\newcommand{\Se}{{\Sigma_e}}
\newcommand{\dS}{{\di\Sigma}}
\newcommand{\ustar}{{u^*}}
\newcommand{\Dt}{{\Delta t}}
% IISD DistMap
\newcommand{\btA}{{\tilde{!A}}}
\newcommand{\btb}{{\tilde{!b}}}
\newcommand{\nproc}{{n_{\mathrm{proc}}}}
%
\newcommand{\AFS}{{Algebraic Fractional Step}}
\newcommand{\mbf}[1]{{\mathbf{#1}}}
\newcommand{\refecu}[1]{{eq. (\ref{#1})}}
\newcommand{\reffig}[1]{{fig.~\ref{#1}}}
%---:---<*>---:---<*>---:---<*>---:---<*>---:---<*>---:---<*>---: 
% Hydrology module
\newcommand{\Gs}{{\Gamma_s}}
%
%\newcommand{\ttlb}{\char'173} 
%\newcommand{\ttrb}{\char'175} 
% \newcommand{\allttbraces}{\let\{=\ttlb \let\}=\ttrb}
%\allttbraces
\newcommand{\tl}{`\<}
\newcommand{\tg}{`\>}
\newcommand{\Arg}[1]{{\rm\slshape \html{[#1]}\latex{\textless#1\textgreater}}}
\newcommand{\vrbsl}[1]{{\rm\slshape #1}}
%\newcommand{\Arg}[1]{{{\tt\char`\<}{\rm\slshape #1}{\tt\char`\>}}}
%\newcommand{\Arg}[1]{{\rm\slshape \textless#1\textgreater}}
%# Current line ===========  
%
% Translates section commands to \SSSSection commands
%
\newcommand{\Section}[1]{\section{#1}}
\newcommand{\SSection}[1]{\subsection{#1}}
\newcommand{\SSSection}[1]{\subsubsection{#1}}
\newcommand{\SSSSection}[1]{\paragraph{#1}}
%\def\defang#1{$\langle${\slshape #1}$\rangle$}
\latex{\newcommand{\funhead}[2]{\boxed{{\tt #1}\textbf{: #2}}}}
\html{\newcommand{\funhead}[2]{{\Large{\tt #1}\textbf{: #2}}}}
%
\latex{\newcommand{\filehead}[1]{\boxed{\hbox{\sl File: {\tt #1}}}}}
\html{\newcommand{\filehead}[1]{\Large{File: \emph{#1}}}}
\renewcommand{\labelitemii}{$\triangleright$}
\newcommand{\manpages}{\htmladdnormallink{manual pages}{../manual/html/index.html}}
%\def\lforce#1{#1}
%\DeclareTextSymbol{\textless}{MY1}{`\<}
%
%<*>---<*>---<*>---<*>---<*>---<*>---<*>---<*>---<*>---<*>---<*> 
%
%
%<*>---<*>---<*>---<*>---<*>---<*>---<*>---<*>---<*>---<*>---<*> 
\input version.tex
\makeindex
\begin{document}
\sloppy
%
\title{\pfem: A General Purpose, Parallel, \\
Multi-Physics FEM Program.\\
User's Guide}
\author{
Mario Storti, Norberto Nigro and Rodrigo Paz\\ 
Centro Internacional de M\'etodos Computacionales en Ingenier\'\i{}a
(CIMEC)\\
Santa Fe, Argentina\\
http://venus.ceride.gov.ar/CIMEC, http://www.cimec.com.ar\\
}
\date{\today}
\maketitle 
%{\tiny\verb+$Id: petscfem.in.tex,v 1.26 2003/10/21 13:23:08 mstorti Exp $+}
%
\begin{abstract}
%fixme:= incluir referencia a PETSc
This is the documentation for \pfem (current version 
{\tt petscfem-\petscfemversion}, a general purpose, parallel,
multi-physics FEM program for CFD applications based on
PETSc. \pfem{} comprises both a library that allows the user to
develop FEM (or FEM-like, i.e. non-structured mesh oriented) programs, and
a suite of application programs. It is written in the C++ language
with an OOP (Object Oriented Programming) philosophy, but always
keeping in mind the scope of efficiency. 
\end{abstract}
%
\newpage
\tableofcontents
%
% OK. This is tricky. :-) 
% I comment and uncomment the following '\inputs' while testing,
% so that we risk to process this file with some parts commented out
% while processing for the final version. So that, before processing 
% we run from the Makefile a one-liner Perl script ($ perl -pi.bak -e
% 's/...') that uncomment this lines.
%
\input gpl            %__UNCOMMENT_IN_FINAL_VERSION__
\input prev1          %__UNCOMMENT_IN_FINAL_VERSION__
\input layout         %__UNCOMMENT_IN_FINAL_VERSION__
\input thash          %__UNCOMMENT_IN_FINAL_VERSION__
\input idmap          %__UNCOMMENT_IN_FINAL_VERSION__
\input compprof       %__UNCOMMENT_IN_FINAL_VERSION__
\input pfmat          %__UNCOMMENT_IN_FINAL_VERSION__
\input advec          %__UNCOMMENT_IN_FINAL_VERSION__
\input advdif         %__UNCOMMENT_IN_FINAL_VERSION__
\input advdif2        %__UNCOMMENT_IN_FINAL_VERSION__
\input ns             %__UNCOMMENT_IN_FINAL_VERSION__
\input tests          %__UNCOMMENT_IN_FINAL_VERSION__
\input fastmat2       %__UNCOMMENT_IN_FINAL_VERSION__
\input dx             %__UNCOMMENT_IN_FINAL_VERSION__
%
% These have to be commented out
%%%% THIS IS TO TRY OUT THE AUXILIARY FILES FOR DOC OPTIONS %%%%%%
%\input trygetopt.tex %__COMMENT_IN_FINAL_VERSION__
%%%%%%%%%%%%%%%%%%%%%%%%%%%%%%%%%%%%%%%%%%%%%%%%%%%%%%%%%%%%%%%%%%

\Section{Authors}
%
\begin{description}
\item[Mario A. Storti (CIMEC)]
PETSc-FEM kernel, NS and AdvDif modules. 
\item[Norberto M. Nigro (CIMEC)]
PETSc-FEM kernel, NS and AdvDif modules. 
\item[Rodrigo R. Paz (CIMEC)]
AdvDif module, specially the hydrology module. 
\end{description}
%
Also many people from the following institutions has contributed
directly or indirectly to the generation of this code:
%
\begin{description}
\item[CIMEC] \emph{International Center for
Computational Methods in Engineering, Santa Fe, Argentina} 
\url{http://www.cimec.org.ar}
\item[INTEC] \emph{Instituto de Desarrollo Tecnol\'ogico para la
  Industria Qu\'\i{}mica}
\url{http://www.intec.unl.edu.ar}
\item[CONICET] \emph{Consejo Nacional de Investigaciones
  Cient\'\i{}ficas y T\'ecnicas}
\url{http://www.conicet.gov.ar}
\item[UNL] \emph{Universidad Nacional del Litoral}
\url{http://www.unl.edu.ar}
\end{description}
%
The code has been developed under financement from the following
grants (list grants here...)

\Section{Symbols and Acronyms}

\SSection{Acronyms} 
\begin{description}
\item[CFD:] Computational Fluid Dynamics
\item[DX:] IBM Data Explorer
\item[ePerl:] embedded Perl
\item[FDM:] Finite Difference Method
\item[FEM:] Finite Element Method
\item[IISD:] Interface Iterative -- Sub-domain Direct method
\item[KWM:] Kinematic Wave Model (see \ref{sec:hydro})
\item[MPI:] Message Passing Interface
\item[OOP:] Object Oriented Programming
\item[Perl:] Practical Extraction and Report Language
\item[PETSC:] Portable Extensible Toolkit for Scientific computations.
\item[SUPG:] Streamline Upwind/Petrov Galerkin
\item[ANN:] Approximate Nearest Neighbor problem. Also refers to the
library developped by David Mount and Sunil Arya
(\url{http://www.cs.umd.edu/~mount/ANN}). 

\end{description}

\Section{Symbols} 

\begin{itemize}
   \item $u,v=$ Streamwise and normal components of velocity.
\end{itemize}

\newpage
\bibliography{petscfem}

\begin{flushleft}
\printindex
\end{flushleft}

\end{document}

%Section{

%**start of math-def
\documentclass{article}
\usepackage{mstorti,amsmath}
\pagestyle{empty} 
\begin{document}
\begin{equation} 
%**end of math-def

%# Current line ===========  
%:eps:uosc:
u(t)\propto \sin(\omega t)
%:eps:tau2:
2\tau
%:eps:ts:
t_s
%:eps:t0:
t_0
%:eps:t1:
t_1
%:eps:phi:
\phi
%:eps:eta:
\eta
%:eps:hb:
h_b
%:eps:bphi:
\bar\phi
%:eps:phi0:
\phi_0
%:eps:phi1:
\phi_1
%:eps:hfilm:
h_{\mathrm{film}}
%:eps:tinf:
T_\infty
%:eps:ts:
T_s
%:eps::

%<*>---<*>---<*>---<*>---<*>---<*>---<*>---<*>---<*>---<*>---<*> 

"DONE
====
%# Current line ===========  
# Local Variables: $
# mode: outline-minor $
# eval:   (set (make-local-variable 'outline-regexp) "\\(\\\\\\|%\\)S*ection{") $
# eval: (setq latex-run-command "make_latex") $
# End: $
