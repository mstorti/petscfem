%__INSERT_LICENSE__
\documentclass[dvips,a4paper,11pt]{article} 
%
%Section{
%\usepackage{showkeys}   %__COMMENT_IN_FINAL_VERSION__
\usepackage{makeidx}
\usepackage[dvips]{graphicx}
\usepackage[dvips]{changebar}
\usepackage{html,htmllist,heqn}
\usepackage{algorithm,algorithmic}
\usepackage{amsmath,alltt,mstorti,moreverb,url}
%
\scrollmode
\sloppy
\bibliographystyle{plain}
%
\latex{
\setlength{\textheight}{22truecm}
\setlength{\textwidth}{15truecm}
\advance\hoffset by -1truecm
\advance\voffset by -2truecm
}
%<*>---<*>---<*>- MATH DEFINITIONS --<*>---<*>---<*>---<*>---<*> 
\newcommand{\Nnod}{{N_{\mathrm{nod}}}}
\newcommand{\dof}{{\mathrm{d.o.f.}}}
\newcommand{\conv}{{\mathrm{conv}}}
\newcommand{\diff}{{\mathrm{diff}}}
\newcommand{\Ndof}{{N_\dof}}
\newcommand{\ndof}{{n_\dof}}
\newcommand{\ndim}{{n_d}}
\newcommand{\nel}{{n_{\mathrm{el}}}}
\newcommand{\NF}{{\mathrm{NF}}}
\newcommand{\barU}{{\bar{!U}}}
\newcommand{\bU}{{\bar U}}
\newcommand{\barQ}{{\bar{!Q}}}
\newcommand{\Nelem}{{n_\mathrm{elem}}}
\newcommand{\nprops}{{n_\mathrm{props}}}
\newcommand{\nfoils}{{n_\mathrm{foils}}}
\newcommand{\refer}{{\mathrm{ref}}}
\newcommand{\atm}{{\mathrm{atm}}}
\newcommand{\bPi}{\boldsymbol{\Pi}}
\newcommand{\btau}{\boldsymbol{\tau}}
\newcommand{\pfem}{PETSc-FEM}
\newcommand{\F}{\mathcal{F}}
%\newcommand{\Hen}{\mathcal{H}} % enthalpy
\newcommand{\D}{\mathcal{D}}
\newcommand{\cU}{{\mathcal{U}}}
\newcommand{\bD}{\bar{\D}}
\newcommand{\bA}{{\bar A}}
\newcommand{\Psupg}{{P_\mathrm{SUPG}}}
\newcommand{\deltasc}{{\delta_\mathrm{sc}}}
\newcommand{\swf}{s_{\mathrm{wf}}}
\newcommand{\cswf}{\factorial\swf}
\newcommand{\bswf}{\bar {s_{\mathrm{wf}}}}
\newcommand{\bsupg}{\beta_{\mathrm{SUPG}}}
\newcommand{\taufac}{\tau_{\mathrm{sel}}}
% For LES
\newcommand{\Se}{{\Sigma_e}}
\newcommand{\dS}{{\di\Sigma}}
\newcommand{\ustar}{{u^*}}
\newcommand{\Dt}{{\Delta t}}
%
\newcommand{\AFS}{{Algebraic Fractional Step}}
%
% \newcommand{\ttlb}{\char'173} 
% \newcommand{\ttrb}{\char'175} 
% \newcommand{\allttbraces}{\let\{=\ttlb \let\}=\ttrb}
%\allttbraces
% \newcommand{\tl}{`\<}
% \newcommand{\tg}{`\>}
\newcommand{\Arg}[1]{{\rm\slshape \html{[#1]}\latex{\textless#1\textgreater}}}
%\newcommand{\Arg}[1]{{{\tt\char`\<}{\rm\slshape #1}{\tt\char`\>}}}
%\newcommand{\Arg}[1]{{\rm\slshape \textless#1\textgreater}}
%# Current line ===========  
%
% Translates section commands to \SSSSection commands
%
\newcommand{\Section}[1]{\section{#1}}
\newcommand{\SSection}[1]{\subsection{#1}}
\newcommand{\SSSection}[1]{\subsubsection{#1}}
\newcommand{\SSSSection}[1]{\paragraph{#1}}
%\def\defang#1{$\langle${\slshape #1}$\rangle$}
\latex{\newcommand{\funhead}[2]{\boxed{{\tt #1}\textbf{: #2}}}}
\html{\newcommand{\funhead}[2]{{\Large{\tt #1}\textbf{: #2}}}}
%
\latex{\newcommand{\filehead}[1]{\boxed{\hbox{\sl File: {\tt #1}}}}}
\html{\newcommand{\filehead}[1]{\Large{File: \emph{#1}}}}
\renewcommand{\labelitemii}{$\triangleright$}
\newcommand{\manpages}{\htmladdnormallink{manual pages}{../manual/html/index.html}}
%\def\lforce#1{#1}
%\DeclareTextSymbol{\textless}{MY1}{`\<}
%
%<*>---<*>---<*>---<*>---<*>---<*>---<*>---<*>---<*>---<*>---<*> 
%
%
%<*>---<*>---<*>---<*>---<*>---<*>---<*>---<*>---<*>---<*>---<*> 
\input version.tex
\makeindex
\begin{document}
\sloppy
%
\title{\pfem: A General Purpose, Parallel, Multi-Physics FEM Program.\\
User's Guide}
\author{
Mario Storti and N. Nigro\\ 
Centro Internacional de M\'etodos Computacionales en Ingenier\'\i{}a
(CIMEC)\\
Santa Fe, Argentina\\
http://venus.arcride.edu.ar/CIMEC\\
}
\date{\today}
\maketitle 
%{\tiny\verb+$Id: petscfem.tex,v 1.18 2001/05/26 17:00:35 mstorti Exp $+}
%
\begin{abstract}
%fixme:= incluir referencia a PETSc
This is the documentation for \pfem (current version 
{\tt petscfem-\petscfemversion}, a general purpose, parallel,
multi-physics FEM program for CFD applications based on
PETSc. \pfem{} comprises both a library that allows the user to
develop FEM (or FEM-like, i.e. non-structured mesh oriented) programs, and
a suite of application programs. It is written in the C++ language
with an OOP (Object Oriented Programming) philosophy, but always
keeping in mind the scope of efficiency. 
\end{abstract}
%
\newpage
\tableofcontents
%
% OK. This is tricky. :-) 
% I comment and uncomment the following '\inputs' while testing,
% so that we risk to process this file with some parts commented out
% while processing for the final version. So that, before processing 
% we run from the Makefile a one-liner Perl script ($ perl -pi.bak -e
% 's/...') that uncomment this lines.
%
\input gpl            %__UNCOMMENT_IN_FINAL_VERSION__
\input prev1          %__UNCOMMENT_IN_FINAL_VERSION__
\input layout         %__UNCOMMENT_IN_FINAL_VERSION__
\input thash          %__UNCOMMENT_IN_FINAL_VERSION__
\input idmap          %__UNCOMMENT_IN_FINAL_VERSION__
\input compprof       %__UNCOMMENT_IN_FINAL_VERSION__
\input advec          %__UNCOMMENT_IN_FINAL_VERSION__
 \input advdif         %__UNCOMMENT_IN_FINAL_VERSION__
\input ns             %__UNCOMMENT_IN_FINAL_VERSION__
\input tests          %__UNCOMMENT_IN_FINAL_VERSION__
\input fastmat2       %__UNCOMMENT_IN_FINAL_VERSION__
%
% These have to be commented out
%%%% THIS IS TO TRY OUT THE AUXILIARY FILES FOR DOC OPTIONS %%%%%%
%\input trygetopt.tex %__COMMENT_IN_FINAL_VERSION__
%%%%%%%%%%%%%%%%%%%%%%%%%%%%%%%%%%%%%%%%%%%%%%%%%%%%%%%%%%%%%%%%%%

\Section{Symbols and Acronyms}

\SSection{Acronyms} 
\begin{description}
\item[CFD:] Computational Fluid Dynamics
\item[eperl:] embedded Perl
\item[FDM:] Finite Difference Method
\item[FEM:] Finite Element Method
\item[MPI:] Message Passing Interface
\item[OOP:] Object Oriented Programming
\item[Perl:] Practical Extraction and Report Language
\item[PETSC:] Portable Extensible Toolkit for Scientific computations.
\item[SUPG:] Streamline Upwind/Petrov Galerkin
\item[ANN:] Approximate Nearest Neighbor problem. Also refers to the
library developped by David Mount and Sunil Arya
(\url{http://www.cs.umd.edu/~mount/ANN}). 

\end{description}

\Section{Symbols} 

\begin{itemize}
   \item $u,v=$ Streamwise and normal components of velocity.
\end{itemize}

\newpage
\bibliography{petscfem}

\begin{flushleft}
\printindex
\end{flushleft}

\end{document}

%Section{

%**start of math-def
\documentclass{article}
\usepackage{mstorti,amsmath}
\pagestyle{empty} 
\begin{document}
\begin{equation} 
%**end of math-def

%# Current line ===========  
%:eps:uosc:
u(t)\propto \sin(\omega t)
%:eps:tau2:
2\tau
%:eps:ts:
t_s
%:eps:t0:
t_0
%:eps:t1:
t_1
%:eps:phi:
\phi
%:eps:bphi:
\bar\phi
%:eps:phi0:
\phi_0
%:eps:phi1:
\phi_1
%:eps:hfilm:
h_{\mathrm{film}}
%:eps:tinf:
T_\infty
%:eps:ts:
T_s
%:eps::

%<*>---<*>---<*>---<*>---<*>---<*>---<*>---<*>---<*>---<*>---<*> 

"DONE
====
%# Current line ===========  
# Local Variables: $
# mode: outline-minor $
# eval: (setq outline-regexp "\\(\\\\\\|%\\)S*ection{") $
# eval: (setq current-pos-regexp "^%# Current line ===========") $
# eval: (font-lock-fontify-buffer) $
# End: $
