%__INSERT_LICENSE__
\Section{The compute\_prof package} 

\SSection{MPI matrices in PETSc} 

\index{compute_prof package@\verb+compute_prof+ package}
%
When defining a matrix in PETSc with ``\verb+MatCreateMPIAIJ+'' you
tell PETSc (see the PETSc documentation)
%
\begin{itemize}
\item How many lines of the matrix live in the actual processoor ``\verb+m+''. 
\item For each line of this processor how many non null elements it
has in the diagonal block ``\verb+d_nnz[j]+''
\item For each line of this processor how many non null elements it
has in the off-diagonal block ``\verb+o_nnz[j]+''
\end{itemize}
%
\index{COMP_MAT_PROF@\verb+COMP_MAT_PROF+}
%
In \pfem{} this quantities are computed in two steps. 
\begin{itemize}
\item Call assemble with ``\verb+ijob=COMP_MAT_PROF+'' or
``\verb+ijob=COMP_FDJ_PROF+'' and appropriated ``\verb+jobinfo+'' in
order to tell the element routine which matrix are you computing.
This returns a Libretto dynamic array ``\verb+*da+'' containing the
sparse structure of the given matrix.

\index{compute_prof@\verb+compute_prof+}
%
\item function ``\verb+compute_prof+'' takes ``\verb+da+'' as argument
and fills ``\verb+d_nnz+'', ``\verb+o_nnz+''. 
\end{itemize} 

\SSection{Profile determination} 

\index{profile determination}
%
Let us describe first how the sparse profile is determined in the
sequential (one processor only) case. We have to determine for each
row index $i$ all the column indices $j$ that have a non null matrix
entry $A_{ij}$. The amount of storage needed is a mattern oof concern
here. It is allmost sure that we will need at least one integer per
each non-null entry. A first attempt is to have a dynamic array for
each $i$ index and store their all the connected $j$ indices. In
typical applications we have $O(\nexpE4)$ to $O(\nexpE5)$ nodes per
processor and the number of connected $j$ indices ranges from 10 to
several hundred. In order to avoid this large number of small dynamic
arrays growing we store all the indices in a big array, behaving as a
sinly linked list. Each entry in
the array is composed of two integers (``\verb+struct Node+''), one
pointing to the next entry for this row, and other with the value of
the row. Consider, for instance, a matrix with the following sparse
structure: 

% Consider, for instance, a mesh like that one in
% figure~\ref{fg:profil}. We assume one field per node, natural boundary
% conditions everywhere and no renumbering of unknowns, so that node
% number is the same as unknown number in the matrix. The 
% element number in the figure shows also the order in which elements
% are loaded. 

\begin{equation} 
  !A = 
\begin{sqmat}{ccc}
      * & * & * \\    
      . & * & . \\    
      * & . & * 
\end{sqmat} 
\end{equation}

Matrix coefficients $(i,j)$ are introduced in the following order (1,1),
(2,2), (1,2), (3,1), (1,3), (3,3).  The dynamic array for this mesh
results in

\begin{figure*}[ht]
\centerline{\includegraphics{./OBJ/profil}}
\caption{Structure of darray}
\label{fg:profil}
\end{figure*}

For all $i$, the $i$-th row starts in position $i-1$. The first
component of the pair shows the value of $j$ and the position of the
next connected $j$ index. The sequence ends with a terminator
$(0,-1)$. In this way, the storage needed is two integers per
coefficient with an overhead of two integers per row for the
terminator, and there is only one dynamic array growing at the same
time. 

To insert a new coefficient, say $(i,j)$, we traverse the $i$-th row,
checking whether the $j$ coefficient is already present or not, and if
the terminator is found, $j$ is inserted in its place, pointinf to the
new terminator, which is placed at the end of the dynamic array.

