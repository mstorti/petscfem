%__INSERT_LICENSE__
\Section{Hooks}

Hooks are functions that you pass to the program and, then, are
executed at particular points, called \emph{``hook-launching
points''}, in the execution of the program. The particular
hook-launching points may depend on the application but, in order to
fix ideas, for the Navier-Stokes module and the Advective-Diffusive
module the standard hooks are:
%
\begin{itemize}
\item{\tt init:} To be executed once, at the start of the program. 
\item{\tt time\_step\_pre:} To be executed before the time step
calculation. 
\item{\tt time\_step\_post:} To be executed after the time step
calculation. 
\item{\tt close:} To be executed once, at the end of the program. 
\end{itemize}
%
There are \emph{``built-in''} hooks included in the modules, for
instance the DX hook that is in charge of communicating with the DX
visualizations program or the \emph{``shell-hook''} that allows you to
execute shell commands, but you can also define your own hooks that
are defined in a C++ piece of code, compiled and dynamically loaded at
run-time. You can do almost anything with your hooks, for instance you
can compress the result files, perform file manipulation, launch
visualization with other software like GMV. Also, hooks are useful
for communicating between different instances of PETSc-FEM. For
instance, if you want to couple a inviscid external flow with an
internal viscous flow, then you can run a PETSc-FEM instance for each
region and perform the communication between the different regions
with hooks. The DX hook is explained in the DX section
(see~\S\ref{sec:dx}) and we will explain here how to write and use
dynamically loaded hooks and the shell hook. 

The hook concept has been borrowed from GNU Emacs.

%<*>---<*>---<*>---<*>---<*>---<*>---<*>---<*>---<*>---<*>---<*> 
\SSection{Launching hooks. The hook list} 

In order to activate a hook you first have to add a for each hook a
pair of strings to the \verb+hook_list+, namely the type pf hook and
the name of the hook. This last one is a unique identifier that makes
that hook unique. 
%
\begin{alltt} 
hook_list \Arg{hook-type-1} \Arg{hook-name-1} \Arg{hook-type-2} \Arg{hook-name-2} ...
\end{alltt}
%
For instance:
%
\begin{verbatim}
hook_list shell_hook compress     \
          dx_hook    my_dx_hook   \
          dl_hook    coupling_hook
\end{verbatim}
%
Here we added a shell hook that probably will compress some files
during execution, the DX hook in order to visualize, and a dynamically
linked hook that will couple the run with another program. Each hook
will after take their own options from special options in the table. 

The \verb+hook_list+ entry must be unique, so that you have to group
all your hooks in a \emph{single} \verb+hook_list+ entry. 
This is not limiting, because you can add as much hooks as you want,
but it is rather syntactically cumbersome, because you end up with a
long string. Also it becomes difficult to comment out some hooks while keeping
others. 

The hooks are executed in the order as you entered them in the hook
list, so that in the previous case you will have, at init time, the
\verb+init+ part of the \verb+compress+ hook to be executed
\emph{before} the init part of the \verb+my_dx_hook+ and finally the
init part of the \verb+coupling_hook+.

%<*>---<*>---<*>---<*>---<*>---<*>---<*>---<*>---<*>---<*>---<*> 
\SSection{Dynamically loaded hooks} 

The easiest way to code a dynamically loadable hook is with a
class. You need to include the corresponding headers {\tt hook.h} and
{\tt dlhook.h}, and the class may define the hook functions for all,
some or none of the hook-launching points.  Consider for instance the
following ``Hello world!'' hook, that prints the message at the
corresponding points in the program.

\begin{verbatim}
#include <src/hook.h>
#include <src/dlhook.h>

class hello_world_hook {
public:
  void init(Mesh &mesh_a,Dofmap &dofmap,
	    TextHashTableFilter *options,const char *name) {
    printf("Hello world! I'm in the \"init\" hook\n");
  }
  void time_step_pre(double time,int step) {
    printf("Hello world! I'm in the \"time_step_pre\" hook\n");
  }
  void time_step_post(double time,int step,
		      const vector<double> &gather_values) {
    printf("Hello world! I'm in the \"time_step_post\" hook\n");
  }
  void close() {
    printf("Hello world! I'm in the \"close\" hook\n");
  }
};

DL_GENERIC_HOOK(hello_world_hook);
\end{verbatim}

You can use almost any conceivable C/C++ library within your hooks. Take
into account that the program may be called in a parallel environment so,
for instance, if you will compress a certain file, then you should
take care of doing that \emph{only} at the master process by, e.g.,
enclosing the code with a \verb+if (!!MY_RANK) { ... }+ construct. 

%<*>---<*>---<*>---<*>---<*>---<*>---<*>---<*>---<*>---<*>---<*> 
\SSection{Shell hook} 

\index{shell-hook} The \emph{shell-hook} allows the user to execute a
certain action at the hook-launching points by simply writing shell
commands.
%
\begin{alltt}
hook_list shell_hook \Arg{name}
\Arg{name} \Arg{shell-command} 
\end{alltt}
%
For instance
%
\begin{verbatim}
hook_list shell_hook hello
hello "echo Hello world"
\end{verbatim}
%
In this case, PETSc-FEM will issue the echo command at each of the
launching points. If you want to issue more complex commands, then
perhaps it's a better idea to bundle them in a script and then execute
the script from the hook:
%
\begin{verbatim}
hook_list shell_hook hello
hello "my_script_hook"
\end{verbatim}
%
where you have previously written a \verb+my_script_hook+ script with
something like
%
\begin{verbatim}
#!/bin/bash
## This is `my_script_hook' file

echo Hello world
\end{verbatim}
%
inside. 

Probably you want to perform some actions depending on which stage you
are, so that you can pass the stage name to the command by including a
\verb+%s+ 
output conversion token in the command. For instance
%
\begin{verbatim}
hello "echo Hello world, stage %s"
\end{verbatim}
%
Moreover, you can have also the time step currently executing and the
current simulation time by including a \verb+%d+
and a \verb+%f+ 
output conversions, for instance
%
\begin{verbatim}
hello "echo Hello world, stage %s, step %d, time %f"
\end{verbatim}
%
The order is important!! That is, the first argument is the step (a
C string), the second the time step (an integer) and the last the
simulation time (a double). 

In fact, basically, what PETSc-FEM does is to build string with
\verb+sprintf()+ and then execute it with \verb+system()+ like
%
\begin{alltt}
sprintf(command,your_command,stage,step,time);
system(command);
\end{alltt}
%
(see the Glibc manual for more info about sprintf and system). If
you need, for some reason to switch the order then use a parameter
number like
%
\begin{verbatim}
hello "echo Hello world, time %3$f, step %2$d, stage %1$s"
\end{verbatim}

If you want to do some things depending on the stage then perhaps you
can write something like this
%
\begin{verbatim}
hook_list shell_hook hello
hello "my_script_hook"
\end{verbatim}
%
and
%
\begin{verbatim}
#!/bin/bash
## File `my_script_hook'

if [ "$1" == "init" ]
then 
    echo "in init"
    ## Do more things in `init' stage
    ## ....
elif [ "$1" == "pre" ] 
then 
    echo "in pre"
    ## Do more things in `pre' stage
    ## ....
elif [ "$1" == "post" ]
then 
    echo "in post"
    ## Do more things in `post' stage
    ## ....
elif [ "$1" == "close" ]
then 
    echo "in close"
    ## Do more things in `close' stage
    ## ....
else 
    ## Catch all. Should not enter here. 
    echo "Don't know how to handle stage: $1"
fi
\end{verbatim}
%
At the \verb+init+ and \verb+close+ hook-launching points the step
number passed is -1 and -2 respectively, so that you can detect
whether you are in a pre/post stage or init/close by checking this
too. The time passed is in both cases 0.

%<*>---<*>---<*>---<*>---<*>---<*>---<*>---<*>---<*>---<*>---<*> 
\SSection{Shell hooks with ``make''} 

If no command is given, i.e. if you write
%
\begin{verbatim}
hook_list shell_hook hello
\end{verbatim}
%
but don't add the
%
\begin{alltt}
hello \Arg{command}  
\end{alltt}
%
line, then PETSc-FEM uses a standard command line like this
%
\begin{verbatim}
make petscfem_step=%2$d petscfem_time=%3$f hello_%1$s 
\end{verbatim}
%
so that it will execute make commands with targets 
\verb+hello_init+, \verb+hello_pre+, \verb+hello_post+
and \verb+hello_close+ like
%
\begin{verbatim}
$ make petscfem_step=-1 petscfem_time=0. hello_init
$ make petscfem_step=1 petscfem_time=0.1 hello_pre
$ make petscfem_step=1 petscfem_time=0.1 hello_post
$ make petscfem_step=2 petscfem_time=0.2 hello_pre
$ make petscfem_step=2 petscfem_time=0.2 hello_post
$ make petscfem_step=3 petscfem_time=0.3 hello_pre
$ make petscfem_step=3 petscfem_time=0.3 hello_post
...
$ make petscfem_step=100 petscfem_time=10. hello_pre
$ make petscfem_step=100 petscfem_time=10. hello_post
$ make petscfem_step=-2 petscfem_time=0. hello_close
\end{verbatim}
%
Inside the \verb+Makefile+ you can use the \verb+make+ variables
\verb+$(petscfem_step)+ and \verb+$(petscfem_time)+. For instance you
can do the \emph{``Hello world''} trick by adding the targets
%
%
\begin{verbatim}
# In Makefile
hello_init:
        echo "In init"
        ## Do more things in `init' stage
        ## ....

hello_pre:  
        echo "In pre"
        ## Do more things in `pre' stage
        ## ....

hello_post:  
        echo "In post"
        ## Do more things in `post' stage
        ## ....

hello_close:  
        echo "In close"
        ## Do more things in `close' stage
        ## ....
\end{verbatim}
%
For instance, I love to gzip my state files with a command like this
%
\begin{verbatim}
## In the PETSc-FEM data file
hook_list shell_hook compress

## In the Makefile
compress_init:
compress_pre:
compress_post:
        for f in *.state.tmp ; do echo "gzipping $f" ; gzip -f $f ; done
compress_close:
\end{verbatim}


% Local Variables: *
% mode: latex *
% tex-main-file: "petscfem.tex" *
% End: *

