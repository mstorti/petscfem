% $Id: femref.tex,v 1.2 2004/11/26 21:44:57 mstorti Exp $
\SSection{Mesh refinement} \label{sec:femref}
%<*>---<*>---<*>---<*>---<*>---<*>---<*>---<*>---<*>---<*>---<*>

\emph{[Warning: This is a work in progress.]}

We conceive a mesh as a graph, connecting nodes and other higher
dimension entities based on these nodes. Consider for example five
nodes (labeled from 0 to 40 and an edge connecting two of these nodes
as in figure~\ref{fg:meshgraph2}. In total, there are 6
\emph{``geometrical objects''} (5 nodes and the edge) in the
figure. In order to identify these higher order objects, we could add
a new index for it, say index 5, but instead we can associate the edge
with the connected pair of node indices, \verb+(edge 2 3)+. Note that,
if we consider that the edge has no orientation, then the sequence
must be considered as a \emph{``set''},
i.e. \verb+(edge 2 3) = (edge 3 2)+, (Lisp-like \emph{``S-exp''}
expressions will be used throughout in order to describe objects)
whereas if oritentation matters, then
\verb+(edge 2 3) != (edge 3 2)+. We could, then define geometrical
objects of type \verb+edge+ as unordered sequences of two node
indices, while the type \verb+ordered-edge+ is associated with ordered
sequences of two nodes. We say that the set of permutations that leave
invariant \verb+edge+ objects is \verb+((0 1) (1 0))+, whereas for the
oriented edge is only the identity permutation \verb+((0 1))+. For
larger objects, the set of permutations that leave invariant the node
sequence that defines the object is more complex than that, it doesn't
reduce to the special case of ordered and unordered sequences as for
edges. Consider for instance the case of triangles. For an oriented
triangle, the set of nodes that leave invariant the triangle is
\verb+((0 1 2) (1 2 0) (2 0 1))+, i.e. shifts clockwise and
counter-clockwise of the node sequence, whereas for unordered objects
the sequences are the same 
\verb+((0 1 2) (1 2 0) (2 0 1) (0 2 1) (1 0 2) (2 1 0))+. 
