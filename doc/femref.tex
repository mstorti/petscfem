% $Id: femref.tex,v 1.8 2004/11/27 19:27:20 mstorti Exp $
\SSection{Mesh refinement} \label{sec:femref}
%<*>---<*>---<*>---<*>---<*>---<*>---<*>---<*>---<*>---<*>---<*>

\emph{[Warning: This is a work in progress.]}

%<*>---<*>---<*>---<*>---<*>---<*>---<*>---<*>---<*>---<*>---<*> 
\begin{figure*}[htb]
\centerline{\includegraphics[scale=1.3]{./OBJ/meshgraph}}
\caption{FEM mesh and graph representation}
\label{fg:meshgraph}
\end{figure*}
%<*>---<*>---<*>---<*>---<*>---<*>---<*>---<*>---<*>---<*>---<*> 

%<*>---<*>---<*>---<*>---<*>---<*>---<*>---<*>---<*>---<*>---<*> 
\begin{figure*}[htb]
\centerline{\includegraphics[scale=1.3]{./OBJ/meshgraph2}}
\caption{Geomtrical objets: 5 nodes and an edge}
\label{fg:meshgraph2}
\end{figure*}
%<*>---<*>---<*>---<*>---<*>---<*>---<*>---<*>---<*>---<*>---<*> 

We conceive a mesh as a graph, connecting nodes and other higher
dimension entities based on these nodes. Consider for example five
nodes (labeled from 0 to 40 and an edge connecting two of these nodes
as in figure~\ref{fg:meshgraph2}. In total, there are 6
\emph{``geometrical objects''} (5 nodes and the edge) in the
figure. In order to identify higher order objects like the edge, we
could add a new index for it, say index 5, but instead we can
associate the edge with the connected pair of node indices,
\verb+(edge 2 3)+. Note that, if we consider that the edge has no
orientation, then the sequence must be considered as a \emph{``set''},
i.e. \verb+(edge 2 3)+ $=$ \verb+(edge 3 2)+, (Lisp-like
\emph{``S-exp''} expressions will be used throughout in order to
describe objects) whereas if oritentation matters, then
\verb+(edge 2 3)+ $\neq$ \verb+(edge 3 2)+. We could, then define
geometrical objects of type \verb+edge+ as unordered sequences of two
node indices, while the type \verb+ordered-edge+ is associated with
ordered sequences of two nodes. We say that the set of permutations
that leave invariant \verb+edge+ objects is \verb+((0 1) (1 0))+,
whereas for the oriented edge is only the identity permutation
\verb+((0 1))+. For larger objects, the set of permutations that leave
invariant the node sequence that defines the object is more complex
than that, it doesn't reduce to the special case of ordered and
unordered sequences as for edges. Consider for instance the case of
triangles. For an oriented triangle, the set of nodes that leave
invariant the triangle is \verb+((0 1 2) (1 2 0) (2 0 1))+,
i.e. shifts clockwise and counter-clockwise of the node sequence,
whereas for unordered objects the sequences are the same
\verb+((0 1 2) (1 2 0) (2 0 1) (0 2 1) (1 0 2) (2 1 0))+. In the last
case, the permutations that leave invariant the object are the whole
set of 6 permutations for 3 objects, so that in this case we can say
that unordered triangles can be represented as unordered sets of three
nodes. But for the unordered triangle edge \verb+(a b c)+ is the same
that \verb+(b c a)+, so that associating unordered triangles with
unordered sequences does not take into account the rotational
symmetry. 

\SSSection{Symmetry group generator} 

The set of permutations \verb+(perm SHAPE)+ that leave the geometrical
object \verb+SHAPE+ invariant are a \emph{``group''}, that means that
if two permutations $p$, $q$ belong to \verb+(perm SHAPE)+,
then the composition \verb+pq+, i.e. the permutation resulting of
applying first $q$ and then $p$, also belong to the group,
as well as the product \verb+qp+. In general, this is a finite group.

As a consequence, if we have two elements $p$ and $q$ of a group $G$,
then $pq$, $qp$, $p^2q$, $pqp$, $pq^2$... all belong to $G$. Given a
set of permutations $S\subset G$, the set $pS$ formed by the
\emph{left} product of all the elements of $S$ with $p$ belongs to
$G$. The same applies to $Sp$, $qS$ and $Sq$. So that, starting with
the set $S_0=\{1\}$, where $1$ is the identity permutation, we can
generate recursively a larger set of symmetries by forming
$S_1=\{S_0,pS_0,S_0p,qS_0,S_0q\}$. As the total number of permutations
in the group is finite (at most $n!$, where $n$ is the number of nodes
in the object), applying recursively the relation above will end with
a set of permutations $H$ that is itself a group included in $G$. We
will call it the group \emph{``generated''} by $p$ an $q$. This can be
applied to any number of generators $p$, $q$, $r$, $s$, .... 

For instance, the symmetries for the oriented triangle can be
generated with the permutation $p=$\verb+(1 2 0)+, since
\verb+(2 0 1)+ can be generated as $p^2$. The symmetries for the
unoriented triangle can be derived from generators \verb+(1 2 0)+
(rotation) and \verb+(0 2 1)+ (inversion). The most relevant
geometrical objects and their symmetries are

\begin{itemize}
\compactlist 

%<*>---<*>---<*>---<*>---<*>---<*>---<*>---<*>---<*>---<*>---<*> 
\begin{figure*}[htb]
\centerline{\includegraphics[scale=1.3]{./OBJ/trigen}}
\caption{Generators for triangle}
\label{fg:trigen}
\end{figure*}
%<*>---<*>---<*>---<*>---<*>---<*>---<*>---<*>---<*>---<*>---<*> 

\item Unoriented triangles are described by numbering their nodes in
  any way. Their symmetries are all the permutations ($6=2!$ en total)
  and are generated by a rotation and an inversion.

\item Oriented triangles are described by numbering their nodes in a
  specified direction. Their symmetries are 3 en total generated by a
  rotation.

%<*>---<*>---<*>---<*>---<*>---<*>---<*>---<*>---<*>---<*>---<*> 
\begin{figure*}[htb]
\centerline{\includegraphics[scale=1.3]{./OBJ/quadgen}}
\caption{Generators for quadrangle}
\label{fg:quadgen}
\end{figure*}
%<*>---<*>---<*>---<*>---<*>---<*>---<*>---<*>---<*>---<*>---<*> 

\item Unoriented quadrangles are described by numbering their nodes in
  clockwise or counterclockwise rotation. Symmetry generators: are
  \verb+(1 2 3 0)+ (rotation) and \verb+(0 3 2 1)+ (inversion), there
  is a total of 8 permutations. 

\item Oriented quadrangles are identical to unoriented quadrangles but
  do not include the inversion (4 rotations). 

%<*>---<*>---<*>---<*>---<*>---<*>---<*>---<*>---<*>---<*>---<*> 
\begin{figure*}[htb]
\centerline{\includegraphics[scale=1.3]{./OBJ/tetragen}}
\caption{Generators for tetras}
\label{fg:tetragen}
\end{figure*}
%<*>---<*>---<*>---<*>---<*>---<*>---<*>---<*>---<*>---<*>---<*> 

\item Unoriented tetrahedra are described by numbering one of the
  faces, and then the oposite node. Symmetries are generated by
  permutation of any two of the faces, for instance \verb+(1 2 0 3)+
  and \verb+(3 0 2 1)+ and inversion \verb+(1 0 2 3)+ (24 permutations
  in total).

\item Oriented tetrahedra is identical to unoriented tetrahedra
  without inversion (24 permutations
  in total). 

%<*>---<*>---<*>---<*>---<*>---<*>---<*>---<*>---<*>---<*>---<*> 
\begin{figure*}[htb]
\centerline{\includegraphics[scale=1.3]{./OBJ/hexagen}}
\caption{Generators for hexas}
\label{fg:hexagen}
\end{figure*}
%<*>---<*>---<*>---<*>---<*>---<*>---<*>---<*>---<*>---<*>---<*> 

 \item Unoriented hexas are described by numbering one of the faces as
  a quad, and then the opposite face in correspondence with the first
  face. Symmetries are generated by $90\degree$ rotations for any two
  of the faces (not opposite), for instance \verb+(1 2 3 0 5 6 7 0)+
  and \verb+(1 5 6 2 0 4 7 3)+ and inversion \verb+(1 0 2 3)+ (232
  permutations in total).

\item Oriented hexahedra is identical to unoriented hexahedra
  without inversion (112 permutations in total). 

%<*>---<*>---<*>---<*>---<*>---<*>---<*>---<*>---<*>---<*>---<*> 
\begin{figure*}[htb]
\centerline{\includegraphics[scale=1.3]{./OBJ/prismgen}}
\caption{Generators for prisms}
\label{fg:prismgen}
\end{figure*}
%<*>---<*>---<*>---<*>---<*>---<*>---<*>---<*>---<*>---<*>---<*> 

 \item Unoriented prisms are described by numbering one of the
  triangular faces as a triangle, and then the opposite face in
  correspondence with the first face. Symmetries are generated by
  $180\degree$ rotations of any two of the quad faces
  (\verb+(4 3 5 1 0 2)+ for instance), a $120\degree$ rotation of any
  of the triangular face (\verb+(1 2 0 4 5 3)+ for instance) and an
  inversion (\verb+(1 0 2 4 3 5)+ for instance). There are 12
  permutations in total. 

\item Oriented hexahedra is identical to unoriented hexahedra
  without inversion (6 permutations in total). 

\end{itemize}
 
%<*>---<*>---<*>---<*>---<*>---<*>---<*>---<*>---<*>---<*>---<*>
\SSSection{Canonical ordering} 

One of the most common operations when manipulating these geometrical
objects is, given two geometrical objects of the same type, to
determine whether they represent the same object. The brute force
solution is to apply all the permutations to one of them and check if
one of the permuted indices coincide with the other node sequence. 

%<*>---<*>---<*>---<*>---<*>---<*>---<*>---<*>---<*>---<*>---<*>
\SSection{Permutation tree} 

In order to make it more efficient we can store all the permutations
for a given shape in a tree like fashion. Consider, for instance, the
oriented tetrahedral shape. Their permutations are the following:

\begin{verbatim}
(0 1 2 3)
(0 2 3 1)
(0 3 1 2)
(1 0 3 2)
(1 2 0 3)
(1 3 2 0)
(2 0 1 3)
(2 1 3 0)
(2 3 0 1)
(3 0 2 1)
(3 1 0 2)
(3 2 1 0)
\end{verbatim}
%
We can describe the generation of a new node numbering in the
following way. First, we can take any of the nodes as the first node
of the new numbering. These is seen from the fact that all the indices
(0 to 3) are present in the first column of the table above. If we
chose node 2 as the first index, then we can chose any of the
reamining as the second node (0 1 3). Once we choose the second index
(say for instance 1) there is only one possibility for the reamining
two: the numbering \verb+(3 1 0 2)+. The remaining possibility
\verb+(3 1 2 0)+ would gnerate an inverted triangle. Part of the tree,
is shown in ~\ref{fg:tetratree}. Every possible permutation is a
\emph{``path''} in the tree from the first node at level 0, to the
last node, which is a leave. 

%<*>---<*>---<*>---<*>---<*>---<*>---<*>---<*>---<*>---<*>---<*> 
\begin{figure*}[htb]
\centerline{\includegraphics[scale=1.3]{./OBJ/tetratree}}
\caption{Tree representing all the permutations for the ordered tetra
  geometry. }
\label{fg:tetratree}
\end{figure*}
%<*>---<*>---<*>---<*>---<*>---<*>---<*>---<*>---<*>---<*>---<*> 

Now, given two possible node orderings for a given geometrical object,
and the tree that describes the permutations for its shape, we simply
have to follow the path that makes one of them to fit in the other. If
we arrive to an internal node without possibility to follow, then the
geometrical objects are distinct. If we reach a leave, then the
objects are the same. 

%<*>---<*>---<*>---<*>---<*>---<*>---<*>---<*>---<*>---<*>---<*>
\SSection{Canonical ordering} 

Another possibility to determine whether two node oderings are
congruent is to have a uniquely determined ordering that can be
computed from the node sequence itself. We we call this ordering
the \emph{``canonical''} ordering for the geometrical object. If this
is possible, then given two orderings we can bring both of them to the
canonical ordering and then compare them as plain sequences. One
possibility is to take the caonical order as that one that gives the
lower node sequence, in lexicographical order. 

This has the advantage that one the canonical orderign is known for
both objects, the comparison is very cheap. Also, it can be used 
as a comparison operator for sorting geometrical objects, i.e. the
order between two objects is the lexicographical order between the two
node sequences in its canonical form. 

The canonical order can be computed efficientlyusing the permutation
tree described above. 

%<*>---<*>---<*>---<*>---<*>---<*>---<*>---<*>---<*>---<*>---<*>
\SSection{Object hashing} 

Another useful technique for comparing objects is by comparing first
some scalar function of the node indices. For instance we can compute
a \emph{``hash-sum''} for the object ${\tt go}$ as
%
\begin{equation} 
  H({\tt go}) = \sum_{j=0}^{n-1} h({\tt (node go j)})
\end{equation
%
where \verb+(node go j)+ is the \verb+j+ node of object \verb+go+. $H$
is the hash-sum for the geometrical object \verb+go+, whereas $h$ is a
\emph{``scalar hashing function''}. A very simple possibility is to
take $h$ as the identity $h(x)=x$. In that case the hash of the object
is simply the sum of their node indices. The idea is that, even for
this simple hashing function we can compare first the hash-sums of two
objects for determining if they are equal. If there is a high
probability of the objects being unequal, then in most cases this
simple check will suffice. If the hash-sum are equal, then the full
comparision procedure, as described above, must be performed. Note
that the hash-sum is (and must be) independent of the ordering. 

In order to reduce the probability that unequal objects give the same
hash-sum we can device better hashing functions. 

 
