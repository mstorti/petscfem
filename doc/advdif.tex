
\Section{The general advective-diffusive elemset} 

\SSection{Introduction to advective systems of equations} 

Advective system of equations are of the form
%
\begin{equation} 
 \dep{U_\mu}t + \dep{\F^\conv_{i\mu}(U)}{x_i} = 
             \dep{\F^\diff_{i\mu}(U,\nabla U)}{x_i} + G_\mu
\end{equation}
%
and its numerical treatment is similar to the advective elemset. 

%<*>---<*>---<*>---<*>---<*>---<*>---<*>---<*>---<*>---<*>---<*> 
\SSSection{Options}

\begin{description}
%<*>---<*>---<*>---<*>---<*>---<*>---<*>---<*>---<*>---<*>---<*> 
\item[General options:] ~

\begin{itemize}
\input odocadvdif
\end{itemize}

%<*>---<*>---<*>---<*>---<*>---<*>---<*>---<*>---<*>---<*>---<*> 
\item[Generic elemset ``{\tt advdife}'':] ~

\begin{itemize}
\input odocadvdife
\end{itemize}

%<*>---<*>---<*>---<*>---<*>---<*>---<*>---<*>---<*>---<*>---<*> 
\item[Flux function``{\tt advec}'':] Generic linear
advective-diffusive system

All fluxes and \(G_\mu\) are linear in \(!U\) and its gradients

\begin{equation}
\begin{split}
  \F^\conv_{i\mu} &= A_{j\mu\nu} U_\nu \\
  \F^{\diff}_{i\mu} &= \D_{ij\mu\nu}\, \prt_j U^\nu\\
   G_\mu &= R_{\mu\nu} U_\nu + S_\nu
\end{split}
\end{equation}

\begin{itemize}
\input odocadvfm2
\end{itemize}

%<*>---<*>---<*>---<*>---<*>---<*>---<*>---<*>---<*>---<*>---<*> 
\item[Flux function``{\tt burgers}'':] Burgers eqs.

\begin{itemize}
\input odocburgers
\end{itemize}

\SSection{Logarithmic variables} 

In many problems some quantities are intrinsically positive (pressure,
density) but, do to numerical inaccuracies (mainly numerical spurious
oscillations) they can acquire null or negative variables. This in
turn can cause a breakdown of the general algorithm, since this
quantities may appear in the denominator of some expressions, or as
the argument of functions which only accept positive values (as the 
\verb+log()+ \verb+sqrt()+ functions. 

A very typical case is the $k-\epsilon$ turbulence model, where both
quantities are intrinsically positive but vary strongly near walls,
and is very common to reach negative values and cause the breakdown of
the code. 

One common source of oscillations is the use of very small time steps
in a diffusive equation (as the heat equation) with consistent mass
matrix. The problem disappears if a lumped mas matrix is used. Another
source is a lack of stabilization of advective terms. The problem is
solved by using a better stabilization scheme or increasing the amount
of stabilization. 

Even with a careful stabilization scheme and time integration, 
it is very common to have code breakdown and so there is interest in
develop numerical schemes that, due to some special tratment of time
or spatial discretization guarantee positivity. 

One possibility is to map the variable space with a logarithmic
transformation. For instance, if we have an ODE of the form
%
\begin{equation} 
   \dot u = f(u,t)
\end{equation}
%
If this equation is discretized with the trapezoidal rule integration
method
%
\begin{equation} 
  \frac{u^{n+1}-u^n}{\Dt} = f(u^{n+\alpha},t^{n+\alpha})
\end{equation}
%
where $t^n=n\Dt$, $t^{n+\alpha}=t^n+\alpha\Dt$, $0\le\alpha\le1$, then
it may happen that $u$ gets negative values, even if $u$ is guaranteed
to be strictly positive in the continuous case. Now consider the
change of variable
%
\begin{equation} 
  u = \ze^\cU
\end{equation}
%
then, the transformed equation is
%
\begin{equation} 
   u \, \dot\cU = f(u,t)
\end{equation}
%
and the trapezoidal discretization is
%
\begin{equation} 
   u^{n+\alpha} \, \frac{\cU^{n+1}-\cU^{n}}{\Dt} = 
             f(u^{n+\alpha},t^{n+\alpha})
\end{equation}



\end{description}


%
% Local Variables: *
% mode: latex *
% tex-main-file: "petscfem.tex" *
% End: *
