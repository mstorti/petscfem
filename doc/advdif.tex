%__INSERT_LICENSE__


%---:---<*>---:---<*>---:---<*>---:---<*>---:---<*>---:---<*>---: 
\SSection{The hydrology module} \label{sec:hydro}

%<*>---<*>---<*>---<*>---<*>---<*>---<*>---<*>---<*>---<*>---<*> 
\begin{figure*}[htb]
\centerline{\includegraphics{./OBJ/aquifer.\EPSPDF}}
\caption{Aquifer/stream system. }
\label{fg:aquifer}
\end{figure*}
%<*>---<*>---<*>---<*>---<*>---<*>---<*>---<*>---<*>---<*>---<*> 

%<*>---<*>---<*>---<*>---<*>---<*>---<*>---<*>---<*>---<*>---<*> 
\begin{figure*}[htb]
\centerline{\includegraphics{./OBJ/stream.\EPSPDF}}
\caption{Aquifer/stream system. Transverse 2D view}
\label{fg:stream}
\end{figure*}
%<*>---<*>---<*>---<*>---<*>---<*>---<*>---<*>---<*>---<*>---<*> 

%<*>---<*>---<*>---<*>---<*>---<*>---<*>---<*>---<*>---<*>---<*> 
\begin{figure*}[htb]
\centerline{\includegraphics{./OBJ/aquist.\EPSPDF}}
\caption{Aquifer/stream system. Discretization.}
\label{fg:aquist}
\end{figure*}
%<*>---<*>---<*>---<*>---<*>---<*>---<*>---<*>---<*>---<*>---<*> 

This module solves the problem of subsurface flow in a free aquifer,
coupled with a surface net of 1D streams. To model such system three
elemsets must be used: an \verb+aquifer+ system representing the
subsurface aquifer, a \verb+stream+ elemset representing the 1D stream
and a \verb+stream_loss+ elemset representing the losses from the
stream to the aquifer (or vice versa) see figures~\ref{fg:aquifer} 
and~\ref{fg:stream}.

The \verb+aquifer+ elemset is a 2D elemset with triangle or quadrangle
elements (see figure~\ref{fg:aquist}). A per-element property
\verb+eta+ represents the height of the aquifer bottom to a given
datum. The corresponding unknown for each node is the piezometric
height or the level of the freatic surface at that point $\phi$. On
the other hand, the \verb+stream+ elemset represents a 1D stream of
water. It has its own nodes, separate from the aquifer nodes, whose
coordinates must coincide with some corresponding node in the
aquifer. For instance, the triangular aquifer element in the figure is connected
to nodes $n1$, $n2$ and $n3$, while the stream element is connected to
nodes $n4$ and $n5$. $n1$ and $n5$ have the same coordinates (but
different unknowns) and also $n2$ and $n4$. 
A node constant field (so called ``H-fields'') represents the
stream bottom height $h_b$, with reference to the datum. So that,
normally, we have for each node two coordinates and the stream bottom
height. ({\tt ndim=2 nu=3 ndof=1}).  The unknown for these nodes is
the height $u$ of the stream free water surface with reference to the
stream bottom. The channel shape and friction model and coefficients
are entered via properties described below. If the stream level is
above the freatic aquifer level ($h_b+u > \phi$) then the stream
losses water to the aquifer and vice versa.

The equation for the aquifer integrated in th verical direction is
%
\begin{equation} 
  \dep{}{t} \LL(S(\phi-\eta)\phi\RR) =
  \nabla\cdot(K(\phi-\eta)\nabla\phi) + \sum G_a
\end{equation}
%
where $S$ is the storativity and $G$ is a source term, due to rain,
losses from streams or other aquifers. 

The equation for the stream is, according to the \emph{``Kinematic
Wave Model'' KWM} approach (\cite{rodriguez}),
%
\begin{equation} 
   \dep {A(u)}t + \dep {Q(A(u))}{s} = G_s
\end{equation}
%
Where $u$ is the unknown field that represents the height of the water
in the channel with respect to the channel bottom as a function of
time and a linear arc coordinate along the stream, $A$ is the
transverse cross section of the stream and depends, through the
geometry of the channel, on the channel water height $u$. $Q$ is the
flow rate and, under the KWM model is a function only of $A$
through the friction law. 
%
\begin{equation} 
   Q = \gamma A^m
\end{equation}
%
where $\gamma=C_h\,S^{1/2}\,P\muno$ and $m=\sfr32$ for the Ch\`ezy
friction model, and $\gamma=\bar a\,n\muno\,S^{1/2}\,P^{-2/3}$ and
$m=\sfr53$ for the Manning model, where $S = \dtots{h_b}{s}$ is the
slope of the stream bottom, $P$ is the wetted perimeter, and $C_h$,
$\bar a$ and $n$ are model constants. $G_s$ represent the gain or loss
of the stream, and the main component is the loss to the aquifer
%
\begin{equation} 
  G_s = P/R_f (\phi-h_b-u)
\end{equation}
%
where $R_f$ is the resistivity factor per unit arc length of the
perimeter. The corresponding gain to the aquifer is 
%
\begin{equation} 
  G_a = -G_s \, \delta_{\Gs}
\end{equation}
%
where $\Gs$ represents the planar curve of the stream and $\delta_\Gs$
is a Dirac's delta distribution  with a unit intensity per unit
length, i.e.
%
\begin{equation} 
  \int f(!x) \, \delta_\Gs \dS = \int_0^L f(!x(s)) \di{s}
\end{equation}
%
The \verb+stream_loss+ elemset represents this loss, and a typical
discretization is shown in figure~\ref{fg:aquist}. The stream loss
element is connected to two nodes on the stream  and two on the
aquifer and must be entered in that order in the element connectivity
table, for instance
%
\begin{alltt} 
elemset stream_loss 4
\Arg{.... elemset properties...}
__END__HASH__
...
\Arg{n5} \Arg{n4} \Arg{n1} \Arg{n2}
...
__END__ELEMSET__
\end{alltt}

\SSSection{Related Options}

\begin{itemize}
\input odocstream
\end{itemize}

%
% Local Variables: *
% mode: latex *
% tex-main-file: "petscfem.tex" *
% End: *
