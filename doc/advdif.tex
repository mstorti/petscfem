%__INSERT_LICENSE__

\Section{The general advective-diffusive elemset} 

\SSection{Introduction to advective systems of equations} 

Advective system of equations are of the form
%
\begin{equation} 
 \dep{!H(!U)_\mu}t + \dep{}{x_i}\F^\conv_{i\mu}(!U) +
             \dep{}{x_i}\F^\diff_{i\mu}(!U,\nabla !U) = G_\mu(!U)
\end{equation}
%
and its numerical treatment is similar to the advective elemset.  The
$!H(!U)$ is a one to one relationship between tha \emph{``state
variables''} $!U$ and the \emph{``conservation variables''}
$!H$. $\F^{\diff,\conv}$ are the convective and diffusive fluxes
respectively. We use the SUPG variational formulation
%
\begin{equation} 
\begin{split}
 \int_\Omega N_p (\dep{!H_\mu}t & - G_\mu
        +\cswf \, \dep{}{x_i}\F^\conv_{i\mu} ) \di\Omega + \\
   &   -\int_\Omega  \dep{N_p}{x_i} \, (\swf \F^\conv_{i\mu} 
        + \F^\diff_{i\mu})  \di\Omega  \\
     & + \int_\Gamma (\swf \F^\conv_{i\mu} 
        + \F^\diff_{i\mu}) n_i \di\Gamma \\
     & + \sum_e \int_{\Omega_e} 
     \taufac \, !P_{ep} \, \LL\{\bsupg \dep{!H_\mu}t + 
        \dep{}{x_i}\F_{i\mu}
              - G_\mu \RR\} \di\Omega =0
\end{split}
\end{equation}
%
where
%
\begin{itemize}
\item $N_p$ is the interpolation function for the $p$-th node. 
\item The variational formulation represents an equation for each node
$p$ and equation $\mu$. 
\index{weak_form@\verb+weak_form+}
\item $\swf$ is the \emph{``weak form selector''} (option
\verb+weak_form+).  It may take the values $0,1$ and $!\swf = 1-\swf$
is its complement. It controls the integration by parts (or not) of
the advective term.
\index{beta_supg@\verb+beta_supg+}
\item $0\le\bsupg\le 1$ (option \verb+beta_supg+) controls whether the
temporal term is treated consistently or not.
\item $P_{ep}(!x)$ is the SUPG \emph{``perturbation function''} for node
$p$ on element $e$. For systems, as it is here, it may be a matrix of
size $\ndof\times\ndof$. 
\item The integral on the boundary is computed in elemsets of type
\verb+bcconv+, whereas the rest are computed in elemset of type
\verb+advdif+. 
\end{itemize}

\SSection{Flux jacobians}

The \emph{``flux jacobians''} are the derivatives of the fluxes to the
different components:

\begin{itemize}
\item The \emph{``advective jacobians''} are
%
\begin{equation}
A_{i\mu\nu} = \dep{}{U_\nu} {\F^\conv_{i\mu}}
\end{equation}
%
The projection of the jacobian on a given direction $!n$ is 
%
\[
  (!A\cdot!n)_{\mu\nu} = A_{i\mu\nu} n_i
\]
%
By the theory of advective systems of equations it is assumed that the
projection of the jacobians on any direction $!n$ must be a
diagonalizable matrix with real eigenvalues.

\item The \emph{``diffusive jacobians''} are 
%
\begin{equation}
D_{ij\mu\nu} = -\dep{}{(\prt_j U_\nu)} {\F^\diff_{i\mu}}
\end{equation}
%
It is assumed that the diffusive jacobians are symmetric possitive definite if
seen as a square matrix of row/column dimension of \(\ndim\ndof\). 

\item The \emph{``reactive jacobians''} are
%
\[ R_{\mu\nu} = -\dep{U_\nu}{G_\mu} \]
%
Normally, it is assumed that the eigenvalues have positive real part,
otherwise an exponential grow may happen. 

\item The \emph{``enthalpy jacobians''} or \emph{``capacitance
matrix''} is
%
\[   C_{\mu\nu} = \dep{U_\nu}{H_\mu}  \]

\end{itemize}

\SSection{The flux function class} 

The particular relations for each physical problem are passed through
a \emph{``flux function''} (class \verb+AdvDifFF+) object whose
abstract interface is the following

\begin{verbatim}
class EnthalpyFun {
public:
  virtual void update(const double *) {};
  virtual void enthalpy(FastMat2 &H, FastMat2 &U)=0;
  virtual void comp_W_Cp_N(FastMat2 &W_Cp_N,FastMat2 &W,FastMat2 &N,
			   double w)=0;
  virtual void comp_P_Cp(FastMat2 &P_Cp,FastMat2 &P_supg)=0;
};

class AdvDifFF {
private:
  // ...
public:
  const Elemset *elemset;
  EnthalpyFun *enthalpy_fun;
  AdvDifFF(Elemset *elemset_=NULL) 
    : elemset(elemset_), enthalpy_fun(NULL) {};
  virtual void start_chunk(int &ret_options) =0;
  virtual void element_hook(ElementIterator &element) =0;
  virtual void comp_A_grad_N(FastMat2 & A,FastMat2 & B)=0;
  virtual void comp_A_jac_n(FastMat2 &A_jac_n, FastMat2 &normal)=0;
  virtual void comp_grad_N_D_grad_N(FastMat2 &grad_N_D_grad_N,
				    FastMat2 & dshapex,double w) =0 ;
  virtual void compute_flux(COMPUTE_FLUX_ARGS) =0;
  virtual void get_log_vars(int &nlog_vars,const int *& log_vars);
  virtual void comp_N_N_C(FastMat2 &N_N_C,FastMat2 &N,double w)=0;
  virtual void comp_N_P_C(FastMat2 &N_P_C, FastMat2 &P_supg,
			  FastMat2 &N,double w)=0;
};
\end{verbatim}

The class contains a number of members that define the different
fluxes and jacobians. Instead of returning explicitly the full
jacobian, the interface allows to only define the action of the
jacobian on other matrices. This may improve efficiency when the
jacobians are sparse or null. For instance the function
\verb+comp_A_grad_N(FastMat2 & A_grad_N,FastMat2 & grad_N)+ should
return the product (\verb+A_grad_N+)$_{p\mu\nu}$ = $A_{i\mu\nu}\,
\deps{N_p}{x_i}$ of size $\nel\times\ndof\times\ndof$ on output when
passed a matrix (\verb+grad_N+)$_{ip} = \deps{N_p}{x_i}$ of size
$\ndim\times\nel$.

%<*>---<*>---<*>---<*>---<*>---<*>---<*>---<*>---<*>---<*>---<*> 
\SSSection{Options}

\begin{description}
%<*>---<*>---<*>---<*>---<*>---<*>---<*>---<*>---<*>---<*>---<*> 
\item[General options:] ~

\begin{itemize}
\input odocadvdif
\end{itemize}

%<*>---<*>---<*>---<*>---<*>---<*>---<*>---<*>---<*>---<*>---<*> 
\item[Generic elemset ``{\tt advdife}'':] ~

\begin{itemize}
\input odocadvdife
\end{itemize}

%<*>---<*>---<*>---<*>---<*>---<*>---<*>---<*>---<*>---<*>---<*> 
\item[Flux function``{\tt advec}'':] Generic linear
advective-diffusive system

All fluxes and \(G_\mu\) are linear in \(!U\) and its gradients

\begin{equation}
\begin{split}
  \F^\conv_{i\mu} &= A_{j\mu\nu} U_\nu \\
  \F^{\diff}_{i\mu} &= -\D_{ij\mu\nu}\, \prt_j U^\nu\\
   G_\mu &= -R_{\mu\nu} U_\nu + S_\nu
\end{split}
\end{equation}

\begin{itemize}
\input odocadvfm2
\end{itemize}

%<*>---<*>---<*>---<*>---<*>---<*>---<*>---<*>---<*>---<*>---<*> 
\item[Flux function``{\tt burgers}'':] Burgers eqs.

\begin{itemize}
\input odocburgers
\end{itemize}

\SSection{Logarithmic variables} 

In many problems some quantities are intrinsically positive (pressure,
density) but, do to numerical inaccuracies (mainly numerical spurious
oscillations) they can acquire null or negative variables. This in
turn can cause a breakdown of the general algorithm, since this
quantities may appear in the denominator of some expressions, or as
the argument of functions which only accept positive values (as the 
\verb+log()+ \verb+sqrt()+ functions. 

A very typical case is the $k-\epsilon$ turbulence model, where both
quantities are intrinsically positive but vary strongly near walls,
and is very common to reach negative values and cause the breakdown of
the code. 

One common source of oscillations is the use of very small time steps
in a diffusive equation (as the heat equation) with consistent mass
matrix. The problem disappears if a lumped mas matrix is used. Another
source is a lack of stabilization of advective terms. The problem is
solved by using a better stabilization scheme or increasing the amount
of stabilization. 

Even with a careful stabilization scheme and time integration, 
it is very common to have code breakdown and so there is interest in
develop numerical schemes that, due to some special tratment of time
or spatial discretization guarantee positivity. 

One possibility is to map the variable space with a logarithmic
transformation. For instance, if we have an ODE of the form
%
\begin{equation} 
   \dot u = f(u,t)
\end{equation}
%
If this equation is discretized with the trapezoidal rule integration
method
%
\begin{equation} 
  \frac{u^{n+1}-u^n}{\Dt} = f(u^{n+\alpha},t^{n+\alpha})
\end{equation}
%
where $t^n=n\Dt$, $t^{n+\alpha}=t^n+\alpha\Dt$, $0\le\alpha\le1$, then
it may happen that $u$ gets negative values, even if $u$ is guaranteed
to be strictly positive in the continuous case. Now consider the
change of variable
%
\begin{equation} 
  u = \ze^\cU
\end{equation}
%
then, the transformed equation is
%
\begin{equation} 
   u \, \dot\cU = f(u,t)
\end{equation}
%
and the trapezoidal discretization is
%
\begin{equation} 
   u^{n+\alpha} \, \frac{\cU^{n+1}-\cU^{n}}{\Dt} = 
             f(u^{n+\alpha},t^{n+\alpha})
\end{equation}



\end{description}


%
% Local Variables: *
% mode: latex *
% tex-main-file: "petscfem.tex" *
% End: *
