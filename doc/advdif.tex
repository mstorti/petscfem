
\Section{The general advective-diffusive elemset} 

\SSection{Introduction to advective systems of equations} 

Advective system of equations are of the form
%
\begin{equation} 
 \dep{U}t + \dep{\F^\conv_i(U)}{x_i} = \dep{\F^\diff_i(U,\nabla U)}{x_i} + G
\end{equation}
%
and its treatment is similar to the advective elemset. 

FIXME:= TO BE DOCUMENTED LATER

%<*>---<*>---<*>---<*>---<*>---<*>---<*>---<*>---<*>---<*>---<*> 
\SSSection{Options}

\begin{description}
%<*>---<*>---<*>---<*>---<*>---<*>---<*>---<*>---<*>---<*>---<*> 
\item[General options:] ~

\begin{itemize}
\input odocadvdif
\end{itemize}

%<*>---<*>---<*>---<*>---<*>---<*>---<*>---<*>---<*>---<*>---<*> 
\item[Generic elemset ``{\tt advdife}'':] ~

\begin{itemize}
\input odocadvdife
\end{itemize}

%<*>---<*>---<*>---<*>---<*>---<*>---<*>---<*>---<*>---<*>---<*> 
\item[Flux function``{\tt advec}'':] Generic linear
advective-diffusive system 

\begin{itemize}
\input odocadvfm2
\end{itemize}

%<*>---<*>---<*>---<*>---<*>---<*>---<*>---<*>---<*>---<*>---<*> 
\item[Flux function``{\tt burgers}'':] Burgers eqs.

\begin{itemize}
\input odocburgers
\end{itemize}


\end{description}


%
% Local Variables: *
% mode: latex *
% tex-main-file: "petscfem.tex" *
% End: *
