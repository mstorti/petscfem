%__INSERT_LICENSE__
% $Id: dx.tex,v 1.6 2003/11/23 00:03:39 mstorti Exp $
\Section{Visualization with DX} \label{eq:dx}

Data Explorer (\url{http://www.opendx.org}) is a system of tools and
user interfaces for visualizing scientific data. Originally an IBM
commercial product, has been released now under the IBM Open Source
License, and maintained by a group of volunteers (see the URL
above). Besides their impressive visualization capabilities, DX has
many features that make it an ideal visualization tool for PETSc-FEM. 

\begin{itemize}
\item DX is Open Source with a License very close to the GPL (not
completely compatible though)
\item It has a \emph{``Visual Program Editor''} which makes it very
configurable for different modules. 
\item It has been linked to PETSc-FEM through sockets, which makes it
possible to visualize a running job, even in background. 
\item It has a scripting language. 
\end{itemize}

If you want to visualize your results with DX you have first to
download it from the URL above and install it. Then you can pass your
results to DX simply by editing the needed {\tt .dx} files or well by
using the {\tt ExtProgImport} DX module. In order to use this last
option you have to

\begin{itemize}
\item Compile PETSc-FEM with the \verb+USE_SSL+ flag enabled (disabled
      by default). Also, if you want to DX be able to communicate asynchronously
       with PETSc-FEM, you have to compile with the
       \verb+USE_PTHREADS+ flag enabled (disabled by default)
\item Load the \verb+dx_hook+ hook in PETSc-FEM and pass it some
options. 
\item Build the dynamically loadable module (file \verb+dx/epimport+) and load it in DX.
\end{itemize}

DX basic visualization units are \verb+Field+ objects, which are
composed of three \verb+Array+ objects called \verb+positions+ (node
coordinates), \verb+connections+ (element connectivities) and
\verb+data+ (computed field values). At each time step, \verb+ExtProgImport+ exports two
\verb+Group+ objects, 

\begin{itemize}
\item a \verb+Group+ of \verb+Arrays+ named \verb+output_array_list+
and 
\item a \verb+Group+ of \verb+Fields+ objects named
\verb+output_fields_list+. 
\end{itemize}

This objects are generated as follows. 

\begin{itemize}
\item For each  \verb+Nodedata+ PETSc-FEM object a \verb+positions+ array is
constructed, this results in, say,  \verb+nn+ array
objects. (Currently PETSc-FEM has only one 
\verb+Nodedata+ object (member name \verb+nodes+), so that \verb+NN=1+). 
\item A \verb+data+ array is constructed in basis to the current state
vector (member name \verb+data+). 
\item For each elemset, a \verb+connections+ array is constructed. You
can disable construction of some particular elemsets through the
\verb+dx+ elemset option, and also which nodes and DX interpolation geometry is
used. This results in other \verb+ne+ connection arrays. The member name of
each array is based in the \verb+name+ option of the elemset. If this
has not been set, the name is set to the type of the elemset. If
collision is produced, a suffix of the form \verb+_0+, \verb+_1+ is
appended, in order to make it unique. Options controlling how the
connection array is constructed can be consulted in
\S\ref{sec:elemset-opt}. 
\end{itemize}

The resulting \verb|nn+ne+1| arrays are grouped in a \verb+Group+
object and sent through the \verb+output_array_list+ output tab. You
can extract the individual components with the \verb+Select+ DX
module, and build field objects. Also a set of field objects is
created automatically and sent through the \verb+output_field_list+
output tab. Basically a field is constructed for each possible
combination of positions, connections and data objects. This may seem
a huge amount of fields, but in fact as the arrays are passed
internally by pointers in DX, the additional memory requirements are
not large. (At the time of writing this, this is a set of
\verb+nn*ne=ne+ fields, since \verb+nn=1+). A name is generated
automatically for each field. 

Some information is send to the DX \emph{``Message Window''}. Also 
it's very useful to put \verb+Print+ modules downstream of the
\verb+ExtProgImport+ module in order to see which arrays and fields
have been created. 

The communication between DX and PETSc-FEM is done through a
socket. PETSc-FEM acts as a \emph{``server''} whereas DX acts as a
\emph{``client''}. PETSc-FEM opens a \emph{``port''} (option
\verb+dx_port+), and DX connects to that port (the \emph{``port''}
input tab in the {\tt ExtProgImport} module). (Currently, the standard
port for the DX/PETSc-FEM communication is 5314.) DX can communicate
with PETSc-FEM running in the background ad even on other machine (the
\emph{``serverhost''} input tab).  At each time step, DX sends a
request to PETSc-FEM which answers sending back \emph{``arrays''} and
\emph{``fields''}.

%<*>---<*>---<*>---<*>---<*>---<*>---<*>---<*>---<*>---<*>---<*> 
\SSection{Asynchronous/synchronous communication}

\begin{itemize}
\item In \emph{``synchronous''} mode ({\tt steps>0}) PETSc-FEM waits each
{\tt steps} time steps a connection from DX. Once DX connects to the
port, PETSc-FEM transmits the required data and resumes
computation. This is the appropriate way of communication when
generating a sequence of frames for a video with a DX sequencer, for
instance. Note that if you don't use a sequencer then you have to
arrange in someway to make \verb+ExtProgImport+ awake  and connect to
PETSc-FEM, otherwise  the update in the visualization is not performed
and the PETSc-FEM job is stopped, waiting for the connection. 

\item In \emph{``asynchronous''} mode ({\tt steps=0}), in contrast,
PETSc-FEM monitors each port after computing a time step. If a DX
client is trying to connect, it answers the request and resumes computing,
otherwise it resumes computing immediately. This is ideal for monitor a
job that is running in background, for instance. Note that in this
case the interference with the PETSc-FEM job is minimal, since once
PETSc-FEM answers the request, resumes processing automatically until
a new connection is requested. 
\end{itemize}

The {\tt steps} state variable is internal to PETSc-FEM. It can be set
initially with a \verb+dx_steps+ options line (1 by default). After
that, it can be changed by changing the {\tt steps} input
tab. However, note that the change doesn't take effect until the next
connection of DX to PETSc-FEM.  If you don't want to change the
internal state of the \verb+steps+ variable then you can set it to
\verb+NULL+ or \verb+-1+.

%<*>---<*>---<*>---<*>---<*>---<*>---<*>---<*>---<*>---<*>---<*> 
\SSection{Building and loading the ExtProgImport module}

This module allows PETSc-FEM to exchange data with DX through a
socket, using a predefined protocol. The module is in the {\tt
\$(PETSCFEM\_DIR)/dx} directory of the PETSc-FEM distribution.  To
built it, you have to compile first the {\tt petscfem} library, and
then {\tt cd} to the {\tt dx} directory and make {\tt \$ make}. This
should build the {\tt dx/epimport} file, which is a dynamically
loadable module for DX. This one altogether with the {\tt
dx/epimport.mdf} (which is a plain text file describing the
inputs/outputs of the module, and other things) are the files needed
by DX in order to run the module. 

To load this module in DX you can
do this either at the moment of launching DX with something like
%
\begin{alltt}
\$ dx -mdf /path/to/epimport.mdf 
\end{alltt}
%
or well from the {\tt dxexec} window (menu {\tt File/Load Module
Descriptions}). 

%<*>---<*>---<*>---<*>---<*>---<*>---<*>---<*>---<*>---<*>---<*> 
\SSection{Inputs/outputs of the ExtProgImport module} 

\begin{itemize}
\item (input) \verb+steps+: type {\tt integer}; default {\tt
           0}. {\sl Description:}  Visualize each "steps" time steps. (0 means
           asynchronously). 
\item (input) \verb+serverhost+; type {\tt string}; default \verb+"localhost"+; 
         {\sl Description:} Name of host where the external program is
         running. May be also set in dot notation
         (e.g. \verb+200.9.223.34+ or  \verb+myhost.mydomain.edu+).  
\item (input)  \verb+port+; type {\tt integer}; default {\tt 5314}; 
       {\sl Description:}Port number
\item (input) \verb+options+; type {\tt string}; default {\tt NULL}; 
     {\sl Description:} Options to be passed to the external program
\item (input) \verb+step+; type {\tt integer}; default {\tt -1}; 
        {\sl Description:} Input for sequencer. This value is passed
        to the \verb+dx_hook+ hook, but currently is ignored by it. It
        could be used in a future in order to synchronize the DX
        internal step number with the PETSc-FEM one. 
\item (input) \verb+state_file+; type {\tt string}; default {\tt
       NULL}; An ASCII file storing a state where to read the state to be
     visualized. 
\item (output) \verb+ output_array_list+; type {\tt field}; 
        {\sl Description:} Group object of imported arrays
\item (output) \verb+output_field_list+; type {\tt field}; 
                {\sl Description:} Group object of imported fields
\end{itemize}

%<*>---<*>---<*>---<*>---<*>---<*>---<*>---<*>---<*>---<*>---<*> 
\SSection{DX hook options}

\begin{itemize}
\input odocdxhook
\end{itemize}

%\begin{itemize}
%\end{itemize}

% Local Variables: *
% mode: latex *
% tex-main-file: "petscfem.tex" *
% End: *

